\section{Construction of $n$-Dependent Ladder Operators}

Although the infinite square well does not admit standard Heisenberg
ladder operators (due to the $n^2$ energy spectrum), it is possible
to construct \emph{generalized} creation and annihilation operators.
These operators, $\hat{A}$ and $\hat{A}^\dagger$, connect
consecutive energy levels with coefficients that depend explicitly on
the quantum number $n$ \cite{CohenTanoudji2023, Jalali2013, Curado2001}.

\subsection{Definition on the Spectral Basis}

Let $\{\ket{n}\}_{n \ge 1}$ denote the orthonormal eigenbasis
of the single-particle Hamiltonian,
\begin{equation} \label{eq:H_eigenbasis}
  \hat{H}_0 \ket{n} = E_n \ket{n}, \qquad E_n =
  \frac{\pi^2\hbar^2}{2mL^2}n^2.
\end{equation}
We define the generalized ladder operators by their matrix elements,
which explicitly connect adjacent states:
\begin{equation}
  \hat{A}^\dagger = \sum_{n=1}^{\infty} c_n \ket{n+1}\bra{n}, \qquad
  \hat{A} = \sum_{n=1}^{\infty} c_n^* \ket{n}\bra{n+1}.
\end{equation}
Their action on the eigenbasis is therefore
\begin{equation}
  \hat{A}^\dagger \ket{n} = c_n \ket{n+1}, \qquad
  \hat{A} \ket{n} = c_{n-1}^* \ket{n-1} \quad (\text{for } n > 1),
\end{equation}
and $\hat{A}\ket{1} = 0$, as $\ket{1}$ is the ground state.
The coefficients $c_n$ are chosen to match the spectral structure.

\subsection{Commutation with the Hamiltonian}

We can find the commutation relation between $\hat{H}_0$ and
$\hat{A}^\dagger$ by applying it to an arbitrary state $\ket{n}$:
\begin{align}
  [\hat{H}_0, \hat{A}^\dagger] \ket{n}
    &= (\hat{H}_0 \hat{A}^\dagger - \hat{A}^\dagger \hat{H}_0) \ket{n}
      \nonumber \\
    &= \hat{H}_0 (c_n \ket{n+1}) - \hat{A}^\dagger (E_n \ket{n})
      \nonumber \\
    &= c_n E_{n+1} \ket{n+1} - E_n (c_n \ket{n+1}) \nonumber \\
    &= (E_{n+1} - E_n) \hat{A}^\dagger \ket{n}.
\end{align}
In contrast to the harmonic oscillator, where $E_{n+1} - E_n$ is a
constant, here the difference depends explicitly on $n$:
\begin{equation}
  E_{n+1} - E_n = \frac{\pi^2\hbar^2}{2mL^2} \left( (n+1)^2 - n^2 \right)
  = \frac{\pi^2\hbar^2}{2mL^2}(2n+1).
\end{equation}
This confirms that no constant $\lambda$ exists to satisfy
$[\hat{H}_0,\hat{A}^\dagger] = \lambda \hat{A}^\dagger$.
Instead, the commutator takes the \emph{functional form}
\begin{equation} \label{eq:deformed_comm_H}
  [\hat{H}_0,\hat{A}^\dagger] = F(\hat{N})\,\hat{A}^\dagger,
\end{equation}
where $\hat{N}$ is the number operator ($\hat{N}\ket{n} = n\ket{n}$)
and $F(\hat{N})$ is the nonlinear step operator
\begin{equation}
  F(\hat{N}) = \frac{\pi^2\hbar^2}{2mL^2}\,(2\hat{N}+1).
\end{equation}
This structure defines a deformed Heisenberg algebra,
characteristic of systems with variable energy spacing.

\subsection{Algebraic Structure}

The choice of coefficients $c_n$ defines the specific algebra.
A common choice, motivated by the factorization method,
is to relate $c_n$ to the energy spectrum.
This choice also defines the second commutator, $[\hat{A}, \hat{A}^\dagger]$.
A simple application of the operators shows
\begin{align}
  \hat{A}\hat{A}^\dagger \ket{n} &= \hat{A}(c_n \ket{n+1})
    = c_n c_n^* \ket{n} = |c_n|^2 \ket{n}, \\
  \hat{A}^\dagger\hat{A} \ket{n} &= \hat{A}^\dagger(c_{n-1}^* \ket{n-1})
    = c_{n-1}^* c_{n-1} \ket{n} = |c_{n-1}|^2 \ket{n}.
\end{align}
Therefore, the commutator is a diagonal operator $G(\hat{N})$:
\begin{equation}
  [\hat{A}, \hat{A}^\dagger] \ket{n} = (|c_n|^2 - |c_{n-1}|^2) \ket{n}
  \equiv G(\hat{N}) \ket{n}.
\end{equation}
The specific forms of $F(\hat{N})$ and $G(\hat{N})$ are linked by the
choice of $c_n$ and form a consistent algebraic description.

\subsection{Two-Particle Operators}

This formalism extends to the two-particle system.
Given the additive Hamiltonian $\hat{H} = \hat{H}_1 + \hat{H}_2$,
we define particle-local operators:
\begin{equation}
  \hat{A}_1^\dagger = \hat{A}^\dagger \otimes \hat{I}, \qquad
  \hat{A}_2^\dagger = \hat{I} \otimes \hat{A}^\dagger.
\end{equation}
To map the antisymmetric subspace $\mathcal{H}_A$ to itself,
the total operator must be symmetric under particle exchange
(i.e., it must commute with the exchange operator $\hat{P}_{12}$).
We therefore construct the symmetric combination:
\begin{equation}
  \hat{A}_{\text{sym}}^\dagger = \frac{1}{\sqrt{2}}
  \left(\hat{A}_1^\dagger + \hat{A}_2^\dagger\right).
\end{equation}
Its action on an antisymmetric state $\ket{\Psi_{n_1, n_2}} =
\frac{1}{\sqrt{2}}(\ket{n_1, n_2} - \ket{n_2, n_1})$ is:
\begin{align}
  \hat{A}_{\text{sym}}^\dagger \ket{\Psi_{n_1, n_2}}
    = \frac{1}{2} & \left[
      c_{n_1}(\ket{n_1+1, n_2} - \ket{n_2, n_1+1}) \right. \nonumber \\
    & \left. + c_{n_2}(\ket{n_1, n_2+1} - \ket{n_2+1, n_1})
      \right].
\end{align}
This resulting state is a superposition of two new antisymmetric
eigenstates, $\ket{\Psi_{n_1+1, n_2}}$ and $\ket{\Psi_{n_1, n_2+1}}$,
confirming that $\hat{A}_{\text{sym}}^\dagger$ correctly
acts as a ``creation operator'' within the fermionic subspace.

\subsection{Conclusion}

Generalized ladder operators $\hat{A}$ and $\hat{A}^\dagger$
for the infinite square well do exist. They act consistently
on the eigenbasis but form a nonlinear, or ``deformed,'' algebra:
\begin{equation}
  [\hat{H}_0,\hat{A}^\dagger] = F(\hat{N})\hat{A}^\dagger, \qquad
  [\hat{A},\hat{A}^\dagger] = G(\hat{N}).
\end{equation}
This construction aligns with the generalized Heisenberg algebras
developed in \cite{CohenTanoudji2023, Jalali2013, Curado2001},
where systems with nonlinear spectra (like the infinite well or
Pöschl–Teller potentials) are successfully described using
$n$-dependent ladder operators.
