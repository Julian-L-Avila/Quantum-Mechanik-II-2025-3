\section{Momentum Representation of the Schr\"odinger Equation} \label{sec:momentum}

It is well-known that transforming problems with hard-wall boundaries
to the momentum representation is inherently challenging.
The difficulty arises not from the potential, but from the
kinetic operator.

We seek the TISE by applying the 2D Fourier transform,
defined over the finite domain $x_i \in [0, L]$:
\begin{equation} \label{eq:fourier-transform}
	\tilde{\psi}(p_1, p_2) = \frac{1}{2\pi \hbar} \int_0^L \int_0^L \psi(x_1, x_2)
	e^{-i (p_1 x_1 + p_2 x_2)/ \hbar} \, \mathrm{d}x_1 \mathrm{d}x_2.
\end{equation}
As in the position-space analysis (\cref{eq:V_int_expectation}),
the fermionic antisymmetry ensures $\psi(x, x) = 0$. Consequently,
the interaction term $\hat{V}_{\text{int}} = g\delta(x_1 - x_2)$
has a null contribution, and the TISE in momentum space simplifies to
$\mathcal{F}[\hat{T}\psi] = E \tilde{\psi}$.

The primary challenge is the kinetic operator. The Fourier transform
of a second derivative over a finite domain, $\mathcal{F}[\partial_x^2 \psi]$,
does not simply map to $-(p^2/\hbar^2) \tilde{\psi}(p)$.
Instead, integration by parts introduces boundary terms.
For simplicity, consider the 1D kinetic operator $\hat{T} =
-\frac{\hbar^2}{2m}\partial_x^2$. Its Fourier transform is
\begin{equation} \label{eq:ft_kinetic_1d}
	\mathcal{F}[\hat{T}\psi](p) = \frac{p^2}{2m} \tilde{\psi}(p)
	- \frac{\hbar^2}{2m} \left( e^{-i p L/\hbar} \psi'(L) - \psi'(0) \right),
\end{equation}
where $\psi'(x) \equiv \partial_x \psi(x)$ and we have used the
Dirichlet conditions $\psi(0) = \psi(L) = 0$.

Generalizing to our 2D system, the TISE in momentum space
becomes a complex integral equation. The transform of the kinetic term
$\mathcal{F}[(\hat{T}_1 + \hat{T}_2)\psi]$ introduces terms dependent
on the (unknown) derivatives of the wavefunction at all four boundaries
($x_1=0, x_1=L, x_2=0, x_2=L$).
\begin{equation}
	\begin{split}
		\mathcal{F}[(\hat{T}_1 + \hat{T}_2)\psi] = \frac{1}{2m}(p_1^2 + p_2^2)
		\tilde{\psi}(p_1, p_2) &- \frac{\hbar^2}{2m} \left( e^{-i p_1 L / \hbar} \partial_1 \psi(L, x_2) -
		\partial_2 \psi(0, x_2) \right) - \\
													 &- \frac{\hbar^2}{2m} \left(e^{-i p_2 L / \hbar} \partial_2 \psi(x_1, L) - \partial_2 \psi(x_1, 0) \right)
		\end{split}
\end{equation}

This TISE becomes an integral equation whose kernel depends on these
unknown boundary values (e.g., $\partial_{1}\psi(L, x_2)$).
Solving this is notoriously difficult and requires advanced techniques,
such as Green's functions or treating the infinite well as the
limit of a finite potential. Given these complexities, which
obscure the simple physics derived in the position representation,
we will not pursue the momentum-space solution further.
