\section{Momentum-Space Wavefunction}

We now explicitly construct the momentum-space wavefunction,
$\tilde{\Psi}_{n_1, n_2}(p_1, p_2, t)$, by applying the
Fourier transform defined in \cref{sec:momentum} to our
position-space solution.

For a stationary state, the time evolution is separable.
The position-space solution (with total energy $E \equiv E_{n_1, n_2}$) is
\begin{equation}
  \Psi(x_1, x_2, t) = \psi_{n_1, n_2}(x_1, x_2) \, e^{-iEt/\hbar},
\end{equation}
where $\psi_{n_1, n_2}$ is the spatial eigenfunction from
\cref{eq:spatial_wavefunction}.
The temporal factor $e^{-iEt/\hbar}$ is independent of the spatial
integration and passes directly through the transform.
The task thus reduces to transforming the spatial component.
The full time-dependent momentum-space wavefunction will be
\begin{equation}
  \tilde{\Psi}_{n_1, n_2}(p_1, p_2, t) = \tilde{\psi}_{n_1, n_2}(p_1, p_2)
  e^{-iEt/\hbar}.
\end{equation}

\subsection{Transformation of the Antisymmetric State}

We apply the 2D transform to the fermionic spatial wavefunction
$\psi_{n_1, n_2}(x_1, x_2)$:
\begin{align}
  \tilde{\psi}_{n_1, n_2}(p_1, p_2) &= \mathcal{F}\left[
    \psi_{n_1, n_2}(x_1, x_2) \right] \\
    &= \mathcal{F}\left[ \frac{1}{\sqrt{2}}\left(
    \phi_{n_1}(x_1)\phi_{n_2}(x_2) -
    \phi_{n_2}(x_1)\phi_{n_1}(x_2)\right) \right].
\end{align}
By the linearity of the transform and its separable property,
$\mathcal{F}\{f(x_1)g(x_2)\} = \mathcal{F}_1\{f(x_1)\} \mathcal{F}_2\{g(x_2)\}$,
this calculation simplifies. Let us define
$\tilde{\phi}_{n}(p_i) = \mathcal{F}_i\{\phi_{n}(x_i)\}$ as the 1D
Fourier transform of a single-particle eigenfunction $\phi_n(x)$.
The transform of each term is:
\begin{align}
  \mathcal{F}\{\phi_{n_1}(x_1)\phi_{n_2}(x_2)\} &=
    \tilde{\phi}_{n_1}(p_1) \tilde{\phi}_{n_2}(p_2), \\
  \mathcal{F}\{\phi_{n_2}(x_1)\phi_{n_1}(x_2)\} &=
    \tilde{\phi}_{n_2}(p_1) \tilde{\phi}_{n_1}(p_2).
\end{align}
Substituting these back establishes the structure of the
momentum-space wavefunction:
\begin{equation}
  \tilde{\psi}_{n_1, n_2}(p_1, p_2) = \frac{1}{\sqrt{2}} \left(
  \tilde{\phi}_{n_1}(p_1) \tilde{\phi}_{n_2}(p_2) -
  \tilde{\phi}_{n_2}(p_1) \tilde{\phi}_{n_1}(p_2) \right).
  \label{eq:momentum_wavefunction_structure}
\end{equation}
This derivation confirms that the antisymmetry of the state is
preserved in the momentum representation. The remaining task
is to find the explicit form of $\tilde{\phi}_n(p)$.

\subsection{Single-Particle Momentum Eigenfunction}

The component $\tilde{\phi}_n(p)$ is the 1D transform of the
normalized single-particle eigenfunction.
We proceed with the explicit calculation:
\begin{align} \label{eq:1d_ft_setup}
  \tilde{\phi}_n(p) &= \frac{1}{\sqrt{2\pi\hbar}} \int_{0}^{L}
    e^{-ipx/\hbar} \phi_n(x) \,\mathrm{d}x \nonumber \\
    &= \frac{1}{\sqrt{2\pi\hbar}} \int_{0}^{L} e^{-ipx/\hbar}
    \sqrt{\frac{2}{L}} \sin\left(\frac{n\pi x}{L}\right)
    \,\mathrm{d}x \nonumber \\
    &= \frac{1}{\sqrt{\pi\hbar L}} \int_{0}^{L} e^{-ipx/\hbar}
    \left[ \frac{e^{i n\pi x/L} - e^{-i n\pi x/L}}{2i} \right]
    \,\mathrm{d}x \nonumber \\
    &= \frac{1}{2i\sqrt{\pi\hbar L}} \left[
    \int_{0}^{L} e^{-i(p/\hbar - n\pi/L)x} \,\mathrm{d}x \right. \nonumber \\
    & \qquad \left. - \int_{0}^{L} e^{-i(p/\hbar + n\pi/L)x} \,\mathrm{d}x
    \right].
\end{align}
Defining $k_p = p/\hbar$ (the free-particle wavenumber) and
$k_n = n\pi/L$ (the quantized state wavenumber), the integrals evaluate to:
\begin{align}
  \tilde{\phi}_n(p) &= \frac{1}{2i\sqrt{\pi\hbar L}} \left[
    \frac{1 - e^{-i(k_p - k_n)L}}{i(k_p - k_n)} -
    \frac{1 - e^{-i(k_p + k_n)L}}{i(k_p + k_n)} \right] \nonumber \\
    &= \frac{1}{2\sqrt{\pi\hbar L}} \left[
    \frac{1 - e^{-i(k_p - k_n)L}}{k_p - k_n} -
    \frac{1 - e^{-i(k_p + k_n)L}}{k_p + k_n} \right].
\end{align}
We use the property $k_n L = (n\pi/L)L = n\pi$, which implies
$e^{\pm i k_n L} = e^{\pm i n\pi} = (-1)^n$.
The numerators in both terms are therefore identical:
\begin{equation}
  1 - e^{-i(k_p \mp k_n)L} = 1 - e^{-ik_p L} e^{\pm i k_n L}
  = 1 - e^{-ipL/\hbar} (-1)^n.
\end{equation}
Substituting this common numerator and combining the fractions:
\begin{align}
  \tilde{\phi}_n(p) &= \frac{1 - (-1)^n e^{-ipL/\hbar}}{2\sqrt{\pi\hbar L}}
    \left[ \frac{1}{k_p - k_n} - \frac{1}{k_p + k_n} \right] \nonumber \\
    &= \frac{1 - (-1)^n e^{-ipL/\hbar}}{2\sqrt{\pi\hbar L}}
    \left[ \frac{(k_p + k_n) - (k_p - k_n)}{k_p^2 - k_n^2} \right] \nonumber \\
    &= \frac{1 - (-1)^n e^{-ipL/\hbar}}{2\sqrt{\pi\hbar L}}
    \left[ \frac{2k_n}{k_p^2 - k_n^2} \right].
\end{align}
Simplifying and re-inserting the definitions of $k_n$ and $k_p$,
we obtain the final form:
\begin{equation}
  \tilde{\phi}_n(p) = \frac{1}{\sqrt{\pi\hbar L}}
  \left( \frac{n\pi/L}{(p/\hbar)^2 - (n\pi/L)^2} \right)
  \left[ 1 - (-1)^n e^{-ipL/\hbar} \right].
  \label{eq:single_particle_momentum_wf}
\end{equation}
