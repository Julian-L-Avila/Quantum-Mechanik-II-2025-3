\section{Constructing the Propagator} \label{sec:propagator}

We now calculate the propagator (or kernel), $K$, for the system.
The propagator is the matrix element of the time-evolution operator
in the position basis. For this two-particle system, it is defined as
\begin{equation}
  K(x_1, x_2, t; x'_1, x'_2, t') =
  \bra{x_1, x_2} e^{-i\hat{H}(t-t')/\hbar} \ket{x'_1, x'_2},
\end{equation}
where $\hat{H}$ is the full system Hamiltonian. We define the
elapsed time as $\tau = t-t'$.

\subsection{Spectral Decomposition}

The propagator is computed via its spectral decomposition. This is
achieved by inserting a complete set of energy eigenstates.
For our fermionic system, the complete set for the antisymmetric
subspace is $\hat{I}_A = \sum_{n_1 < n_2} \ket{\Psi_{n_1, n_2}}
\bra{\Psi_{n_1, n_2}}$.

The sum is restricted to $n_1 < n_2$ (or $n_1 > n_2$) to count each
distinct eigenstate exactly once. Inserting this identity, we find
\begin{align} \label{eq:propagator_spectral_sum}
  K(x_f; x_i, \tau)
    &= \bra{x_f} e^{-i\hat{H}\tau/\hbar} \hat{I}_A \ket{x_i} \nonumber \\
    &= \sum_{n_1 < n_2} \bra{x_f} e^{-i\hat{H}\tau/\hbar}
    \ket{\Psi_{n_1, n_2}} \braket{\Psi_{n_1, n_2} | x_i} \nonumber \\
    &= \sum_{n_1 < n_2} e^{-iE_{n_1, n_2}\tau/\hbar}
    \braket{x_f | \Psi_{n_1, n_2}} \braket{\Psi_{n_1, n_2} | x_i}
    \nonumber \\
    &= \sum_{n_1 < n_2} \Psi_{n_1, n_2}(x_1, x_2) \,
    \Psi_{n_1, n_2}^*(x'_1, x'_2) \, e^{-iE_{n_1, n_2}\tau/\hbar},
\end{align}
where $x_f \equiv (x_1, x_2)$, $x_i \equiv (x'_1, x'_2)$,
and $E_{n_1, n_2} = E_{n_1} + E_{n_2}$.
We have used $\braket{\Psi | x} = \braket{x | \Psi}^* = \Psi^*(x)$.

This spectral decomposition is exact. As established previously,
our eigenfunctions $\Psi_{n_1, n_2}$ are the true eigenstates of
the full Hamiltonian $\hat{H} = \hat{H}_0 + g\,\delta(x_1 - x_2)$,
because the interaction term $\hat{V}_{\text{int}}$ vanishes
when applied to any antisymmetric wavefunction.

\subsection{Simplification via Unrestricted Sum}

We now substitute the explicit Slater determinant form for
$\Psi_{n_1, n_2}$. The single-particle states $\phi_n(x)$ are real,
so $\phi_n^* = \phi_n$.
\begin{align}
  K = \sum_{n_1>n_2}
    & \left[ \frac{1}{\sqrt{2}}(\phi_{n_1}(x_1)\phi_{n_2}(x_2) -
    \phi_{n_2}(x_1)\phi_{n_1}(x_2)) \right] \nonumber \\
    & \left[ \frac{1}{\sqrt{2}}(\phi_{n_1}(x'_1)\phi_{n_2}(x'_2) -
    \phi_{n_2}(x'_1)\phi_{n_1}(x'_2)) \right]
    e^{-i(E_{n_1}+E_{n_2})\tau/\hbar}.
\end{align}
The normalization factors combine to give a prefactor of $\frac{1}{2}$.
Let us define $T(n_1, n_2)$ as the unnormalized summand:
\begin{align}
  T(n_1, n_2) =
    & (\phi_{n_1}(x_1)\phi_{n_2}(x_2) - \phi_{n_2}(x_1)\phi_{n_1}(x_2))
    \nonumber \\
    & (\phi_{n_1}(x'_1)\phi_{n_2}(x'_2) - \phi_{n_2}(x'_1)\phi_{n_1}(x'_2))
    \, e^{-i(E_{n_1}+E_{n_2})\tau/\hbar}.
\end{align}
The propagator, as derived from \cref{eq:propagator_spectral_sum}, is
thus written as a restricted sum:
\begin{equation} \label{eq:prop_restricted_sum}
  K = \frac{1}{2} \sum_{n_1>n_2} T(n_1, n_2).
\end{equation}
Our goal is to convert this restricted sum into an unrestricted sum
over all $(n_1, n_2)$, which is separable and easier to evaluate.
The summand $T(n_1, n_2)$ has two key properties:
\begin{enumerate}
  \item \textbf{Vanishing Diagonal:} $T(n, n) = 0$, since the
    determinants become zero.
  \item \textbf{Exchange Symmetry:} $T(n_1, n_2) = T(n_2, n_1)$, since
    swapping $n_1 \leftrightarrow n_2$ negates both determinants
    and $E_{n_1}+E_{n_2}$ is symmetric.
\end{enumerate}
Using these properties, we decompose the full, unrestricted sum:
\begin{align}
  \sum_{n_1, n_2} T(n_1, n_2)
    &= \sum_{n_1 > n_2} T(n_1, n_2) + \sum_{n_1 < n_2} T(n_1, n_2)
    + \sum_{n_1 = n_2} T(n_1, n_2) \nonumber \\
    &= \sum_{n_1 > n_2} T(n_1, n_2) + \sum_{n_1 > n_2} T(n_1, n_2) + 0
    \nonumber \\
    &= 2 \sum_{n_1 > n_2} T(n_1, n_2).
\end{align}
This gives the identity we need:
$\sum_{n_1 > n_2} T(n_1, n_2) = \frac{1}{2} \sum_{n_1, n_2} T(n_1, n_2)$.

Substituting this identity back into our expression for $K$
in \cref{eq:prop_restricted_sum}:
\begin{equation}
  K = \frac{1}{2} \left[ \frac{1}{2} \sum_{n_1, n_2} T(n_1, n_2) \right]
  = \frac{1}{4} \sum_{n_1, n_2} T(n_1, n_2).
\end{equation}
We now expand $T(n_1, n_2)$ into its four constituent terms within this
unrestricted sum:
\begin{align} \label{eq:prop_sum_full_expanded}
  K = \frac{1}{4} \sum_{n_1, n_2} e^{-iE_{n_1}\tau/\hbar} e^{-iE_{n_2}\tau/\hbar}
  \bigg[
      & \phantom{+} \phi_{n_1}(x_1)\phi_{n_2}(x_2)
      \phi_{n_1}(x'_1)\phi_{n_2}(x'_2) \nonumber \\
      & - \phi_{n_1}(x_1)\phi_{n_2}(x_2)
      \phi_{n_2}(x'_1)\phi_{n_1}(x'_2) \nonumber \\
      & - \phi_{n_2}(x_1)\phi_{n_1}(x_2)
      \phi_{n_1}(x'_1)\phi_{n_2}(x'_2) \nonumber \\
      & + \phi_{n_2}(x_1)\phi_{n_1}(x_2)
      \phi_{n_2}(x'_1)\phi_{n_1}(x'_2)
    \bigg].
\end{align}
Let us define the single-particle propagator $K_0$ for the
non-interacting particle in the well:
\begin{equation}
  K_0(x, x', \tau) = \sum_{n=1}^{\infty} \phi_n(x) \phi_n(x')
  e^{-iE_n\tau/\hbar}.
\end{equation}
Each of the four terms in \cref{eq:prop_sum_full_expanded} is a
product of separable sums. For example, the first term is:
\begin{align}
  \text{Term 1} = \frac{1}{4}
    & \left( \sum_{n_1} \phi_{n_1}(x_1)\phi_{n_1}(x'_1)
    e^{-iE_{n_1}\tau/\hbar} \right) \nonumber \\
    & \left( \sum_{n_2} \phi_{n_2}(x_2)\phi_{n_2}(x'_2)
    e^{-iE_{n_2}\tau/\hbar} \right) \nonumber \\
    = \frac{1}{4}
    & K_0(x_1, x'_1, \tau) K_0(x_2, x'_2, \tau).
\end{align}
Applying this separation to all four terms yields:
\begin{align}
  K = \frac{1}{4} \bigg[
      & K_0(x_1, x'_1, \tau) K_0(x_2, x'_2, \tau)
      - K_0(x_1, x'_2, \tau) K_0(x_2, x'_1, \tau) \nonumber \\
  - & K_0(x_2, x'_1, \tau) K_0(x_1, x'_2, \tau)
  + K_0(x_2, x'_2, \tau) K_0(x_1, x'_1, \tau)
\bigg].
\end{align}
The first and fourth terms are identical, as are the second and third.
Combining these terms, we arrive at the final expression:
\begin{align}
  K &= \frac{1}{4} \bigg[ 2 K_0(x_1, x'_1, \tau) K_0(x_2, x'_2, \tau)
  - 2 K_0(x_1, x'_2, \tau) K_0(x_2, x'_1, \tau) \bigg] \nonumber \\
    &= \frac{1}{2} \bigg[ K_0(x_1, x'_1, \tau) K_0(x_2, x'_2, \tau)
    - K_0(x_1, x'_2, \tau) K_0(x_2, x'_1, \tau) \bigg].
  \end{align}
  This is the correct propagator for two indistinguishable (anti-symmetric)
  particles, expressed as half the Slater determinant of the
  single-particle propagators:
  \begin{equation} \label{eq:propagator_slater_det}
    K(x_1, x_2, t; x'_1, x'_2, t') = \frac{1}{2} \det
    \begin{vmatrix}
      K_0(x_1, x'_1, \tau) & K_0(x_1, x'_2, \tau) \\
      K_0(x_2, x'_1, \tau) & K_0(x_2, x'_2, \tau)
    \end{vmatrix}.
  \end{equation}
