\section{Expected Values and Uncertainties}

\subsection{A General Property of Single-Particle Operators}

We now derive a general property for the expectation value of any
single-particle operator, which will greatly simplify all
subsequent calculations.
Let $\hat{A}$ be an operator that acts only on particle 1:
\begin{equation}
  \hat{A} = \hat{a}_1 \otimes \hat{I}_2,
\end{equation}
where $\hat{a}$ is the corresponding single-particle operator.
We calculate the expectation value using the fully normalized
antisymmetric state $\ket{\psi_{n_1, n_2}}$ from \cref{eq:slater_ket}
and the normalized single-particle orbitals $\ket{{\phi}_n}$
(where $\braket{{\phi}_n | {\phi}_m} = \delta_{nm}$).

The expectation value is $\langle \hat{A} \rangle =
\bra{\psi_{n_1, n_2}} \hat{A} \ket{\psi_{n_1, n_2}}$.
Substituting the state definition:
\begin{align}
  \langle \hat{A} \rangle = \frac{1}{2}
    & \left( \bra{{\phi}_{n_1}(1){\phi}_{n_2}(2)} -
      \bra{{\phi}_{n_2}(1){\phi}_{n_1}(2)} \right)
      (\hat{a}_1 \otimes \hat{I}_2) \nonumber \\
    & \left( \ket{{\phi}_{n_1}(1){\phi}_{n_2}(2)} -
      \ket{{\phi}_{n_2}(1){\phi}_{n_1}(2)} \right).
\end{align}
This product expands into four terms. The operator $\hat{a}_1$ acts
only on the first particle in the tensor product, and $\hat{I}_2$ acts
on the second.
\begin{align}
  \langle \hat{A} \rangle = \frac{1}{2}
    \bigg[
      & \bra{{\phi}_{n_1}}\hat{a}_1\ket{{\phi}_{n_1}}
        \braket{{\phi}_{n_2} | {\phi}_{n_2}}
      - \bra{{\phi}_{n_1}}\hat{a}_1\ket{{\phi}_{n_2}}
        \braket{{\phi}_{n_2} | {\phi}_{n_1}} \nonumber \\
    - & \bra{{\phi}_{n_2}}\hat{a}_1\ket{{\phi}_{n_1}}
        \braket{{\phi}_{n_1} | {\phi}_{n_2}}
      + \bra{{\phi}_{n_2}}\hat{a}_1\ket{{\phi}_{n_2}}
        \braket{{\phi}_{n_1} | {\phi}_{n_1}}
    \bigg].
\end{align}
We now use orthonormality. The single-particle expectation value
is $\langle \hat{a} \rangle_n = \bra{{\phi}_n}\hat{a}\ket{{\phi}_n}$.
The inner products are $\braket{{\phi}_n | {\phi}_m} =
\delta_{nm}$. Since $n_1 \neq n_2$, the cross-terms vanish:
\begin{equation}
  \langle \hat{A} \rangle = \frac{1}{2}
    \bigg[ \langle \hat{a} \rangle_{n_1} + \langle \hat{a} \rangle_{n_2} \bigg].
\end{equation}
This yields the central result for any single-particle operator:
\begin{equation} \label{eq:general_property}
  \langle \hat{A} \rangle = \frac{1}{2} \left(
  \langle \hat{a} \rangle_{n_1} + \langle \hat{a} \rangle_{n_2} \right).
\end{equation}
This confirms that for this indistinguishable two-particle state,
the expectation value for a single-particle operator is the
simple average of the single-particle expectation values
for the two occupied orbitals.

\subsection{Position $\langle \hat{x}_i \rangle$}
We first find the single-particle expectation value,
$\langle \hat{x} \rangle_n = \bra{{\phi}_n} \hat{x}
\ket{{\phi}_n}$:
\begin{equation}
  \langle \hat{x} \rangle_n = \frac{2}{L}
  \int_0^L x \sin^2\left(\frac{n\pi x}{L}\right) \,\mathrm{d}x.
\end{equation}
The integrand is symmetric about $x = L/2$. Therefore, the
integral must evaluate to $L/2$ for all $n$.
Applying our general property from \cref{eq:general_property}:
\begin{equation}
  \langle \hat{x}_1 \rangle = \frac{1}{2}
  \left( \langle \hat{x} \rangle_{n_1} +
  \langle \hat{x} \rangle_{n_2} \right) =
  \frac{1}{2} \left( \frac{L}{2} + \frac{L}{2} \right) = \frac{L}{2}.
\end{equation}
By symmetry, $\langle \hat{x}_2 \rangle = L/2$.

\subsection{Momentum $\langle \hat{p}_i \rangle$}
The single-particle states $\ket{{\phi}_n}$ are stationary
states (real-valued standing waves). They represent an equal
superposition of momentum $+k_n$ and $-k_n$, where
$k_n = n\pi/L$. This symmetry ensures that the expectation
value of momentum is zero for all $n$:
\begin{equation}
  \langle \hat{p} \rangle_n = 0.
\end{equation}
Applying our general property:
\begin{equation}
  \langle \hat{p}_1 \rangle = \frac{1}{2}
  \left( \langle \hat{p} \rangle_{n_1} +
  \langle \hat{p} \rangle_{n_2} \right) =
  \frac{1}{2} (0 + 0) = 0.
\end{equation}
By symmetry, $\langle \hat{p}_2 \rangle = 0$.

\subsection{Position Uncertainty $\Delta x_i$}
The variance is defined as $(\Delta A)^2 = \langle \hat{A}^2 \rangle
- \langle \hat{A} \rangle^2$.

We first require $\langle \hat{x}^2 \rangle_n$:
\begin{equation}
  \langle \hat{x}^2 \rangle_n = \frac{2}{L}
  \int_0^L x^2 \sin^2\left(\frac{n\pi x}{L}\right) \,\mathrm{d}x.
\end{equation}
The standard integral evaluates to
$\int_0^L x^2 \sin^2(k_n x) \, \mathrm{d}x =
\frac{L^3}{6} - \frac{L^3}{4n^2\pi^2}$.
This gives the single-particle expectation value:
\begin{equation}
  \langle \hat{x}^2 \rangle_n =
  \frac{2}{L} \left( \frac{L^3}{6} - \frac{L^3}{4n^2\pi^2} \right)
  = L^2 \left( \frac{1}{3} - \frac{1}{2n^2\pi^2} \right).
\end{equation}
Using \cref{eq:general_property} for the two-particle state:
\begin{align}
  \langle \hat{x}_1^2 \rangle &= \frac{1}{2}
  \left( \langle \hat{x}^2 \rangle_{n_1} +
  \langle \hat{x}^2 \rangle_{n_2} \right) \nonumber \\
  &= \frac{1}{2} \left[ L^2 \left( \frac{1}{3} -
  \frac{1}{2n_1^2\pi^2} \right) +
  L^2 \left( \frac{1}{3} - \frac{1}{2n_2^2\pi^2} \right) \right] \nonumber \\
  &= L^2 \left[ \frac{1}{3} - \frac{1}{4\pi^2}
  \left( \frac{1}{n_1^2} + \frac{1}{n_2^2} \right) \right].
\end{align}
The variance $(\Delta x_1)^2 = \langle \hat{x}_1^2 \rangle
- \langle \hat{x}_1 \rangle^2$ is then
\begin{align}
  (\Delta x_1)^2 &= L^2 \left[ \frac{1}{3} - \frac{1}{4\pi^2}
  \left( \frac{1}{n_1^2} + \frac{1}{n_2^2} \right) \right]
  - \left( \frac{L}{2} \right)^2 \nonumber \\
  &= L^2 \left[ \frac{1}{12} - \frac{1}{4\pi^2}
  \left( \frac{1}{n_1^2} + \frac{1}{n_2^2} \right) \right].
\end{align}

\subsection{Momentum Uncertainty $\Delta p_i$}
We need $\langle \hat{p}^2 \rangle_n$. The states $\ket{{\phi}_n}$
are eigenfunctions of the single-particle Hamiltonian
$\hat{H}_0 = \hat{p}^2/(2m)$. Therefore,
$\langle \hat{p}^2 \rangle_n = 2m \langle \hat{H}_0 \rangle_n = 2mE_n$.
\begin{equation}
  \langle \hat{p}^2 \rangle_n = 2m
  \left( \frac{n^2 \pi^2 \hbar^2}{2mL^2} \right)
  = \left( \frac{n\pi\hbar}{L} \right)^2.
\end{equation}
Applying the general property to find $\langle \hat{p}_1^2 \rangle$:
\begin{align}
  \langle \hat{p}_1^2 \rangle &= \frac{1}{2}
  \left( \langle \hat{p}^2 \rangle_{n_1} +
  \langle \hat{p}^2 \rangle_{n_2} \right) \nonumber \\
  &= \frac{1}{2} \left[ \left( \frac{n_1\pi\hbar}{L} \right)^2 +
  \left( \frac{n_2\pi\hbar}{L} \right)^2 \right] \nonumber \\
  &= \frac{\pi^2 \hbar^2}{2L^2} (n_1^2 + n_2^2).
\end{align}
Since $\langle \hat{p}_1 \rangle = 0$, the variance is
$(\Delta p_1)^2 = \langle \hat{p}_1^2 \rangle$:
\begin{equation}
  (\Delta p_1)^2 = \frac{\pi^2 \hbar^2 (n_1^2 + n_2^2)}{2L^2}.
\end{equation}
By symmetry, $(\Delta x_2)^2 = (\Delta x_1)^2$ and
$(\Delta p_2)^2 = (\Delta p_1)^2$.

\section{Verification of the Uncertainty Principle}

We now verify that the product of these uncertainties,
$\Pi^2 = (\Delta x_1)^2 (\Delta p_1)^2$,
satisfies the Heisenberg principle, $\Pi^2 \ge \hbar^2/4$.
\begin{align}
  \Pi^2 &= \left( L^2 \left[ \frac{1}{12} - \frac{1}{4\pi^2}
    \left( \frac{1}{n_1^2} + \frac{1}{n_2^2} \right) \right] \right)
    \left( \frac{\pi^2 \hbar^2 (n_1^2 + n_2^2)}{2L^2} \right) \nonumber \\
  &= \frac{\pi^2 \hbar^2 (n_1^2 + n_2^2)}{2}
    \left[ \frac{1}{12} - \frac{1}{4\pi^2}
    \left( \frac{n_1^2 + n_2^2}{n_1^2 n_2^2} \right) \right] \nonumber \\
  &= \hbar^2 \left[ \frac{\pi^2 (n_1^2 + n_2^2)}{24} -
    \frac{(n_1^2 + n_2^2)^2}{8 n_1^2 n_2^2} \right].
\end{align}
We test this for the fermionic ground state,
$\{n_1, n_2\} = \{1, 2\}$. This yields $n_1^2 + n_2^2 = 5$
and $n_1^2 n_2^2 = 4$. Substituting these values:
\begin{equation}
  \Pi^2 = \hbar^2 \left[ \frac{5\pi^2}{24} - \frac{5^2}{8(4)} \right]
  = \hbar^2 \left( \frac{5\pi^2}{24} - \frac{25}{32} \right).
\end{equation}
We check this against the Heisenberg limit $\hbar^2/4$:
\begin{equation}
  \frac{5\pi^2}{24} - \frac{25}{32} \ge \frac{1}{4}.
\end{equation}
Multiplying by the common denominator (96):
\begin{align*}
  (4 \cdot 5\pi^2) - (3 \cdot 25) &\ge (24 \cdot 1) \\
  20\pi^2 - 75 &\ge 24 \\
  20\pi^2 &\ge 99 \\
  \pi^2 &\ge 4.95.
\end{align*}
Since $\pi^2 \approx 9.87$, the inequality $9.87 \ge 4.95$ holds.
The uncertainty principle is sustained.
