\documentclass[pdflatex,sn-mathphys-num]{sn-jnl}


\usepackage{graphicx}
\usepackage{multirow}%
\usepackage{amsmath,amssymb,amsfonts}%
\usepackage{amsthm}%
\usepackage[title]{appendix}%
\usepackage{xcolor}%
\usepackage{textcomp}%
\usepackage{manyfoot}%
\usepackage{booktabs,bookmark}
\usepackage{algorithm}%
\usepackage{algorithmicx}%
\usepackage{algpseudocode}%
\usepackage{listings}%
\usepackage{microtype}
\usepackage{csquotes}
\usepackage{enumitem}
\usepackage{hyperref}
\usepackage{cleveref}
\usepackage{siunitx}
\usepackage{derivative}
\usepackage{braket}

\renewcommand{\arraystretch}{1.2}
\sisetup{group-digits=true,
  group-separator={\,},
  separate-uncertainty
}

\hypersetup{
  colorlinks=true,
  linkcolor=black,
  urlcolor=blue,
  pdftitle={Identical Particles in Infinite Well},
  pdfauthor={Avila, Herrera, Rodríguez},
  pdfsubject={Quantum Mechanics II},
  pdfkeywords={Homework 7}
}
\urlstyle{same}

\theoremstyle{thmstyleone}%
\newtheorem{theorem}{Theorem}%  meant for continuous numbers
\newtheorem{proposition}[theorem]{Proposition}%

\theoremstyle{thmstyletwo}%
\newtheorem{example}{Example}%
\newtheorem{remark}{Remark}%

\theoremstyle{thmstylethree}%
\newtheorem{definition}{Definition}%

\raggedbottom

\begin{document}

\title[Two Particles in an Infinite Well]{\textbf{Two Particles in an Infinite Well Potential}}
\author[]{\fnm{Julian} \sur{Avila}}
\author[]{\fnm{Laura} \sur{Herrera}}
\author[]{\fnm{Sebastian} \sur{Rodríguez}}
\affil[]{\orgname{Universidad Distrital Francisco José de Caldas}}

\abstract{
We analyze two indistinguishable fermions in a 1D infinite well
with a $g\delta(x_1 - x_2)$ contact potential. We demonstrate that the
required spatial antisymmetry forces the wavefunction to vanish at
particle coincidence ($\Psi(x,x)=0$), rendering the interaction inert.
The system is then solved as two non-interacting particles.
We derive the position-space eigenfunctions, the momentum-space
wavefunctions via Fourier transform, and the two-particle propagator,
which takes the form of a Slater determinant.

We establish a general theorem for single-particle operator expectation
values, $\langle \hat{A}_1 \rangle = \frac{1}{2}(\langle \hat{a}
\rangle_{n_1} + \langle \hat{a} \rangle_{n_2})$, using it to
calculate uncertainties and verify the Heisenberg principle.
Finally, we construct generalized, $n$-dependent ladder operators,
showing the system admits a deformed Heisenberg algebra consistent
with its non-linear spectrum.
}
\keywords{
  Identical Particles, Infinite Potential Well, Contact Interaction
}

\maketitle

\section{Problem Setting} \label{sec:problem}

We consider two indistinguishable particles, each confined to a
one-dimensional infinite potential well of length $L$.
The single-particle Hilbert space is $\mathcal{H}_i \cong L^2([0, L])$.
The total Hilbert space for the non-relativistic system is the tensor
product of the individual spaces:
\begin{equation}
	\mathcal{H} = \mathcal{H}_1 \otimes \mathcal{H}_2.
\end{equation}
The configuration space is $(x_1, x_2) \in [0, L] \times [0, L]$.
The potential imposes Dirichlet boundary conditions,
requiring the wavefunction $\Psi(x_1, x_2)$ to vanish at the boundaries.

The system's dynamics are governed by the Hamiltonian $\hat{H}$,
which we decompose into kinetic, external potential, and interaction terms:
\begin{equation}
	\hat{H} = \hat{T} + \hat{V}_{\text{ext}} + \hat{V}_{\text{int}}.
\end{equation}
The total kinetic energy $\hat{T} = \hat{T}_1 + \hat{T}_2$ is the sum
of the single-particle operators,
\begin{align}
	\hat{T}_1 &= \frac{\hat{p}_1^2}{2m} \otimes \hat{I}, \\
	\hat{T}_2 &= \hat{I} \otimes \frac{\hat{p}_2^2}{2m},
\end{align}
where $\hat{I}$ is the identity operator on the single-particle space.

The external potential $\hat{V}_{\text{ext}}$ is zero within the well and
infinite otherwise, a constraint already enforced by the boundary
conditions. The particles interact via a contact potential
$\hat{V}_{\text{int}}$, which is proportional to a Dirac delta function:
\begin{equation}
	\hat{V}_{\text{int}}(x_1, x_2) = g\,\delta(x_1 - x_2).
\end{equation}
Here, $g$ represents the coupling strength of the interaction.

In the position representation, the Hamiltonian operator acts on
the wavefunction as
\begin{equation} \label{eq:hamiltonian}
	\hat{H} = -\frac{\hbar^2}{2m}\left(\frac{\partial^2}{\partial x_1^2} +
	\frac{\partial^2}{\partial x_2^2}\right) + g\,\delta(x_1 - x_2).
\end{equation}
The configuration space is the square $[0, L] \times [0, L]$,
and the interaction $\hat{V}_{\text{int}}$ is active only along
the diagonal $x_1 = x_2$.

\section{Symmetry and Indistinguishability} \label{sec:symmetry}

The indistinguishability of the particles implies a fundamental symmetry.
We introduce the particle exchange operator $\hat{P}_{12}$,
whose action on the two-particle wavefunction is defined as
\begin{equation}
	\hat{P}_{12}\Psi(x_1, x_2) = \Psi(x_2, x_1).
\end{equation}
This operator commutes with the Hamiltonian, $[\hat{P}_{12}, \hat{H}] = 0$,
as both the kinetic term and the interaction term are symmetric
under the exchange $x_1 \leftrightarrow x_2$.
This commutation is a crucial property: it ensures that the
exchange symmetry of a state is conserved over time.
Consequently, eigenstates of $\hat{H}$ can be chosen as simultaneous eigenstates
of $\hat{P}_{12}$ with eigenvalues $p_{12} = \pm 1$.
\begin{itemize}
	\item \textbf{Bosons (Symmetric):} $p_{12} = +1$.
		$\Psi_S(x_1, x_2) = \Psi_S(x_2, x_1)$.
	\item \textbf{Fermions (Antisymmetric):} $p_{12} = -1$.
		$\Psi_A(x_1, x_2) = -\Psi_A(x_2, x_1)$.
\end{itemize}
The Spin-Statistics Theorem connects this symmetry to the particle's
intrinsic spin. In this work, we restrict our analysis to the fermionic
case, requiring the total wavefunction to be antisymmetric under
particle exchange.

The state vector must therefore belong to the antisymmetric subspace
$\mathcal{H}_A \subset \mathcal{H}$.
For a state constructed from two distinct single-particle orbitals,
$\ket{\phi_a}$ and $\ket{\phi_b}$, the normalized antisymmetric state is
\begin{equation} \label{eq:slater_ket}
	\ket{\Psi_A} = \frac{1}{\sqrt{2}} \left( \ket{\phi_a} \otimes \ket{\phi_b}
	- \ket{\phi_b} \otimes \ket{\phi_a} \right).
\end{equation}
In this notation, the first ket in each product refers to particle 1
and the second to particle 2.

Projecting \cref{eq:slater_ket} into the position basis
($\Psi_A(x_1, x_2) = \braket{x_1, x_2 | \Psi_A}$) yields the
Slater determinant for the wavefunction:
\begin{equation}
	\Psi_A(x_1, x_2) = \frac{1}{\sqrt{2}} \left( \phi_a(x_1)\phi_b(x_2)
	- \phi_b(x_1)\phi_a(x_2) \right).
\end{equation}
Note that if $\ket{\phi_a} = \ket{\phi_b}$, the state vanishes,
in accordance with the Pauli Exclusion Principle.


\section{Position Representation of the Schrödinger Equation}

The dynamics are governed by the time-dependent Schrödinger equation (TDSE).
In the position representation, using the Hamiltonian from
\cref{eq:hamiltonian}, this reads:
\begin{equation} \label{eq:tdse}
	i\hbar\partial_t\Psi(x_1, x_2, t) = \left(
		-\frac{\hbar^2}{2m}(\partial_{1}^2 + \partial_{2}^2) + g\delta(x_1 - x_2)
	\right) \Psi(x_1, x_2, t).
\end{equation}

A central consequence of the fermionic symmetry, discussed in
\cref{sec:symmetry}, is the antisymmetry of the wavefunction:
$\Psi(x_1, x_2, t) = -\Psi(x_2, x_1, t)$.
This requirement has a profound effect on the interaction term.
If we evaluate the wavefunction along the diagonal $x_1 = x_2 = x$,
the antisymmetry implies
\begin{equation}
	\Psi(x, x, t) = -\Psi(x, x, t) \quad \implies \quad \Psi(x, x, t) = 0.
\end{equation}
The wavefunction must be identically zero for any configuration where
the two particles are at the same position.

Because the delta-function potential $\hat{V}_{\text{int}} = g\delta(x_1 - x_2)$
has support only on this diagonal (where the wavefunction vanishes),
the interaction term has no effect on the system.
We can confirm this by examining the expected value of the
interaction potential:
\begin{align} \label{eq:V_int_expectation}
	\braket{\hat{V}_{\text{int}}}
		&= \braket{\Psi | g\delta(x_1 - x_2) | \Psi} \nonumber \\
		&= g \int_{0}^L \int_{0}^L \Psi^*(x_1, x_2)\Psi(x_1, x_2)
		\delta(x_1 - x_2) \, \mathrm{d}x_2 \mathrm{d}x_1 \nonumber \\
		&= g \int_{0}^L \Psi^*(x_1, x_1) \Psi(x_1, x_1) \, \mathrm{d}x_1 \nonumber \\
		&= g \int_{0}^L |0|^2 \, \mathrm{d}x_1 = 0.
\end{align}
Therefore, for fermions, the problem simplifies remarkably. The system
behaves as two non-interacting identical particles in an infinite well,
and the governing equation reduces to
\begin{equation} \label{eq:tdse_free}
	i\hbar\partial_t\Psi(x_1, x_2, t) =
	-\frac{\hbar^2}{2m}\left(\partial_{1}^2 + \partial_{2}^2\right)
	\Psi(x_1, x_2, t).
\end{equation}
We seek stationary-state solutions by applying the separation of variables,
positing an ansatz of the form
\begin{equation}
	\Psi(x_1, x_2, t) = \psi(x_1, x_2) \varphi(t).
\end{equation}
Substituting this into \cref{eq:tdse_free} and dividing by
$\Psi(x_1, x_2, t)$ separates the spatial and temporal components:
\begin{equation}
	\frac{1}{\psi(x_1, x_2)} \left[
		-\frac{\hbar^2}{2m}(\partial_{1}^2 + \partial_{2}^2)
	\right] \psi(x_1, x_2)
	= i\hbar \frac{1}{\varphi(t)} \mdv{t}!_{\varphi}.
\end{equation}
The left side depends only on position and the right side only on time,
so both must equal a separation constant, which we identify as the
total energy $E$. This yields two independent equations: the
Time-Independent Schrödinger Equation (TISE)
\begin{equation} \label{eq:tise}
	\left[ -\frac{\hbar^2}{2m} (\partial_{1}^2 + \partial_{2}^2) \right]
	\psi(x_1, x_2) = E \psi(x_1, x_2),
\end{equation}
and the temporal equation
\begin{equation} \label{eq:time_eq}
	i\hbar \mdv{t}!_{\varphi} = E \phi(t).
\end{equation}

\subsection{Time Component}
The solution to the temporal equation, \cref{eq:time_eq}, is
straightforward,
\begin{equation}
	\varphi(t) = e^{-iEt/\hbar},
\end{equation}
where we have set the initial phase $\phi(0) = 1$.
The full time-dependent solution for a stationary state is thus
$\Psi(x_1, x_2, t) = \psi(x_1, x_2) e^{-iEt/\hbar}$. The time evolution
manifests only as a global phase, which is unobservable.

\subsection{Spatial Component and Energy}
We now solve the spatial TISE, \cref{eq:tise}. Since the Hamiltonian
is a sum of non-interacting single-particle Hamiltonians
($\hat{H} = \hat{H}_1 + \hat{H}_2$), we can construct the solution
from the single-particle energy eigenfunctions.

The normalized stationary states for a single particle in the well are
\begin{equation}
	\phi_n(x) = \sqrt{\frac{2}{L}} \sin\left(\frac{n\pi x}{L}\right),
	\quad n = 1, 2, 3, \dots
\end{equation}
These are eigenfunctions of the single-particle Hamiltonian,
$\hat{H}_i {\phi}_n(x_i) = E_n {\phi}_n(x_i)$,
with corresponding energy eigenvalues
\begin{equation}
	E_n = \frac{n^2 \pi^2 \hbar^2}{2mL^2}.
\end{equation}
As required for fermions, we construct the two-particle spatial
wavefunction as the normalized Slater determinant
\begin{equation} \label{eq:spatial_wavefunction}
	\psi{(n_1, n_2 ; x_1, x_2)} = \frac{1}{\sqrt{2}} \left[
		{\phi}_{n_1}(x_1){\phi}_{n_2}(x_2) -
		{\phi}_{n_1}(x_2){\phi}_{n_2}(x_1)
	\right]
\end{equation}
which is valid for $n_1 \neq n_2$, in accordance with the
Pauli Exclusion Principle.

We verify this is an eigenfunction of the total spatial Hamiltonian
$\hat{H} = \hat{H}_1 + \hat{H}_2$:
\begin{align}
	\hat{H} \psi
		&= (\hat{H}_1 + \hat{H}_2) \frac{1}{\sqrt{2}} \left[
			{\phi}_{n_1}(x_1){\phi}_{n_2}(x_2) -
		{\phi}_{n_1}(x_2){\phi}_{n_2}(x_1) \right] \nonumber \\
		&= (E_{n_1} + E_{n_2}) \psi_{n_1, n_2}.
\end{align}
By comparing this with the TISE, $\hat{H}\psi = E\psi$, we identify the
total energy of the system as the sum of the single-particle energies:
\begin{equation}
	E_{n_1, n_2} = E_{n_1} + E_{n_2} =
	\frac{\pi^2 \hbar^2}{2mL^2} (n_1^2 + n_2^2).
\end{equation}
The prefactor $1/\sqrt{2}$ in \cref{eq:spatial_wavefunction} ensures
the state is normalized ($\braket{\psi | \psi} = 1$) due to the
orthonormality of the single-particle orbitals ${\phi}_n$.

\subsection{Visualization and Discussion}

To conclude the analysis in the position representation,
we visualize the probability density $|\psi(x_1, x_2)|^2$
for several low-energy stationary states.

\begin{figure}[htbp!]
	\centering
	\includegraphics[width=0.9\textwidth]{./figures/fermions_3D_collage-1.pdf}
	\caption{Probability density $|\psi(x_1, x_2)|^2$ for six antisymmetric
		two-fermion configurations $\{n_1,n_2\} = \{1,2\}, \{1,3\}, \{2,3\},
	\{2,4\}, \{3,4\}, \{1,4\}$.}
	\label{fig:fermion_3D_collage}
\end{figure}

\Cref{fig:fermion_3D_collage} illustrates the spatial characteristics of
the fermionic states. Although the phase information is lost when
taking the modulus squared, a ``nodal line'' is clearly visible along
the diagonal $x_1 = x_2$, where the probability density is
identically zero.

As these are stationary states, the time evolution
$\Psi(x_1,x_2,t) = \psi(x_1,x_2)e^{-iEt/\hbar}$ introduces only a
global phase. The probability density is therefore static:
$|\Psi(x_1,x_2,t)|^2 = |\psi(x_1,x_2)|^2$.

It is crucial to contrast this result with the bosonic case.
The derivation above, which led to the vanishing of the interaction,
is valid only for the antisymmetric spatial sector (fermions).
For bosons, the spatial wavefunction $\psi_S(x_1, x_2)$ is symmetric,
so $\psi_S(x, x) \neq 0$. The delta interaction would therefore be
non-trivial, yielding corrections to the energy eigenvalues.

This fermionic solution trivially nullifies the delta contribution
only because we have assumed a spatially antisymmetric wavefunction.
This implies the fermions are either spinless (a theoretical construct)
or are in a spin-symmetric state (a triplet state),
which forces the spatial part to be antisymmetric.

If, however, the two fermions (e.g., electrons) were in a
spin-antisymmetric (singlet) state, the total wavefunction
$\ket{\Psi}_{\text{total}} = \ket{\psi}_{\text{spatial}}
\otimes \ket{\chi}_{\text{spin}}$
would require a spatially symmetric wavefunction $\psi_S(x_1, x_2)$
to maintain total antisymmetry. In that scenario,
$\psi_S(x, x) \neq 0$, the delta-function interaction would apply,
and the problem would become non-trivial.


\section{Momentum Representation of the Schr\"odinger Equation} \label{sec:momentum}

It is well-known that transforming problems with hard-wall boundaries
to the momentum representation is inherently challenging.
The difficulty arises not from the potential, but from the
kinetic operator.

We seek the TISE by applying the 2D Fourier transform,
defined over the finite domain $x_i \in [0, L]$:
\begin{equation}
	\tilde{\psi}(p_1, p_2) = \frac{1}{2\pi \hbar} \int_0^L \int_0^L \psi(x_1, x_2)
	e^{-i (p_1 x_1 + p_2 x_2)/ \hbar} \, \mathrm{d}x_1 \mathrm{d}x_2.
\end{equation}
As in the position-space analysis (\cref{eq:V_int_expectation}),
the fermionic antisymmetry ensures $\psi(x, x) = 0$. Consequently,
the interaction term $\hat{V}_{\text{int}} = g\delta(x_1 - x_2)$
has a null contribution, and the TISE in momentum space simplifies to
$\mathcal{F}[\hat{T}\psi] = E \tilde{\psi}$.

The primary challenge is the kinetic operator. The Fourier transform
of a second derivative over a finite domain, $\mathcal{F}[\partial_x^2 \psi]$,
does not simply map to $-(p^2/\hbar^2) \tilde{\psi}(p)$.
Instead, integration by parts introduces boundary terms.
For simplicity, consider the 1D kinetic operator $\hat{T} =
-\frac{\hbar^2}{2m}\partial_x^2$. Its Fourier transform is
\begin{equation} \label{eq:ft_kinetic_1d}
	\mathcal{F}[\hat{T}\psi](p) = \frac{p^2}{2m} \tilde{\psi}(p)
	- \frac{\hbar^2}{2m} \left( e^{-i p L/\hbar} \psi'(L) - \psi'(0) \right),
\end{equation}
where $\psi'(x) \equiv \partial_x \psi(x)$ and we have used the
Dirichlet conditions $\psi(0) = \psi(L) = 0$.

Generalizing to our 2D system, the TISE in momentum space
becomes a complex integral equation. The transform of the kinetic term
$\mathcal{F}[(\hat{T}_1 + \hat{T}_2)\psi]$ introduces terms dependent
on the (unknown) derivatives of the wavefunction at all four boundaries
($x_1=0, x_1=L, x_2=0, x_2=L$).
\begin{equation}
	\begin{split}
		\mathcal{F}[(\hat{T}_1 + \hat{T}_2)\psi] = \frac{1}{2m}(p_1^2 + p_2^2)
		\tilde{\psi}(p_1, p_2) &- \frac{\hbar^2}{2m} \left( e^{-i p_1 L / \hbar} \partial_1 \psi(L, x_2) -
		\partial_2 \psi(0, x_2) \right) - \\
													 &- \frac{\hbar^2}{2m} \left(e^{-i p_2 L / \hbar} \partial_2 \psi(x_1, L) - \partial_2 \psi(x_1, 0) \right)
		\end{split}
\end{equation}

This TISE becomes an integral equation whose kernel depends on these
unknown boundary values (e.g., $\partial_{1}\psi(L, x_2)$).
Solving this is notoriously difficult and requires advanced techniques,
such as Green's functions or treating the infinite well as the
limit of a finite potential. Given these complexities, which
obscure the simple physics derived in the position representation,
we will not pursue the momentum-space solution further.


\section{Momentum-Space Wavefunction}

We now explicitly construct the momentum-space wavefunction,
$\tilde{\Psi}_{n_1, n_2}(p_1, p_2, t)$, by applying the
Fourier transform defined in \cref{sec:momentum} to our
position-space solution.

For a stationary state, the time evolution is separable.
The position-space solution (with total energy $E \equiv E_{n_1, n_2}$) is
\begin{equation}
  \Psi(x_1, x_2, t) = \psi_{n_1, n_2}(x_1, x_2) \, e^{-iEt/\hbar},
\end{equation}
where $\psi_{n_1, n_2}$ is the spatial eigenfunction from
\cref{eq:spatial_wavefunction}.
The temporal factor $e^{-iEt/\hbar}$ is independent of the spatial
integration and passes directly through the transform.
The task thus reduces to transforming the spatial component.
The full time-dependent momentum-space wavefunction will be
\begin{equation}
  \tilde{\Psi}_{n_1, n_2}(p_1, p_2, t) = \tilde{\psi}_{n_1, n_2}(p_1, p_2)
  e^{-iEt/\hbar}.
\end{equation}

\subsection{Transformation of the Antisymmetric State}

We apply the 2D transform to the fermionic spatial wavefunction
$\psi_{n_1, n_2}(x_1, x_2)$:
\begin{align}
  \tilde{\psi}_{n_1, n_2}(p_1, p_2) &= \mathcal{F}\left[
    \psi_{n_1, n_2}(x_1, x_2) \right] \\
    &= \mathcal{F}\left[ \frac{1}{\sqrt{2}}\left(
    \phi_{n_1}(x_1)\phi_{n_2}(x_2) -
    \phi_{n_2}(x_1)\phi_{n_1}(x_2)\right) \right].
\end{align}
By the linearity of the transform and its separable property,
$\mathcal{F}\{f(x_1)g(x_2)\} = \mathcal{F}_1\{f(x_1)\} \mathcal{F}_2\{g(x_2)\}$,
this calculation simplifies. Let us define
$\tilde{\phi}_{n}(p_i) = \mathcal{F}_i\{\phi_{n}(x_i)\}$ as the 1D
Fourier transform of a single-particle eigenfunction $\phi_n(x)$.
The transform of each term is:
\begin{align}
  \mathcal{F}\{\phi_{n_1}(x_1)\phi_{n_2}(x_2)\} &=
    \tilde{\phi}_{n_1}(p_1) \tilde{\phi}_{n_2}(p_2), \\
  \mathcal{F}\{\phi_{n_2}(x_1)\phi_{n_1}(x_2)\} &=
    \tilde{\phi}_{n_2}(p_1) \tilde{\phi}_{n_1}(p_2).
\end{align}
Substituting these back establishes the structure of the
momentum-space wavefunction:
\begin{equation}
  \tilde{\psi}_{n_1, n_2}(p_1, p_2) = \frac{1}{\sqrt{2}} \left(
  \tilde{\phi}_{n_1}(p_1) \tilde{\phi}_{n_2}(p_2) -
  \tilde{\phi}_{n_2}(p_1) \tilde{\phi}_{n_1}(p_2) \right).
  \label{eq:momentum_wavefunction_structure}
\end{equation}
This derivation confirms that the antisymmetry of the state is
preserved in the momentum representation. The remaining task
is to find the explicit form of $\tilde{\phi}_n(p)$.

\subsection{Single-Particle Momentum Eigenfunction}

The component $\tilde{\phi}_n(p)$ is the 1D transform of the
normalized single-particle eigenfunction.
We proceed with the explicit calculation:
\begin{align} \label{eq:1d_ft_setup}
  \tilde{\phi}_n(p) &= \frac{1}{\sqrt{2\pi\hbar}} \int_{0}^{L}
    e^{-ipx/\hbar} \phi_n(x) \,\mathrm{d}x \nonumber \\
    &= \frac{1}{\sqrt{2\pi\hbar}} \int_{0}^{L} e^{-ipx/\hbar}
    \sqrt{\frac{2}{L}} \sin\left(\frac{n\pi x}{L}\right)
    \,\mathrm{d}x \nonumber \\
    &= \frac{1}{\sqrt{\pi\hbar L}} \int_{0}^{L} e^{-ipx/\hbar}
    \left[ \frac{e^{i n\pi x/L} - e^{-i n\pi x/L}}{2i} \right]
    \,\mathrm{d}x \nonumber \\
    &= \frac{1}{2i\sqrt{\pi\hbar L}} \left[
    \int_{0}^{L} e^{-i(p/\hbar - n\pi/L)x} \,\mathrm{d}x \right. \nonumber \\
    & \qquad \left. - \int_{0}^{L} e^{-i(p/\hbar + n\pi/L)x} \,\mathrm{d}x
    \right].
\end{align}
Defining $k_p = p/\hbar$ (the free-particle wavenumber) and
$k_n = n\pi/L$ (the quantized state wavenumber), the integrals evaluate to:
\begin{align}
  \tilde{\phi}_n(p) &= \frac{1}{2i\sqrt{\pi\hbar L}} \left[
    \frac{1 - e^{-i(k_p - k_n)L}}{i(k_p - k_n)} -
    \frac{1 - e^{-i(k_p + k_n)L}}{i(k_p + k_n)} \right] \nonumber \\
    &= \frac{1}{2\sqrt{\pi\hbar L}} \left[
    \frac{1 - e^{-i(k_p - k_n)L}}{k_p - k_n} -
    \frac{1 - e^{-i(k_p + k_n)L}}{k_p + k_n} \right].
\end{align}
We use the property $k_n L = (n\pi/L)L = n\pi$, which implies
$e^{\pm i k_n L} = e^{\pm i n\pi} = (-1)^n$.
The numerators in both terms are therefore identical:
\begin{equation}
  1 - e^{-i(k_p \mp k_n)L} = 1 - e^{-ik_p L} e^{\pm i k_n L}
  = 1 - e^{-ipL/\hbar} (-1)^n.
\end{equation}
Substituting this common numerator and combining the fractions:
\begin{align}
  \tilde{\phi}_n(p) &= \frac{1 - (-1)^n e^{-ipL/\hbar}}{2\sqrt{\pi\hbar L}}
    \left[ \frac{1}{k_p - k_n} - \frac{1}{k_p + k_n} \right] \nonumber \\
    &= \frac{1 - (-1)^n e^{-ipL/\hbar}}{2\sqrt{\pi\hbar L}}
    \left[ \frac{(k_p + k_n) - (k_p - k_n)}{k_p^2 - k_n^2} \right] \nonumber \\
    &= \frac{1 - (-1)^n e^{-ipL/\hbar}}{2\sqrt{\pi\hbar L}}
    \left[ \frac{2k_n}{k_p^2 - k_n^2} \right].
\end{align}
Simplifying and re-inserting the definitions of $k_n$ and $k_p$,
we obtain the final form:
\begin{equation}
  \tilde{\phi}_n(p) = \frac{1}{\sqrt{\pi\hbar L}}
  \left( \frac{n\pi/L}{(p/\hbar)^2 - (n\pi/L)^2} \right)
  \left[ 1 - (-1)^n e^{-ipL/\hbar} \right].
  \label{eq:single_particle_momentum_wf}
\end{equation}


\section{Constructing the Propagator} \label{sec:propagator}

We now calculate the propagator (or kernel), $K$, for the system.
The propagator is the matrix element of the time-evolution operator
in the position basis. For this two-particle system, it is defined as
\begin{equation}
  K(x_1, x_2, t; x'_1, x'_2, t') =
  \bra{x_1, x_2} e^{-i\hat{H}(t-t')/\hbar} \ket{x'_1, x'_2},
\end{equation}
where $\hat{H}$ is the full system Hamiltonian. We define the
elapsed time as $\tau = t-t'$.

\subsection{Spectral Decomposition}

The propagator is computed via its spectral decomposition. This is
achieved by inserting a complete set of energy eigenstates.
For our fermionic system, the complete set for the antisymmetric
subspace is $\hat{I}_A = \sum_{n_1 < n_2} \ket{\Psi_{n_1, n_2}}
\bra{\Psi_{n_1, n_2}}$.

The sum is restricted to $n_1 < n_2$ (or $n_1 > n_2$) to count each
distinct eigenstate exactly once. Inserting this identity, we find
\begin{align} \label{eq:propagator_spectral_sum}
  K(x_f; x_i, \tau)
    &= \bra{x_f} e^{-i\hat{H}\tau/\hbar} \hat{I}_A \ket{x_i} \nonumber \\
    &= \sum_{n_1 < n_2} \bra{x_f} e^{-i\hat{H}\tau/\hbar}
    \ket{\Psi_{n_1, n_2}} \braket{\Psi_{n_1, n_2} | x_i} \nonumber \\
    &= \sum_{n_1 < n_2} e^{-iE_{n_1, n_2}\tau/\hbar}
    \braket{x_f | \Psi_{n_1, n_2}} \braket{\Psi_{n_1, n_2} | x_i}
    \nonumber \\
    &= \sum_{n_1 < n_2} \Psi_{n_1, n_2}(x_1, x_2) \,
    \Psi_{n_1, n_2}^*(x'_1, x'_2) \, e^{-iE_{n_1, n_2}\tau/\hbar},
\end{align}
where $x_f \equiv (x_1, x_2)$, $x_i \equiv (x'_1, x'_2)$,
and $E_{n_1, n_2} = E_{n_1} + E_{n_2}$.
We have used $\braket{\Psi | x} = \braket{x | \Psi}^* = \Psi^*(x)$.

This spectral decomposition is exact. As established previously,
our eigenfunctions $\Psi_{n_1, n_2}$ are the true eigenstates of
the full Hamiltonian $\hat{H} = \hat{H}_0 + g\,\delta(x_1 - x_2)$,
because the interaction term $\hat{V}_{\text{int}}$ vanishes
when applied to any antisymmetric wavefunction.

\subsection{Simplification via Unrestricted Sum}

We now substitute the explicit Slater determinant form for
$\Psi_{n_1, n_2}$. The single-particle states $\phi_n(x)$ are real,
so $\phi_n^* = \phi_n$.
\begin{align}
  K = \sum_{n_1>n_2}
    & \left[ \frac{1}{\sqrt{2}}(\phi_{n_1}(x_1)\phi_{n_2}(x_2) -
    \phi_{n_2}(x_1)\phi_{n_1}(x_2)) \right] \nonumber \\
    & \left[ \frac{1}{\sqrt{2}}(\phi_{n_1}(x'_1)\phi_{n_2}(x'_2) -
    \phi_{n_2}(x'_1)\phi_{n_1}(x'_2)) \right]
    e^{-i(E_{n_1}+E_{n_2})\tau/\hbar}.
\end{align}
The normalization factors combine to give a prefactor of $\frac{1}{2}$.
Let us define $T(n_1, n_2)$ as the unnormalized summand:
\begin{align}
  T(n_1, n_2) =
    & (\phi_{n_1}(x_1)\phi_{n_2}(x_2) - \phi_{n_2}(x_1)\phi_{n_1}(x_2))
    \nonumber \\
    & (\phi_{n_1}(x'_1)\phi_{n_2}(x'_2) - \phi_{n_2}(x'_1)\phi_{n_1}(x'_2))
    \, e^{-i(E_{n_1}+E_{n_2})\tau/\hbar}.
\end{align}
The propagator, as derived from \cref{eq:propagator_spectral_sum}, is
thus written as a restricted sum:
\begin{equation} \label{eq:prop_restricted_sum}
  K = \frac{1}{2} \sum_{n_1>n_2} T(n_1, n_2).
\end{equation}
Our goal is to convert this restricted sum into an unrestricted sum
over all $(n_1, n_2)$, which is separable and easier to evaluate.
The summand $T(n_1, n_2)$ has two key properties:
\begin{enumerate}
  \item \textbf{Vanishing Diagonal:} $T(n, n) = 0$, since the
    determinants become zero.
  \item \textbf{Exchange Symmetry:} $T(n_1, n_2) = T(n_2, n_1)$, since
    swapping $n_1 \leftrightarrow n_2$ negates both determinants
    and $E_{n_1}+E_{n_2}$ is symmetric.
\end{enumerate}
Using these properties, we decompose the full, unrestricted sum:
\begin{align}
  \sum_{n_1, n_2} T(n_1, n_2)
    &= \sum_{n_1 > n_2} T(n_1, n_2) + \sum_{n_1 < n_2} T(n_1, n_2)
    + \sum_{n_1 = n_2} T(n_1, n_2) \nonumber \\
    &= \sum_{n_1 > n_2} T(n_1, n_2) + \sum_{n_1 > n_2} T(n_1, n_2) + 0
    \nonumber \\
    &= 2 \sum_{n_1 > n_2} T(n_1, n_2).
\end{align}
This gives the identity we need:
$\sum_{n_1 > n_2} T(n_1, n_2) = \frac{1}{2} \sum_{n_1, n_2} T(n_1, n_2)$.

Substituting this identity back into our expression for $K$
in \cref{eq:prop_restricted_sum}:
\begin{equation}
  K = \frac{1}{2} \left[ \frac{1}{2} \sum_{n_1, n_2} T(n_1, n_2) \right]
  = \frac{1}{4} \sum_{n_1, n_2} T(n_1, n_2).
\end{equation}
We now expand $T(n_1, n_2)$ into its four constituent terms within this
unrestricted sum:
\begin{align} \label{eq:prop_sum_full_expanded}
  K = \frac{1}{4} \sum_{n_1, n_2} e^{-iE_{n_1}\tau/\hbar} e^{-iE_{n_2}\tau/\hbar}
  \bigg[
      & \phantom{+} \phi_{n_1}(x_1)\phi_{n_2}(x_2)
      \phi_{n_1}(x'_1)\phi_{n_2}(x'_2) \nonumber \\
      & - \phi_{n_1}(x_1)\phi_{n_2}(x_2)
      \phi_{n_2}(x'_1)\phi_{n_1}(x'_2) \nonumber \\
      & - \phi_{n_2}(x_1)\phi_{n_1}(x_2)
      \phi_{n_1}(x'_1)\phi_{n_2}(x'_2) \nonumber \\
      & + \phi_{n_2}(x_1)\phi_{n_1}(x_2)
      \phi_{n_2}(x'_1)\phi_{n_1}(x'_2)
    \bigg].
\end{align}
Let us define the single-particle propagator $K_0$ for the
non-interacting particle in the well:
\begin{equation}
  K_0(x, x', \tau) = \sum_{n=1}^{\infty} \phi_n(x) \phi_n(x')
  e^{-iE_n\tau/\hbar}.
\end{equation}
Each of the four terms in \cref{eq:prop_sum_full_expanded} is a
product of separable sums. For example, the first term is:
\begin{align}
  \text{Term 1} = \frac{1}{4}
    & \left( \sum_{n_1} \phi_{n_1}(x_1)\phi_{n_1}(x'_1)
    e^{-iE_{n_1}\tau/\hbar} \right) \nonumber \\
    & \left( \sum_{n_2} \phi_{n_2}(x_2)\phi_{n_2}(x'_2)
    e^{-iE_{n_2}\tau/\hbar} \right) \nonumber \\
    = \frac{1}{4}
    & K_0(x_1, x'_1, \tau) K_0(x_2, x'_2, \tau).
\end{align}
Applying this separation to all four terms yields:
\begin{align}
  K = \frac{1}{4} \bigg[
      & K_0(x_1, x'_1, \tau) K_0(x_2, x'_2, \tau)
      - K_0(x_1, x'_2, \tau) K_0(x_2, x'_1, \tau) \nonumber \\
  - & K_0(x_2, x'_1, \tau) K_0(x_1, x'_2, \tau)
  + K_0(x_2, x'_2, \tau) K_0(x_1, x'_1, \tau)
\bigg].
\end{align}
The first and fourth terms are identical, as are the second and third.
Combining these terms, we arrive at the final expression:
\begin{align}
  K &= \frac{1}{4} \bigg[ 2 K_0(x_1, x'_1, \tau) K_0(x_2, x'_2, \tau)
  - 2 K_0(x_1, x'_2, \tau) K_0(x_2, x'_1, \tau) \bigg] \nonumber \\
    &= \frac{1}{2} \bigg[ K_0(x_1, x'_1, \tau) K_0(x_2, x'_2, \tau)
    - K_0(x_1, x'_2, \tau) K_0(x_2, x'_1, \tau) \bigg].
  \end{align}
  This is the correct propagator for two indistinguishable (anti-symmetric)
  particles, expressed as half the Slater determinant of the
  single-particle propagators:
  \begin{equation} \label{eq:propagator_slater_det}
    K(x_1, x_2, t; x'_1, x'_2, t') = \frac{1}{2} \det
    \begin{vmatrix}
      K_0(x_1, x'_1, \tau) & K_0(x_1, x'_2, \tau) \\
      K_0(x_2, x'_1, \tau) & K_0(x_2, x'_2, \tau)
    \end{vmatrix}.
  \end{equation}


\section{Expected Values and Uncertainties}

\subsection{A General Property of Single-Particle Operators}

We now derive a general property for the expectation value of any
single-particle operator, which will greatly simplify all
subsequent calculations.
Let $\hat{A}$ be an operator that acts only on particle 1:
\begin{equation}
  \hat{A} = \hat{a}_1 \otimes \hat{I}_2,
\end{equation}
where $\hat{a}$ is the corresponding single-particle operator.
We calculate the expectation value using the fully normalized
antisymmetric state $\ket{\psi_{n_1, n_2}}$ from \cref{eq:slater_ket}
and the normalized single-particle orbitals $\ket{{\phi}_n}$
(where $\braket{{\phi}_n | {\phi}_m} = \delta_{nm}$).

The expectation value is $\langle \hat{A} \rangle =
\bra{\psi_{n_1, n_2}} \hat{A} \ket{\psi_{n_1, n_2}}$.
Substituting the state definition:
\begin{align}
  \langle \hat{A} \rangle = \frac{1}{2}
    & \left( \bra{{\phi}_{n_1}(1){\phi}_{n_2}(2)} -
      \bra{{\phi}_{n_2}(1){\phi}_{n_1}(2)} \right)
      (\hat{a}_1 \otimes \hat{I}_2) \nonumber \\
    & \left( \ket{{\phi}_{n_1}(1){\phi}_{n_2}(2)} -
      \ket{{\phi}_{n_2}(1){\phi}_{n_1}(2)} \right).
\end{align}
This product expands into four terms. The operator $\hat{a}_1$ acts
only on the first particle in the tensor product, and $\hat{I}_2$ acts
on the second.
\begin{align}
  \langle \hat{A} \rangle = \frac{1}{2}
    \bigg[
      & \bra{{\phi}_{n_1}}\hat{a}_1\ket{{\phi}_{n_1}}
        \braket{{\phi}_{n_2} | {\phi}_{n_2}}
      - \bra{{\phi}_{n_1}}\hat{a}_1\ket{{\phi}_{n_2}}
        \braket{{\phi}_{n_2} | {\phi}_{n_1}} \nonumber \\
    - & \bra{{\phi}_{n_2}}\hat{a}_1\ket{{\phi}_{n_1}}
        \braket{{\phi}_{n_1} | {\phi}_{n_2}}
      + \bra{{\phi}_{n_2}}\hat{a}_1\ket{{\phi}_{n_2}}
        \braket{{\phi}_{n_1} | {\phi}_{n_1}}
    \bigg].
\end{align}
We now use orthonormality. The single-particle expectation value
is $\langle \hat{a} \rangle_n = \bra{{\phi}_n}\hat{a}\ket{{\phi}_n}$.
The inner products are $\braket{{\phi}_n | {\phi}_m} =
\delta_{nm}$. Since $n_1 \neq n_2$, the cross-terms vanish:
\begin{equation}
  \langle \hat{A} \rangle = \frac{1}{2}
    \bigg[ \langle \hat{a} \rangle_{n_1} + \langle \hat{a} \rangle_{n_2} \bigg].
\end{equation}
This yields the central result for any single-particle operator:
\begin{equation} \label{eq:general_property}
  \langle \hat{A} \rangle = \frac{1}{2} \left(
  \langle \hat{a} \rangle_{n_1} + \langle \hat{a} \rangle_{n_2} \right).
\end{equation}
This confirms that for this indistinguishable two-particle state,
the expectation value for a single-particle operator is the
simple average of the single-particle expectation values
for the two occupied orbitals.

\subsection{Position $\langle \hat{x}_i \rangle$}
We first find the single-particle expectation value,
$\langle \hat{x} \rangle_n = \bra{{\phi}_n} \hat{x}
\ket{{\phi}_n}$:
\begin{equation}
  \langle \hat{x} \rangle_n = \frac{2}{L}
  \int_0^L x \sin^2\left(\frac{n\pi x}{L}\right) \,\mathrm{d}x.
\end{equation}
The integrand is symmetric about $x = L/2$. Therefore, the
integral must evaluate to $L/2$ for all $n$.
Applying our general property from \cref{eq:general_property}:
\begin{equation}
  \langle \hat{x}_1 \rangle = \frac{1}{2}
  \left( \langle \hat{x} \rangle_{n_1} +
  \langle \hat{x} \rangle_{n_2} \right) =
  \frac{1}{2} \left( \frac{L}{2} + \frac{L}{2} \right) = \frac{L}{2}.
\end{equation}
By symmetry, $\langle \hat{x}_2 \rangle = L/2$.

\subsection{Momentum $\langle \hat{p}_i \rangle$}
The single-particle states $\ket{{\phi}_n}$ are stationary
states (real-valued standing waves). They represent an equal
superposition of momentum $+k_n$ and $-k_n$, where
$k_n = n\pi/L$. This symmetry ensures that the expectation
value of momentum is zero for all $n$:
\begin{equation}
  \langle \hat{p} \rangle_n = 0.
\end{equation}
Applying our general property:
\begin{equation}
  \langle \hat{p}_1 \rangle = \frac{1}{2}
  \left( \langle \hat{p} \rangle_{n_1} +
  \langle \hat{p} \rangle_{n_2} \right) =
  \frac{1}{2} (0 + 0) = 0.
\end{equation}
By symmetry, $\langle \hat{p}_2 \rangle = 0$.

\subsection{Position Uncertainty $\Delta x_i$}
The variance is defined as $(\Delta A)^2 = \langle \hat{A}^2 \rangle
- \langle \hat{A} \rangle^2$.

We first require $\langle \hat{x}^2 \rangle_n$:
\begin{equation}
  \langle \hat{x}^2 \rangle_n = \frac{2}{L}
  \int_0^L x^2 \sin^2\left(\frac{n\pi x}{L}\right) \,\mathrm{d}x.
\end{equation}
The standard integral evaluates to
$\int_0^L x^2 \sin^2(k_n x) \, \mathrm{d}x =
\frac{L^3}{6} - \frac{L^3}{4n^2\pi^2}$.
This gives the single-particle expectation value:
\begin{equation}
  \langle \hat{x}^2 \rangle_n =
  \frac{2}{L} \left( \frac{L^3}{6} - \frac{L^3}{4n^2\pi^2} \right)
  = L^2 \left( \frac{1}{3} - \frac{1}{2n^2\pi^2} \right).
\end{equation}
Using \cref{eq:general_property} for the two-particle state:
\begin{align}
  \langle \hat{x}_1^2 \rangle &= \frac{1}{2}
  \left( \langle \hat{x}^2 \rangle_{n_1} +
  \langle \hat{x}^2 \rangle_{n_2} \right) \nonumber \\
  &= \frac{1}{2} \left[ L^2 \left( \frac{1}{3} -
  \frac{1}{2n_1^2\pi^2} \right) +
  L^2 \left( \frac{1}{3} - \frac{1}{2n_2^2\pi^2} \right) \right] \nonumber \\
  &= L^2 \left[ \frac{1}{3} - \frac{1}{4\pi^2}
  \left( \frac{1}{n_1^2} + \frac{1}{n_2^2} \right) \right].
\end{align}
The variance $(\Delta x_1)^2 = \langle \hat{x}_1^2 \rangle
- \langle \hat{x}_1 \rangle^2$ is then
\begin{align}
  (\Delta x_1)^2 &= L^2 \left[ \frac{1}{3} - \frac{1}{4\pi^2}
  \left( \frac{1}{n_1^2} + \frac{1}{n_2^2} \right) \right]
  - \left( \frac{L}{2} \right)^2 \nonumber \\
  &= L^2 \left[ \frac{1}{12} - \frac{1}{4\pi^2}
  \left( \frac{1}{n_1^2} + \frac{1}{n_2^2} \right) \right].
\end{align}

\subsection{Momentum Uncertainty $\Delta p_i$}
We need $\langle \hat{p}^2 \rangle_n$. The states $\ket{{\phi}_n}$
are eigenfunctions of the single-particle Hamiltonian
$\hat{H}_0 = \hat{p}^2/(2m)$. Therefore,
$\langle \hat{p}^2 \rangle_n = 2m \langle \hat{H}_0 \rangle_n = 2mE_n$.
\begin{equation}
  \langle \hat{p}^2 \rangle_n = 2m
  \left( \frac{n^2 \pi^2 \hbar^2}{2mL^2} \right)
  = \left( \frac{n\pi\hbar}{L} \right)^2.
\end{equation}
Applying the general property to find $\langle \hat{p}_1^2 \rangle$:
\begin{align}
  \langle \hat{p}_1^2 \rangle &= \frac{1}{2}
  \left( \langle \hat{p}^2 \rangle_{n_1} +
  \langle \hat{p}^2 \rangle_{n_2} \right) \nonumber \\
  &= \frac{1}{2} \left[ \left( \frac{n_1\pi\hbar}{L} \right)^2 +
  \left( \frac{n_2\pi\hbar}{L} \right)^2 \right] \nonumber \\
  &= \frac{\pi^2 \hbar^2}{2L^2} (n_1^2 + n_2^2).
\end{align}
Since $\langle \hat{p}_1 \rangle = 0$, the variance is
$(\Delta p_1)^2 = \langle \hat{p}_1^2 \rangle$:
\begin{equation}
  (\Delta p_1)^2 = \frac{\pi^2 \hbar^2 (n_1^2 + n_2^2)}{2L^2}.
\end{equation}
By symmetry, $(\Delta x_2)^2 = (\Delta x_1)^2$ and
$(\Delta p_2)^2 = (\Delta p_1)^2$.

\section{Verification of the Uncertainty Principle}

We now verify that the product of these uncertainties,
$\Pi^2 = (\Delta x_1)^2 (\Delta p_1)^2$,
satisfies the Heisenberg principle, $\Pi^2 \ge \hbar^2/4$.
\begin{align}
  \Pi^2 &= \left( L^2 \left[ \frac{1}{12} - \frac{1}{4\pi^2}
    \left( \frac{1}{n_1^2} + \frac{1}{n_2^2} \right) \right] \right)
    \left( \frac{\pi^2 \hbar^2 (n_1^2 + n_2^2)}{2L^2} \right) \nonumber \\
  &= \frac{\pi^2 \hbar^2 (n_1^2 + n_2^2)}{2}
    \left[ \frac{1}{12} - \frac{1}{4\pi^2}
    \left( \frac{n_1^2 + n_2^2}{n_1^2 n_2^2} \right) \right] \nonumber \\
  &= \hbar^2 \left[ \frac{\pi^2 (n_1^2 + n_2^2)}{24} -
    \frac{(n_1^2 + n_2^2)^2}{8 n_1^2 n_2^2} \right].
\end{align}
We test this for the fermionic ground state,
$\{n_1, n_2\} = \{1, 2\}$. This yields $n_1^2 + n_2^2 = 5$
and $n_1^2 n_2^2 = 4$. Substituting these values:
\begin{equation}
  \Pi^2 = \hbar^2 \left[ \frac{5\pi^2}{24} - \frac{5^2}{8(4)} \right]
  = \hbar^2 \left( \frac{5\pi^2}{24} - \frac{25}{32} \right).
\end{equation}
We check this against the Heisenberg limit $\hbar^2/4$:
\begin{equation}
  \frac{5\pi^2}{24} - \frac{25}{32} \ge \frac{1}{4}.
\end{equation}
Multiplying by the common denominator (96):
\begin{align*}
  (4 \cdot 5\pi^2) - (3 \cdot 25) &\ge (24 \cdot 1) \\
  20\pi^2 - 75 &\ge 24 \\
  20\pi^2 &\ge 99 \\
  \pi^2 &\ge 4.95.
\end{align*}
Since $\pi^2 \approx 9.87$, the inequality $9.87 \ge 4.95$ holds.
The uncertainty principle is sustained.


\section{Construction of $n$-Dependent Ladder Operators}

Although the infinite square well does not admit standard Heisenberg
ladder operators (due to the $n^2$ energy spectrum), it is possible
to construct \emph{generalized} creation and annihilation operators.
These operators, $\hat{A}$ and $\hat{A}^\dagger$, connect
consecutive energy levels with coefficients that depend explicitly on
the quantum number $n$ \cite{CohenTanoudji2023, Jalali2013, Curado2001}.

\subsection{Definition on the Spectral Basis}

Let $\{\ket{n}\}_{n \ge 1}$ denote the orthonormal eigenbasis
of the single-particle Hamiltonian,
\begin{equation} \label{eq:H_eigenbasis}
  \hat{H}_0 \ket{n} = E_n \ket{n}, \qquad E_n =
  \frac{\pi^2\hbar^2}{2mL^2}n^2.
\end{equation}
We define the generalized ladder operators by their matrix elements,
which explicitly connect adjacent states:
\begin{equation}
  \hat{A}^\dagger = \sum_{n=1}^{\infty} c_n \ket{n+1}\bra{n}, \qquad
  \hat{A} = \sum_{n=1}^{\infty} c_n^* \ket{n}\bra{n+1}.
\end{equation}
Their action on the eigenbasis is therefore
\begin{equation}
  \hat{A}^\dagger \ket{n} = c_n \ket{n+1}, \qquad
  \hat{A} \ket{n} = c_{n-1}^* \ket{n-1} \quad (\text{for } n > 1),
\end{equation}
and $\hat{A}\ket{1} = 0$, as $\ket{1}$ is the ground state.
The coefficients $c_n$ are chosen to match the spectral structure.

\subsection{Commutation with the Hamiltonian}

We can find the commutation relation between $\hat{H}_0$ and
$\hat{A}^\dagger$ by applying it to an arbitrary state $\ket{n}$:
\begin{align}
  [\hat{H}_0, \hat{A}^\dagger] \ket{n}
    &= (\hat{H}_0 \hat{A}^\dagger - \hat{A}^\dagger \hat{H}_0) \ket{n}
      \nonumber \\
    &= \hat{H}_0 (c_n \ket{n+1}) - \hat{A}^\dagger (E_n \ket{n})
      \nonumber \\
    &= c_n E_{n+1} \ket{n+1} - E_n (c_n \ket{n+1}) \nonumber \\
    &= (E_{n+1} - E_n) \hat{A}^\dagger \ket{n}.
\end{align}
In contrast to the harmonic oscillator, where $E_{n+1} - E_n$ is a
constant, here the difference depends explicitly on $n$:
\begin{equation}
  E_{n+1} - E_n = \frac{\pi^2\hbar^2}{2mL^2} \left( (n+1)^2 - n^2 \right)
  = \frac{\pi^2\hbar^2}{2mL^2}(2n+1).
\end{equation}
This confirms that no constant $\lambda$ exists to satisfy
$[\hat{H}_0,\hat{A}^\dagger] = \lambda \hat{A}^\dagger$.
Instead, the commutator takes the \emph{functional form}
\begin{equation} \label{eq:deformed_comm_H}
  [\hat{H}_0,\hat{A}^\dagger] = F(\hat{N})\,\hat{A}^\dagger,
\end{equation}
where $\hat{N}$ is the number operator ($\hat{N}\ket{n} = n\ket{n}$)
and $F(\hat{N})$ is the nonlinear step operator
\begin{equation}
  F(\hat{N}) = \frac{\pi^2\hbar^2}{2mL^2}\,(2\hat{N}+1).
\end{equation}
This structure defines a deformed Heisenberg algebra,
characteristic of systems with variable energy spacing.

\subsection{Algebraic Structure}

The choice of coefficients $c_n$ defines the specific algebra.
A common choice, motivated by the factorization method,
is to relate $c_n$ to the energy spectrum.
This choice also defines the second commutator, $[\hat{A}, \hat{A}^\dagger]$.
A simple application of the operators shows
\begin{align}
  \hat{A}\hat{A}^\dagger \ket{n} &= \hat{A}(c_n \ket{n+1})
    = c_n c_n^* \ket{n} = |c_n|^2 \ket{n}, \\
  \hat{A}^\dagger\hat{A} \ket{n} &= \hat{A}^\dagger(c_{n-1}^* \ket{n-1})
    = c_{n-1}^* c_{n-1} \ket{n} = |c_{n-1}|^2 \ket{n}.
\end{align}
Therefore, the commutator is a diagonal operator $G(\hat{N})$:
\begin{equation}
  [\hat{A}, \hat{A}^\dagger] \ket{n} = (|c_n|^2 - |c_{n-1}|^2) \ket{n}
  \equiv G(\hat{N}) \ket{n}.
\end{equation}
The specific forms of $F(\hat{N})$ and $G(\hat{N})$ are linked by the
choice of $c_n$ and form a consistent algebraic description.

\subsection{Two-Particle Operators}

This formalism extends to the two-particle system.
Given the additive Hamiltonian $\hat{H} = \hat{H}_1 + \hat{H}_2$,
we define particle-local operators:
\begin{equation}
  \hat{A}_1^\dagger = \hat{A}^\dagger \otimes \hat{I}, \qquad
  \hat{A}_2^\dagger = \hat{I} \otimes \hat{A}^\dagger.
\end{equation}
To map the antisymmetric subspace $\mathcal{H}_A$ to itself,
the total operator must be symmetric under particle exchange
(i.e., it must commute with the exchange operator $\hat{P}_{12}$).
We therefore construct the symmetric combination:
\begin{equation}
  \hat{A}_{\text{sym}}^\dagger = \frac{1}{\sqrt{2}}
  \left(\hat{A}_1^\dagger + \hat{A}_2^\dagger\right).
\end{equation}
Its action on an antisymmetric state $\ket{\Psi_{n_1, n_2}} =
\frac{1}{\sqrt{2}}(\ket{n_1, n_2} - \ket{n_2, n_1})$ is:
\begin{align}
  \hat{A}_{\text{sym}}^\dagger \ket{\Psi_{n_1, n_2}}
    = \frac{1}{2} & \left[
      c_{n_1}(\ket{n_1+1, n_2} - \ket{n_2, n_1+1}) \right. \nonumber \\
    & \left. + c_{n_2}(\ket{n_1, n_2+1} - \ket{n_2+1, n_1})
      \right].
\end{align}
This resulting state is a superposition of two new antisymmetric
eigenstates, $\ket{\Psi_{n_1+1, n_2}}$ and $\ket{\Psi_{n_1, n_2+1}}$,
confirming that $\hat{A}_{\text{sym}}^\dagger$ correctly
acts as a ``creation operator'' within the fermionic subspace.

\subsection{Conclusion}

Generalized ladder operators $\hat{A}$ and $\hat{A}^\dagger$
for the infinite square well do exist. They act consistently
on the eigenbasis but form a nonlinear, or ``deformed,'' algebra:
\begin{equation}
  [\hat{H}_0,\hat{A}^\dagger] = F(\hat{N})\hat{A}^\dagger, \qquad
  [\hat{A},\hat{A}^\dagger] = G(\hat{N}).
\end{equation}
This construction aligns with the generalized Heisenberg algebras
developed in \cite{CohenTanoudji2023, Jalali2013, Curado2001},
where systems with nonlinear spectra (like the infinite well or
Pöschl–Teller potentials) are successfully described using
$n$-dependent ladder operators.


\bibliography{./references.bib}
\nocite{*}

\end{document}
