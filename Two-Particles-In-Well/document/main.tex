\documentclass[pdflatex,sn-mathphys-num]{sn-jnl}


\usepackage{graphicx}
\usepackage{multirow}%
\usepackage{amsmath,amssymb,amsfonts}%
\usepackage{amsthm}%
\usepackage[title]{appendix}%
\usepackage{xcolor}%
\usepackage{textcomp}%
\usepackage{manyfoot}%
\usepackage{booktabs,bookmark}
\usepackage{algorithm}%
\usepackage{algorithmicx}%
\usepackage{algpseudocode}%
\usepackage{listings}%
\usepackage{microtype}
\usepackage{csquotes}
\usepackage{enumitem}
\usepackage{hyperref}
\usepackage{cleveref}
\usepackage{siunitx}
\usepackage{derivative}
\usepackage{braket}

\renewcommand{\arraystretch}{1.2}
\sisetup{group-digits=true,
  group-separator={\,},
  separate-uncertainty
}

\hypersetup{
  colorlinks=true,
  linkcolor=black,
  urlcolor=blue,
  pdftitle={Identical Particles in Infinite Well},
  pdfauthor={Avila, Herrera, Rodríguez},
  pdfsubject={Quantum Mechanics II},
  pdfkeywords={Homework 7}
}
\urlstyle{same}

\theoremstyle{thmstyleone}%
\newtheorem{theorem}{Theorem}%  meant for continuous numbers
\newtheorem{proposition}[theorem]{Proposition}%

\theoremstyle{thmstyletwo}%
\newtheorem{example}{Example}%
\newtheorem{remark}{Remark}%

\theoremstyle{thmstylethree}%
\newtheorem{definition}{Definition}%

\raggedbottom

\begin{document}

\title[Two Particles in an Infinite Well]{\textbf{Two Particles in an Infinite Well Potential}}
\author[]{\fnm{Julian} \sur{Avila}}
\author[]{\fnm{Laura} \sur{Herrera}}
\author[]{\fnm{Sebastian} \sur{Rodríguez}}
\affil[]{\orgname{Universidad Distrital Francisco José de Caldas}}

\abstract{
}
\keywords{
}

\maketitle

\section{Problem Setting} \label{sec:problem}

We consider two indistinguishable particles, each confined to a
one-dimensional infinite potential well of length $L$.
The single-particle Hilbert space is $\mathcal{H}_i \cong L^2([0, L])$.
The total Hilbert space for the non-relativistic system is the tensor
product of the individual spaces:
\begin{equation}
	\mathcal{H} = \mathcal{H}_1 \otimes \mathcal{H}_2.
\end{equation}
The configuration space is $(x_1, x_2) \in [0, L] \times [0, L]$.
The potential imposes Dirichlet boundary conditions,
requiring the wavefunction $\Psi(x_1, x_2)$ to vanish at the boundaries.

The system's dynamics are governed by the Hamiltonian $\hat{H}$,
which we decompose into kinetic, external potential, and interaction terms:
\begin{equation}
	\hat{H} = \hat{T} + \hat{V}_{\text{ext}} + \hat{V}_{\text{int}}.
\end{equation}
The total kinetic energy $\hat{T} = \hat{T}_1 + \hat{T}_2$ is the sum
of the single-particle operators,
\begin{align}
	\hat{T}_1 &= \frac{\hat{p}_1^2}{2m} \otimes \hat{I}, \\
	\hat{T}_2 &= \hat{I} \otimes \frac{\hat{p}_2^2}{2m},
\end{align}
where $\hat{I}$ is the identity operator on the single-particle space.

The external potential $\hat{V}_{\text{ext}}$ is zero within the well and
infinite otherwise, a constraint already enforced by the boundary
conditions. The particles interact via a contact potential
$\hat{V}_{\text{int}}$, which is proportional to a Dirac delta function:
\begin{equation}
	\hat{V}_{\text{int}}(x_1, x_2) = g\,\delta(x_1 - x_2).
\end{equation}
Here, $g$ represents the coupling strength of the interaction.

In the position representation, the Hamiltonian operator acts on
the wavefunction as
\begin{equation} \label{eq:hamiltonian}
	\hat{H} = -\frac{\hbar^2}{2m}\left(\frac{\partial^2}{\partial x_1^2} +
	\frac{\partial^2}{\partial x_2^2}\right) + g\,\delta(x_1 - x_2).
\end{equation}
The configuration space is the square $[0, L] \times [0, L]$,
and the interaction $\hat{V}_{\text{int}}$ is active only along
the diagonal $x_1 = x_2$.

\section{Symmetry and Indistinguishability} \label{sec:symmetry}

The indistinguishability of the particles implies a fundamental symmetry.
We introduce the particle exchange operator $\hat{P}_{12}$,
whose action on the two-particle wavefunction is defined as
\begin{equation}
	\hat{P}_{12}\Psi(x_1, x_2) = \Psi(x_2, x_1).
\end{equation}
This operator commutes with the Hamiltonian, $[\hat{P}_{12}, \hat{H}] = 0$,
as both the kinetic term and the interaction term are symmetric
under the exchange $x_1 \leftrightarrow x_2$.
This commutation is a crucial property: it ensures that the
exchange symmetry of a state is conserved over time.
Consequently, eigenstates of $\hat{H}$ can be chosen as simultaneous eigenstates
of $\hat{P}_{12}$ with eigenvalues $p_{12} = \pm 1$.
\begin{itemize}
	\item \textbf{Bosons (Symmetric):} $p_{12} = +1$.
		$\Psi_S(x_1, x_2) = \Psi_S(x_2, x_1)$.
	\item \textbf{Fermions (Antisymmetric):} $p_{12} = -1$.
		$\Psi_A(x_1, x_2) = -\Psi_A(x_2, x_1)$.
\end{itemize}
The Spin-Statistics Theorem connects this symmetry to the particle's
intrinsic spin. In this work, we restrict our analysis to the fermionic
case, requiring the total wavefunction to be antisymmetric under
particle exchange.

The state vector must therefore belong to the antisymmetric subspace
$\mathcal{H}_A \subset \mathcal{H}$.
For a state constructed from two distinct single-particle orbitals,
$\ket{\phi_a}$ and $\ket{\phi_b}$, the normalized antisymmetric state is
\begin{equation} \label{eq:slater_ket}
	\ket{\Psi_A} = \frac{1}{\sqrt{2}} \left( \ket{\phi_a} \otimes \ket{\phi_b}
	- \ket{\phi_b} \otimes \ket{\phi_a} \right).
\end{equation}
In this notation, the first ket in each product refers to particle 1
and the second to particle 2.

Projecting \cref{eq:slater_ket} into the position basis
($\Psi_A(x_1, x_2) = \braket{x_1, x_2 | \Psi_A}$) yields the
Slater determinant for the wavefunction:
\begin{equation}
	\Psi_A(x_1, x_2) = \frac{1}{\sqrt{2}} \left( \phi_a(x_1)\phi_b(x_2)
	- \phi_b(x_1)\phi_a(x_2) \right).
\end{equation}
Note that if $\ket{\phi_a} = \ket{\phi_b}$, the state vanishes,
in accordance with the Pauli Exclusion Principle.


%\bibliography{../references.bib}
\nocite{*}

\end{document}
