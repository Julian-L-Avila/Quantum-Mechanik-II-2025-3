\documentclass[pdflatex,sn-mathphys-num]{sn-jnl}


\usepackage{graphicx}
\usepackage{multirow}%
\usepackage{amsmath,amssymb,amsfonts}%
\usepackage{amsthm}%
\usepackage[title]{appendix}%
\usepackage{xcolor}%
\usepackage{textcomp}%
\usepackage{manyfoot}%
\usepackage{booktabs,bookmark}
\usepackage{algorithm}%
\usepackage{algorithmicx}%
\usepackage{algpseudocode}%
\usepackage{listings}%
\usepackage{microtype}
\usepackage{csquotes}
\usepackage{enumitem}
\usepackage{hyperref}
\usepackage{cleveref}
\usepackage{siunitx}
\usepackage{derivative}
\usepackage{braket}

\renewcommand{\arraystretch}{1.2}
\sisetup{group-digits=true,
  group-separator={\,},
  separate-uncertainty
}

\hypersetup{
  colorlinks=true,
  linkcolor=black,
  urlcolor=blue,
  pdftitle={Identical Particles in Infinite Well},
  pdfauthor={Avila, Herrera, Rodríguez},
  pdfsubject={Quantum Mechanics II},
  pdfkeywords={Homework 7}
}
\urlstyle{same}

\theoremstyle{thmstyleone}%
\newtheorem{theorem}{Theorem}%  meant for continuous numbers
\newtheorem{proposition}[theorem]{Proposition}%

\theoremstyle{thmstyletwo}%
\newtheorem{example}{Example}%
\newtheorem{remark}{Remark}%

\theoremstyle{thmstylethree}%
\newtheorem{definition}{Definition}%

\raggedbottom

\begin{document}

\title[Two Particles in an Infinite Well]{\textbf{Two Particles in an Infinite Well Potential}}
\author[]{\fnm{Julian} \sur{Avila}}
\author[]{\fnm{Laura} \sur{Herrera}}
\author[]{\fnm{Sebastian} \sur{Rodríguez}}
\affil[]{\orgname{Universidad Distrital Francisco José de Caldas}}

\abstract{
}
\keywords{
}

\maketitle

\section{Problem Setting} \label{sec:problem}

We consider two indistinguishable particles, each confined to a
one-dimensional infinite potential well of length $L$.
The single-particle Hilbert space is $\mathcal{H}_i \cong L^2([0, L])$.
The total Hilbert space for the non-relativistic system is the tensor
product of the individual spaces:
\begin{equation}
	\mathcal{H} = \mathcal{H}_1 \otimes \mathcal{H}_2.
\end{equation}
The configuration space is $(x_1, x_2) \in [0, L] \times [0, L]$.
The potential imposes Dirichlet boundary conditions,
requiring the wavefunction $\Psi(x_1, x_2)$ to vanish at the boundaries.

The system's dynamics are governed by the Hamiltonian $\hat{H}$,
which we decompose into kinetic, external potential, and interaction terms:
\begin{equation}
	\hat{H} = \hat{T} + \hat{V}_{\text{ext}} + \hat{V}_{\text{int}}.
\end{equation}
The total kinetic energy $\hat{T} = \hat{T}_1 + \hat{T}_2$ is the sum
of the single-particle operators,
\begin{align}
	\hat{T}_1 &= \frac{\hat{p}_1^2}{2m} \otimes \hat{I}, \\
	\hat{T}_2 &= \hat{I} \otimes \frac{\hat{p}_2^2}{2m},
\end{align}
where $\hat{I}$ is the identity operator on the single-particle space.

The external potential $\hat{V}_{\text{ext}}$ is zero within the well and
infinite otherwise, a constraint already enforced by the boundary
conditions. The particles interact via a contact potential
$\hat{V}_{\text{int}}$, which is proportional to a Dirac delta function:
\begin{equation}
	\hat{V}_{\text{int}}(x_1, x_2) = g\,\delta(x_1 - x_2).
\end{equation}
Here, $g$ represents the coupling strength of the interaction.

In the position representation, the Hamiltonian operator acts on
the wavefunction as
\begin{equation} \label{eq:hamiltonian}
	\hat{H} = -\frac{\hbar^2}{2m}\left(\partial_{x_1}^2 +
	\partial_{x_2}^2\right) + g\,\delta(x_1 - x_2).
\end{equation}
The configuration space is the square $[0, L] \times [0, L]$,
and the interaction $\hat{V}_{\text{int}}$ is active only along
the diagonal $x_1 = x_2$.

\section{Symmetry and Indistinguishability} \label{sec:symmetry}

The indistinguishability of the particles implies a fundamental symmetry.
We introduce the particle exchange operator $\hat{P}_{12}$,
whose action on the two-particle wavefunction is defined as
\begin{equation}
	\hat{P}_{12}\Psi(x_1, x_2) = \Psi(x_2, x_1).
\end{equation}
This operator commutes with the Hamiltonian, $[\hat{P}_{12}, \hat{H}] = 0$,
as both the kinetic term and the interaction term are symmetric
under the exchange $x_1 \leftrightarrow x_2$.
This commutation is a crucial property: it ensures that the
exchange symmetry of a state is conserved over time.
Consequently, eigenstates of $\hat{H}$ can be chosen as simultaneous eigenstates
of $\hat{P}_{12}$ with eigenvalues $p_{12} = \pm 1$.
\begin{itemize}
	\item \textbf{Bosons (Symmetric):} $p_{12} = +1$.
		$\Psi_S(x_1, x_2) = \Psi_S(x_2, x_1)$.
	\item \textbf{Fermions (Antisymmetric):} $p_{12} = -1$.
		$\Psi_A(x_1, x_2) = -\Psi_A(x_2, x_1)$.
\end{itemize}
The Spin-Statistics Theorem connects this symmetry to the particle's
intrinsic spin. In this work, we restrict our analysis to the fermionic
case, requiring the total wavefunction to be antisymmetric under
particle exchange.

The state vector must therefore belong to the antisymmetric subspace
$\mathcal{H}_A \subset \mathcal{H}$.
For a state constructed from two distinct single-particle orbitals,
$\ket{\phi_a}$ and $\ket{\phi_b}$, the normalized antisymmetric state is
\begin{equation} \label{eq:slater_ket}
	\ket{\Psi_A} = \frac{1}{\sqrt{2}} \left( \ket{\phi_a} \otimes \ket{\phi_b}
	- \ket{\phi_b} \otimes \ket{\phi_a} \right).
\end{equation}
In this notation, the first ket in each product refers to particle 1
and the second to particle 2.

Projecting \cref{eq:slater_ket} into the position basis
($\Psi_A(x_1, x_2) = \braket{x_1, x_2 | \Psi_A}$) yields the
Slater determinant for the wavefunction:
\begin{equation}
	\Psi_A(x_1, x_2) = \frac{1}{\sqrt{2}} \left( \phi_a(x_1)\phi_b(x_2)
	- \phi_b(x_1)\phi_a(x_2) \right).
\end{equation}
Note that if $\ket{\phi_a} = \ket{\phi_b}$, the state vanishes,
in accordance with the Pauli Exclusion Principle.


\section{Position Representation of the Schrödinger Equation}

The dynamics are governed by the time-dependent Schrödinger equation (TDSE).
In the position representation, using the Hamiltonian from
\cref{eq:hamiltonian}, this reads:
\begin{equation} \label{eq:tdse}
	i\hbar\partial_t\Psi(x_1, x_2, t) = \left(
		-\frac{\hbar^2}{2m}(\partial_{x_1}^2 + \partial_{x_2}^2) + g\delta(x_1 - x_2)
	\right) \Psi(x_1, x_2, t).
\end{equation}

A central consequence of the fermionic symmetry, discussed in
\cref{sec:symmetry}, is the antisymmetry of the wavefunction:
$\Psi(x_1, x_2, t) = -\Psi(x_2, x_1, t)$.
This requirement has a profound effect on the interaction term.
If we evaluate the wavefunction along the diagonal $x_1 = x_2 = x$,
the antisymmetry implies
\begin{equation}
	\Psi(x, x, t) = -\Psi(x, x, t) \quad \implies \quad \Psi(x, x, t) = 0.
\end{equation}
The wavefunction must be identically zero for any configuration where
the two particles are at the same position.

Because the delta-function potential $\hat{V}_{\text{int}} = g\delta(x_1 - x_2)$
has support only on this diagonal (where the wavefunction vanishes),
the interaction term has no effect on the system.
We can confirm this by examining the expected value of the
interaction potential:
\begin{align} \label{eq:V_int_expectation}
	\braket{\hat{V}_{\text{int}}}
		&= \braket{\Psi | g\delta(x_1 - x_2) | \Psi} \nonumber \\
		&= g \int_{0}^L \int_{0}^L \Psi^*(x_1, x_2)\Psi(x_1, x_2)
		\delta(x_1 - x_2) \, \mathrm{d}x_2 \mathrm{d}x_1 \nonumber \\
		&= g \int_{0}^L \Psi^*(x_1, x_1) \Psi(x_1, x_1) \, \mathrm{d}x_1 \nonumber \\
		&= g \int_{0}^L |0|^2 \, \mathrm{d}x_1 = 0.
\end{align}
Therefore, for fermions, the problem simplifies remarkably. The system
behaves as two non-interacting identical particles in an infinite well,
and the governing equation reduces to
\begin{equation} \label{eq:tdse_free}
	i\hbar\partial_t\Psi(x_1, x_2, t) =
	-\frac{\hbar^2}{2m}\left(\partial_{x_1}^2 + \partial_{x_2}^2\right)
	\Psi(x_1, x_2, t).
\end{equation}
We seek stationary-state solutions by applying the separation of variables,
positing an ansatz of the form
\begin{equation}
	\Psi(x_1, x_2, t) = \psi(x_1, x_2) \varphi(t).
\end{equation}
Substituting this into \cref{eq:tdse_free} and dividing by
$\Psi(x_1, x_2, t)$ separates the spatial and temporal components:
\begin{equation}
	\frac{1}{\psi(x_1, x_2)} \left[
		-\frac{\hbar^2}{2m}(\partial_{x_1}^2 + \partial_{x_2}^2)
	\right] \psi(x_1, x_2)
	= i\hbar \frac{1}{\varphi(t)} \mdv{t}!_{\varphi}.
\end{equation}
The left side depends only on position and the right side only on time,
so both must equal a separation constant, which we identify as the
total energy $E$. This yields two independent equations: the
Time-Independent Schrödinger Equation (TISE)
\begin{equation} \label{eq:tise}
	\left[ -\frac{\hbar^2}{2m} (\partial_{x_1}^2 + \partial_{x_2}^2) \right]
	\psi(x_1, x_2) = E \psi(x_1, x_2),
\end{equation}
and the temporal equation
\begin{equation} \label{eq:time_eq}
	i\hbar \mdv{t}!_{\varphi} = E \phi(t).
\end{equation}

\subsection{Time Component}
The solution to the temporal equation, \cref{eq:time_eq}, is
straightforward,
\begin{equation}
	\varphi(t) = e^{-iEt/\hbar},
\end{equation}
where we have set the initial phase $\phi(0) = 1$.
The full time-dependent solution for a stationary state is thus
$\Psi(x_1, x_2, t) = \psi(x_1, x_2) e^{-iEt/\hbar}$. The time evolution
manifests only as a global phase, which is unobservable.

\subsection{Spatial Component and Energy}
We now solve the spatial TISE, \cref{eq:tise}. Since the Hamiltonian
is a sum of non-interacting single-particle Hamiltonians
($\hat{H} = \hat{H}_1 + \hat{H}_2$), we can construct the solution
from the single-particle energy eigenfunctions.

The normalized stationary states for a single particle in the well are
\begin{equation}
	\phi_n(x) = \sqrt{\frac{2}{L}} \sin\left(\frac{n\pi x}{L}\right),
	\quad n = 1, 2, 3, \dots
\end{equation}
These are eigenfunctions of the single-particle Hamiltonian,
$\hat{H}_i {\phi}_n(x_i) = E_n {\phi}_n(x_i)$,
with corresponding energy eigenvalues
\begin{equation}
	E_n = \frac{n^2 \pi^2 \hbar^2}{2mL^2}.
\end{equation}
As required for fermions, we construct the two-particle spatial
wavefunction as the normalized Slater determinant
\begin{equation} \label{eq:spatial_wavefunction}
	\psi{(n_1, n_2 ; x_1, x_2)} = \frac{1}{\sqrt{2}} \left[
		{\phi}_{n_1}(x_1){\phi}_{n_2}(x_2) -
		{\phi}_{n_1}(x_2){\phi}_{n_2}(x_1)
	\right]
\end{equation}
which is valid for $n_1 \neq n_2$, in accordance with the
Pauli Exclusion Principle.

We verify this is an eigenfunction of the total spatial Hamiltonian
$\hat{H} = \hat{H}_1 + \hat{H}_2$:
\begin{align}
	\hat{H} \psi
		&= (\hat{H}_1 + \hat{H}_2) \frac{1}{\sqrt{2}} \left[
			{\phi}_{n_1}(x_1){\phi}_{n_2}(x_2) -
		{\phi}_{n_1}(x_2){\phi}_{n_2}(x_1) \right] \nonumber \\
		&= (E_{n_1} + E_{n_2}) \psi_{n_1, n_2}.
\end{align}
By comparing this with the TISE, $\hat{H}\psi = E\psi$, we identify the
total energy of the system as the sum of the single-particle energies:
\begin{equation}
	E_{n_1, n_2} = E_{n_1} + E_{n_2} =
	\frac{\pi^2 \hbar^2}{2mL^2} (n_1^2 + n_2^2).
\end{equation}
The prefactor $1/\sqrt{2}$ in \cref{eq:spatial_wavefunction} ensures
the state is normalized ($\braket{\psi | \psi} = 1$) due to the
orthonormality of the single-particle orbitals ${\phi}_n$.


\section{Momentum Representation of the Schr\"odinger Equation} \label{sec:momentum}

It is well-known that transforming problems with hard-wall boundaries
to the momentum representation is inherently challenging.
The difficulty arises not from the potential, but from the
kinetic operator.

We seek the TISE by applying the 2D Fourier transform,
defined over the finite domain $x_i \in [0, L]$:
\begin{equation} \label{eq:fourier-transform}
	\tilde{\psi}(p_1, p_2) = \frac{1}{2\pi \hbar} \int_0^L \int_0^L \psi(x_1, x_2)
	e^{-i (p_1 x_1 + p_2 x_2)/ \hbar} \, \mathrm{d}x_1 \mathrm{d}x_2.
\end{equation}
As in the position-space analysis (\cref{eq:V_int_expectation}),
the fermionic antisymmetry ensures $\psi(x, x) = 0$. Consequently,
the interaction term $\hat{V}_{\text{int}} = g\delta(x_1 - x_2)$
has a null contribution, and the TISE in momentum space simplifies to
$\mathcal{F}[\hat{T}\psi] = E \tilde{\psi}$.

The primary challenge is the kinetic operator. The Fourier transform
of a second derivative over a finite domain, $\mathcal{F}[\partial_x^2 \psi]$,
does not simply map to $-(p^2/\hbar^2) \tilde{\psi}(p)$.
Instead, integration by parts introduces boundary terms.
For simplicity, consider the 1D kinetic operator $\hat{T} =
-\frac{\hbar^2}{2m}\partial_x^2$. Its Fourier transform is
\begin{equation} \label{eq:ft_kinetic_1d}
	\mathcal{F}[\hat{T}\psi](p) = \frac{p^2}{2m} \tilde{\psi}(p)
	- \frac{\hbar^2}{2m} \left( e^{-i p L/\hbar} \psi'(L) - \psi'(0) \right),
\end{equation}
where $\psi'(x) \equiv \partial_x \psi(x)$ and we have used the
Dirichlet conditions $\psi(0) = \psi(L) = 0$.

Generalizing to our 2D system, the TISE in momentum space
becomes a complex integral equation. The transform of the kinetic term
$\mathcal{F}[(\hat{T}_1 + \hat{T}_2)\psi]$ introduces terms dependent
on the (unknown) derivatives of the wavefunction at all four boundaries
($x_1=0, x_1=L, x_2=0, x_2=L$).
\begin{equation}
	\begin{split}
		\mathcal{F}[(\hat{T}_1 + \hat{T}_2)\psi] = \frac{1}{2m}(p_1^2 + p_2^2)
		\tilde{\psi}(p_1, p_2) &- \frac{\hbar^2}{2m} \left( e^{-i p_1 L / \hbar} \partial_1 \psi(L, x_2) -
		\partial_2 \psi(0, x_2) \right) - \\
													 &- \frac{\hbar^2}{2m} \left(e^{-i p_2 L / \hbar} \partial_2 \psi(x_1, L) - \partial_2 \psi(x_1, 0) \right)
		\end{split}
\end{equation}

This TISE becomes an integral equation whose kernel depends on these
unknown boundary values (e.g., $\partial_{1}\psi(L, x_2)$).
Solving this is notoriously difficult and requires advanced techniques,
such as Green's functions or treating the infinite well as the
limit of a finite potential. Given these complexities, which
obscure the simple physics derived in the position representation,
we will not pursue the momentum-space solution further.


%\bibliography{../references.bib}
\nocite{*}

\end{document}
