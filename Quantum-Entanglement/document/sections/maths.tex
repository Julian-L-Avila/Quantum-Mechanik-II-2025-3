\section{Mathematical Framework for Entanglement}
\label{sec:math_framework}

Although quantum entanglement first emerged as a mathematical feature of the
early formulations of quantum mechanics, subsequent experimental confirmations
and theoretical advancements have enabled a more rigorous description.
The core of this description lies in how we represent the state of a
composite quantum system. While the framework is well-established, its
physical interpretation---a common theme in quantum theory---remains a
subject of active debate.

\subsection{Departure from Classical State Spaces}
\label{sub:departure_classical}

In any formalism of classical mechanics (e.g., Hamiltonian), the state of a
composite system is constructed from the states of its individual parts.
For a two-particle system, $A$ and $B$, the total state is a point
$(x_A, x_B)$ in the Cartesian product of the individual
phase spaces, $\Gamma_{AB} = \Gamma_A \times \Gamma_B$.
Here, $x_i = (q_i, p_i)$ represents the generalized coordinates and
momenta for particle $i$. This structure implies that specifying the state of
each subsystem completely specifies the state of the whole.

A naive extrapolation to quantum mechanics would suggest that the state space
of a two-particle system is the Cartesian product of the individual Hilbert
spaces, $\mathcal{H}_{AB} = \mathcal{H}_A \times \mathcal{H}_B$. However, this
structure is insufficient because it fails to incorporate the
superposition principle correctly. A state in
$\mathcal{H}_A \times \mathcal{H}_B$ is an ordered pair $(\ket{\psi_A},
\ket{\phi_B})$, but this framework does not naturally accommodate the
linear combinations of such pairs that are physically essential.

To resolve this, the state space for a composite quantum system is
constructed using the tensor product of the individual Hilbert
spaces:
\begin{equation}
	\mathcal{H}_{AB} = \mathcal{H}_A \otimes \mathcal{H}_B.
\end{equation}
A general pure state $\ket{\Psi}$ in this composite space is a superposition
of basis states:
\begin{equation}
	\ket{\Psi} = \sum_{i,j} c_{ij} (\ket{a_i} \otimes \ket{b_j}),
	\label{eq:general_state}
\end{equation}
where $\{\ket{a_i}\}$ and $\{\ket{b_j}\}$ are orthonormal bases for
$\mathcal{H}_A$ and $\mathcal{H}_B$, respectively. The dimension of this new
space is the product of the dimensions of the constituent spaces,
$\dim(\mathcal{H}_{AB}) = \dim(\mathcal{H}_A) \cdot \dim(\mathcal{H}_B)$.

\subsection{Separable and Entangled States}
\label{sub:separable_entangled}

The structure of the tensor product space $\mathcal{H}_{AB}$ allows for two
fundamentally different classes of states.

\paragraph{Separable States.}
A state is called separable (or a product state) if it
can be written as a single tensor product of states from the subsystems,
i.e.,
\begin{equation}
	\ket{\Psi}_{\text{sep}} = \ket{\psi_A} \otimes \ket{\psi_B},
\end{equation}
where $\ket{\psi_A} \in \mathcal{H}_A$ and $\ket{\psi_B} \in \mathcal{H}_B$.
For these states, the subsystems have definite individual properties,
independent of one another, analogous to the classical case.

\paragraph{Entangled States.}
If a state $\ket{\Psi} \in \mathcal{H}_{AB}$ cannot be written in the
separable form above, it is called an entangled state. These states
represent a uniquely quantum-mechanical correlation between subsystems.
Measuring a property of subsystem $A$ instantaneously influences the
properties of subsystem $B$, regardless of the physical distance
separating them.

\subsection{Operators and the Density Matrix Formalism}
\label{sub:operators_density}

An operator $O_A$ acting only on subsystem $A$ is represented on the
composite space as $O_A \otimes I_B$, where $I_B$ is the
identity operator on $\mathcal{H}_B$. Such operators are called
local operators. A general operator on $\mathcal{H}_{AB}$ is a
linear combination of tensor products of local operators.

The inner product on $\mathcal{H}_{AB}$ is defined by extending the inner
products of the subsystems linearly:
\begin{equation}
	(\bra{\phi_A} \otimes \bra{\phi_B}) (\ket{\psi_A} \otimes \ket{\psi_B})
	= \braket{\phi_A | \psi_A}_{\mathcal{H}_A} \braket{\phi_B | \psi_B}_{\mathcal{H}_B}.
\end{equation}

While pure states $\ket{\Psi}$ describe systems with maximal information,
entanglement forces us to consider situations of incomplete information,
especially when looking at only one part of an entangled pair. The
density matrix (or density operator), $\rho$, provides the necessary
formalism.

For a pure state $\ket{\Psi}$, the density matrix is the projection operator:
\begin{equation}
	\rho_{\text{pure}} = \ket{\Psi}\bra{\Psi}.
\end{equation}
A key property of a pure state's density matrix is that it is a projector,
meaning $\rho^2 = \rho$. This leads to a simple test for purity:
\begin{equation}
	\mathrm{Tr}(\rho^2) = 1,
\end{equation}
where the value $\mathrm{Tr}(\rho^2)$ is known as the \textit{purity} of the
state.

If we have statistical knowledge of a system (an ensemble of pure states
$\{\ket{\psi_i}\}$ with probabilities $\{p_i\}$), the state is a
mixed state, described by the density matrix:
\begin{equation}
	\rho_{\text{mixed}} = \sum_i p_i \ket{\psi_i}\bra{\psi_i}.
\end{equation}
For any mixed state, the purity is less than one:
$\mathrm{Tr}(\rho_{\text{mixed}}^2) < 1$.

\subsection{Quantifying Uncertainty: Von Neumann Entropy}
\label{sub:von_neumann_entropy}

The density matrix allows us to assign a single number to quantify the
degree of uncertainty or ``mixedness'' of a quantum state. This measure is the
Von Neumann entropy, defined as:
\begin{equation}
	S(\rho) = - \mathrm{Tr}(\rho \ln \rho).
\end{equation}
This is the quantum mechanical analogue of the Gibbs entropy in statistical
mechanics. If the eigenvalues of $\rho$ are $\lambda_i$, the entropy can be
calculated as $S = - \sum_i \lambda_i \ln \lambda_i$.

The Von Neumann entropy has several key properties:
\begin{itemize}
	\item For any pure state, $\rho = \ket{\psi}\bra{\psi}$, the
		entropy is zero, $S(\rho) = 0$. This corresponds to a state of
		maximal knowledge or zero uncertainty.
	\item For a maximally mixed state in a $d$-dimensional Hilbert
		space, where $\rho = I/d$, the entropy is maximal,
		$S(\rho) = \ln d$. This corresponds to a state of minimal knowledge.
\end{itemize}

The true power of this measure becomes apparent when applied to entangled
systems. Consider a composite system $AB$ in a pure state
$\ket{\Psi}_{AB}$, for which $S(\rho_{AB})=0$. If we are interested only in
subsystem $A$, its state is described by the reduced density matrix,
$\rho_A$, obtained by tracing over subsystem $B$:
\begin{equation}
	\rho_A = \mathrm{Tr}_B(\rho_{AB}) = \mathrm{Tr}_B(\ket{\Psi}_{AB}\bra{\Psi}_{AB}).
\end{equation}
If the global state $\ket{\Psi}_{AB}$ is entangled, the local state $\rho_A$
will be a mixed state, and its entropy $S(\rho_A)$ will be greater than zero.
This non-zero entropy of a subsystem, despite the global system being in a
pure state, is a fundamental signature of entanglement. For any pure
bipartite state, $S(\rho_A) = S(\rho_B)$, and this value is used as the
entropy of entanglement.

\section{Quantifying and Detecting Entanglement}
\label{sec:quantifying_detecting}

With the mathematical framework established, we need tools to answer two
practical questions: How much entanglement does a state possess? And how can
we experimentally detect it, especially in mixed states?

\subsection{Pure States: Schmidt Decomposition and Entropy of Entanglement}
\label{sub:pure_state_measures}

For pure bipartite states, a powerful theorem known as the
Schmidt decomposition provides a canonical form. Any state
$\ket{\Psi}_{AB}$ can be written as a single sum:
\begin{equation}
	\ket{\Psi}_{AB} = \sum_{k=1}^{r_S} \sqrt{\lambda_k} \ket{k}_A \otimes \ket{k}_B,
\end{equation}
where $\{\ket{k}_A\}$ and $\{\ket{k}_B\}$ are orthonormal bases for their
respective subsystems, and $\lambda_k > 0$ with $\sum_k \lambda_k = 1$. The
number of terms, $r_S$, is the Schmidt rank. A state is separable if
and only if its Schmidt rank is 1.

The degree of entanglement is uniquely quantified by the
entropy of entanglement. This is simply the Von Neumann entropy of
the reduced density matrix of either subsystem:
\begin{equation}
	E(\ket{\Psi}) = S(\rho_A) = -\mathrm{Tr}(\rho_A \ln \rho_A) = -\sum_{k=1}^{r_S} \lambda_k \ln \lambda_k.
\end{equation}
The Schmidt coefficients $\sqrt{\lambda_k}$ are directly related to the
eigenvalues of the reduced density matrix, elegantly connecting the
state's structure to its entanglement content.

\subsection{Mixed States: A Zoo of Measures}
\label{sub:mixed_state_measures}
Quantifying entanglement in mixed states is far more complex, as they lack
a unique decomposition. Several measures exist, each with a different
operational or geometric motivation.

\paragraph{Entanglement of Formation ($E_F$).}
This measure asks: what is the minimum average entanglement (of pure states)
required to create a given mixed state $\rho$? It is defined via a
``convex roof'' construction:
\begin{equation}
	E_F(\rho) = \min_{\{p_i, \ket{\psi_i}\}} \sum_i p_i E(\ket{\psi_i}),
\end{equation}
where the minimization is over all possible pure-state ensembles
$\{p_i, \ket{\psi_i}\}$ that decompose $\rho$.

\paragraph{Concurrence ($C$).}
For the specific case of two-qubit systems, the Concurrence is a widely
used measure computationally related to $E_F$. It has the great advantage
of having a closed-form analytical solution, making it highly practical.

\paragraph{Relative Entropy of Entanglement ($E_R$).}
This measure takes a geometric approach, defining the entanglement of a
state $\rho$ as its ``distance'' to the set of all separable states (SEP):
\begin{equation}
	E_R(\rho) = \min_{\sigma \in \text{SEP}} S(\rho || \sigma),
\end{equation}
where $S(\rho || \sigma) = \mathrm{Tr}(\rho \ln \rho - \rho \ln \sigma)$ is
the quantum relative entropy.

\subsection{Detecting Mixed-State Entanglement}
\label{sub:detecting_entanglement}
Distinguishing a separable mixed state from an entangled one is a major
challenge. Several criteria have been developed.

\paragraph{The PPT Criterion.}
A simple but powerful test is the Positive Partial Transpose (PPT)
criterion. If a state $\rho_{AB}$ is separable, then its partial transpose
with respect to one subsystem, $\rho^{T_B}$, must be a valid (positive
semidefinite) density matrix. While this is a necessary condition for
separability in all dimensions, it is only a sufficient one for
$2 \times 2$ and $2 \times 3$ systems. The degree to which a state fails
this test can be quantified by a measure called Negativity.

\paragraph{Worked Example: The Werner State.}
To make this concrete, let's analyse a two-qubit Werner state, a mixture
of a maximally entangled Bell state and a maximally mixed state. Using the
singlet state $\ket{\Psi^-} = \frac{1}{\sqrt{2}}(\ket{01}-\ket{10})$, the
Werner state is defined for $p \in [0, 1]$ as:
\begin{equation}
	\rho_W = p \ket{\Psi^-}\bra{\Psi^-} + \frac{1-p}{4} \mathbb{I}_4.
\end{equation}
In the computational basis $\{\ket{00}, \ket{01}, \ket{10}, \ket{11}\}$, this is:
\begin{equation}
	\rho_W = \frac{1}{4}
	\begin{pmatrix}
		1-p & 0 & 0 & 0 \\
		0 & 1+p & -2p & 0 \\
		0 & -2p & 1+p & 0 \\
		0 & 0 & 0 & 1-p
	\end{pmatrix}.
\end{equation}
We apply the partial transpose to the second qubit (subsystem B):
\begin{equation}
	\rho_W^{T_B} = \frac{1}{4}
	\begin{pmatrix}
		1-p & 0 & 0 & -2p \\
		0 & 1+p & 0 & 0 \\
		0 & 0 & 1+p & 0 \\
		-2p & 0 & 0 & 1-p
	\end{pmatrix}.
\end{equation}
The state is entangled if this matrix has a negative eigenvalue. The four
eigenvalues of $\rho_W^{T_B}$ are found to be:
\begin{equation}
	\lambda_{1,2,3} = \frac{1+p}{4}, \quad \lambda_4 = \frac{1-3p}{4}.
\end{equation}
The first three are always non-negative for $p \in [0,1]$. However,
$\lambda_4$ becomes negative if $1-3p < 0$, which means $p > 1/3$.
Therefore, the Werner state is entangled for $p > 1/3$. The Negativity for
this state is the absolute value of the single negative eigenvalue:
\begin{equation}
	\mathcal{N}(\rho_W) = \max\left(0, -\lambda_4\right) = \max\left(0, \frac{3p-1}{4}\right).
\end{equation}

\paragraph{Entanglement Witnesses.}
An entanglement witness is a Hermitian operator $W$ that acts as a
detector. It is constructed such that its expectation value is non-negative
for all separable states but can be negative for at least one entangled state:
\begin{itemize}
	\item $\mathrm{Tr}(W \rho_{\text{sep}}) \ge 0$ for all $\rho_{\text{sep}} \in \text{SEP}$.
	\item If one measures $\mathrm{Tr}(W \rho) < 0$, the state $\rho$ is certified as entangled.
\end{itemize}
Geometrically, $W$ defines a hyperplane that separates a specific entangled
state from the convex set of separable states, making it a targeted tool for
experimental detection.

\section{Alternative Perspectives}
\label{sec:alternative_perspectives}

While the Hilbert space formalism is the standard, alternative mathematical
frameworks offer different physical insights. One such approach is through
Geometric Algebra, which interprets entanglement not as an intrinsic
property of a state, but as a relationship between reference frames, described
by a superposition of relative rotations. This shifts the focus from analysing
states to analysing the fundamental transformation operators themselves.
