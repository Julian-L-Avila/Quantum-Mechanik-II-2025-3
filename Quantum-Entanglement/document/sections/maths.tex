\section{Mathematical Framework for Entanglement}
\label{sec:math_framework}

Although quantum entanglement first emerged as a mathematical feature of the
early formulations of quantum mechanics, subsequent experimental confirmations
and theoretical advancements have enabled a more rigorous description.
The core of this description lies in how we represent the state of a
composite quantum system. While the framework is well-established, its
physical interpretation---a common theme in quantum theory---remains a
subject of active debate.

\subsection{Departure from Classical State Spaces}
\label{sub:departure_classical}

In any formalism of classical mechanics (e.g., Hamiltonian), the state of a
composite system is constructed from the states of its individual parts.
For a two-particle system, $A$ and $B$, the total state is a point
$(x_A, x_B)$ in the Cartesian product of the individual
phase spaces, $\Gamma_{AB} = \Gamma_A \times \Gamma_B$.
Here, $x_i = (q_i, p_i)$ represents the generalized coordinates and
momenta for particle $i$. This structure implies that specifying the state of
each subsystem completely specifies the state of the whole.

A naive extrapolation to quantum mechanics would suggest that the state space
of a two-particle system is the Cartesian product of the individual Hilbert
spaces, $\mathcal{H}_{AB} = \mathcal{H}_A \times \mathcal{H}_B$. However, this
structure is insufficient because it fails to incorporate the
superposition principle correctly. A state in
$\mathcal{H}_A \times \mathcal{H}_B$ is an ordered pair $(\ket{\psi_A},
\ket{\phi_B})$, but this framework does not naturally accommodate the
linear combinations of such pairs that are physically essential.

To resolve this, the state space for a composite quantum system is
constructed using the \emph{tensor product} of the individual Hilbert
spaces:
\begin{equation}
	\mathcal{H}_{AB} = \mathcal{H}_A \otimes \mathcal{H}_B.
\end{equation}
A general pure state $\ket{\Psi}$ in this composite space is a superposition
of basis states:
\begin{equation}
	\ket{\Psi} = \sum_{i,j} c_{ij} (\ket{a_i} \otimes \ket{b_j}),
	\label{eq:general_state}
\end{equation}
where $\{\ket{a_i}\}$ and $\{\ket{b_j}\}$ are orthonormal bases for
$\mathcal{H}_A$ and $\mathcal{H}_B$, respectively. The dimension of this new
space is the product of the dimensions of the constituent spaces,
$\dim(\mathcal{H}_{AB}) = \dim(\mathcal{H}_A) \cdot \dim(\mathcal{H}_B)$.

\subsection{Separable and Entangled States}
\label{sub:separable_entangled}

The structure of the tensor product space $\mathcal{H}_{AB}$ allows for two
fundamentally different classes of states.

\paragraph{Separable States.}
A state is called \emph{separable} (or a product state) if it
can be written as a single tensor product of states from the subsystems,
i.e.,
\begin{equation}
	\ket{\Psi}_{\text{sep}} = \ket{\psi_A} \otimes \ket{\psi_B},
\end{equation}
where $\ket{\psi_A} \in \mathcal{H}_A$ and $\ket{\psi_B} \in \mathcal{H}_B$.
For these states, the subsystems have definite individual properties,
independent of one another, analogous to the classical case.

\paragraph{Entangled States.}
If a state $\ket{\Psi} \in \mathcal{H}_{AB}$ cannot be written in the
separable form above, it is called an \emph{entangled state}. These states
represent a uniquely quantum-mechanical correlation between subsystems.
Measuring a property of subsystem $A$ instantaneously influences the
properties of subsystem $B$, regardless of the physical distance
separating them.

\subsection{Operators and the Density Matrix Formalism}
\label{sub:operators_density}

An operator $O_A$ acting only on subsystem $A$ is represented on the
composite space as $O_A \otimes I_B$, where $I_B$ is the
identity operator on $\mathcal{H}_B$. Such operators are called
local operators. A general operator on $\mathcal{H}_{AB}$ is a
linear combination of tensor products of local operators.

The inner product on $\mathcal{H}_{AB}$ is defined by extending the inner
products of the subsystems linearly:
\begin{equation}
	(\bra{\phi_A} \otimes \bra{\phi_B}) (\ket{\psi_A} \otimes \ket{\psi_B})
	= \braket{\phi_A | \psi_A}_{\mathcal{H}_A} \braket{\phi_B | \psi_B}_{\mathcal{H}_B}.
\end{equation}

While pure states $\ket{\Psi}$ describe systems with maximal information,
entanglement forces us to consider situations of incomplete information,
especially when looking at only one part of an entangled pair. The
density matrix (or density operator), $\rho$, provides the necessary
formalism.

For a pure state $\ket{\Psi}$, the density matrix is the projection operator:
\begin{equation}
	\rho_{\text{pure}} = \ket{\Psi}\bra{\Psi}.
\end{equation}
A key property of a pure state's density matrix is that it is a projector,
meaning $\rho^2 = \rho$. This leads to a simple test for purity:
\begin{equation}
	\mathrm{Tr}(\rho^2) = 1,
\end{equation}
where the value $\mathrm{Tr}(\rho^2)$ is known as the \textit{purity} of the
state.

If we have statistical knowledge of a system (an ensemble of pure states
$\{\ket{\psi_i}\}$ with probabilities $\{p_i\}$), the state is a
\emph{mixed state}, described by the density matrix:
\begin{equation}
	\rho_{\text{mixed}} = \sum_i p_i \ket{\psi_i}\bra{\psi_i}.
\end{equation}
For any mixed state, the purity is less than one:
$\mathrm{Tr}(\rho_{\text{mixed}}^2) < 1$.
