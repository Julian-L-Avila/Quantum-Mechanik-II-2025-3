\section{Mathematical Framework for Entanglement}
\label{sec:math_framework}

Although quantum entanglement first emerged as a mathematical feature of the
early formulations of quantum mechanics, subsequent experimental confirmations
and theoretical advancements have enabled a more rigorous description.
The core of this description lies in how we represent the state of a
composite quantum system. While the framework is well-established, its
physical interpretation---a common theme in quantum theory---remains a
subject of active debate.

\subsection{Departure from Classical State Spaces}
\label{sub:departure_classical}

In any formalism of classical mechanics (e.g., Hamiltonian), the state of a
composite system is constructed from the states of its individual parts.
For a two-particle system, $A$ and $B$, the total state is a point
$(x_A, x_B)$ in the Cartesian product of the individual
phase spaces, $\Gamma_{AB} = \Gamma_A \times \Gamma_B$.
Here, $x_i = (q_i, p_i)$ represents the generalized coordinates and
momenta for particle $i$. This structure implies that specifying the state of
each subsystem completely specifies the state of the whole.

A naive extrapolation to quantum mechanics would suggest that the state space
of a two-particle system is the Cartesian product of the individual Hilbert
spaces, $\mathcal{H}_{AB} = \mathcal{H}_A \times \mathcal{H}_B$. However, this
structure is insufficient because it fails to incorporate the
superposition principle correctly. A state in
$\mathcal{H}_A \times \mathcal{H}_B$ is an ordered pair $(\ket{\psi_A},
\ket{\phi_B})$, but this framework does not naturally accommodate the
linear combinations of such pairs that are physically essential.

To resolve this, the state space for a composite quantum system is
constructed using the tensor product of the individual Hilbert
spaces:
\begin{equation}
	\mathcal{H}_{AB} = \mathcal{H}_A \otimes \mathcal{H}_B.
\end{equation}
A general pure state $\ket{\Psi}$ in this composite space is a superposition
of basis states:
\begin{equation}
	\ket{\Psi} = \sum_{i,j} c_{ij} (\ket{a_i} \otimes \ket{b_j}),
	\label{eq:general_state}
\end{equation}
where $\{\ket{a_i}\}$ and $\{\ket{b_j}\}$ are orthonormal bases for
$\mathcal{H}_A$ and $\mathcal{H}_B$, respectively. The dimension of this new
space is the product of the dimensions of the constituent spaces,
$\dim(\mathcal{H}_{AB}) = \dim(\mathcal{H}_A) \cdot \dim(\mathcal{H}_B)$.

\subsection{Separable and Entangled States}
\label{sub:separable_entangled}

The structure of the tensor product space $\mathcal{H}_{AB}$ allows for two
fundamentally different classes of states.

\paragraph{Separable States.}
A state is called separable (or a product state) if it
can be written as a single tensor product of states from the subsystems,
i.e.,
\begin{equation}
	\ket{\Psi}_{\text{sep}} = \ket{\psi_A} \otimes \ket{\psi_B},
\end{equation}
where $\ket{\psi_A} \in \mathcal{H}_A$ and $\ket{\psi_B} \in \mathcal{H}_B$.
For these states, the subsystems have definite individual properties,
independent of one another, analogous to the classical case.

\paragraph{Entangled States.}
If a state $\ket{\Psi} \in \mathcal{H}_{AB}$ cannot be written in the
separable form above, it is called an entangled state. These states
represent a uniquely quantum-mechanical correlation between subsystems.
Measuring a property of subsystem $A$ instantaneously influences the
properties of subsystem $B$, regardless of the physical distance
separating them.

\subsection{Indistinguishable Particles and Forced Entanglement}
\label{sub:indistinguishable_particles}

The formalism developed thus far, based on the tensor product structure
$\mathcal{H}_{AB} = \mathcal{H}_A \otimes \mathcal{H}_B$, implicitly assumes that the
constituent subsystems, $A$ and $B$, are distinguishable. One can, at least
in principle, assign a permanent label to each particle. However, when
dealing with a system of identical particles, such as two electrons, quantum
mechanics imposes a fundamental constraint known as the symmetrization
postulate. This postulate states that the total wave function for a
collection of identical particles must be either symmetric (for bosons) or
antisymmetric (for fermions) under the exchange of any two particles.

This requirement fundamentally alters the structure of the state space and the
very notion of entanglement. For two identical fermions, the allowed states do
not live in the full tensor product space $\mathcal{H} \otimes \mathcal{H}$,
but in its antisymmetric subspace, often denoted by
$\mathcal{H} \wedge \mathcal{H}$. Consequently, the standard definition of
separability based on partitioning the system by particle label is no longer
applicable. A state like $\ket{\psi_A} \otimes \ket{\phi_B}$ is not
physically meaningful because the labels $A$ and $B$ are arbitrary.

\paragraph{Redefining Entanglement for Fermions.}
For systems of identical particles, entanglement is reformulated based on the
correlations between different degrees of freedom (e.g., spatial and spin)
or between modes occupied by the particles, rather than between the particles
themselves. The role of a separable (unentangled) state is played by a
Slater determinant -- it is the exterior product in clifford formalism --, which
for two fermions in single-particle states
$\ket{\phi_1}$ and $\ket{\phi_2}$ is given by:
\begin{equation}
	\ket{\Psi}_{\text{Slater}} = \frac{1}{\sqrt{2}}
	\left( \ket{\phi_1}_1 \otimes \ket{\phi_2}_2 -
	\ket{\phi_2}_1 \otimes \ket{\phi_1}_2 \right).
\end{equation}
A general pure state of two fermions is a superposition of such Slater
determinants. The entanglement of the state is then quantified by the minimum
number of Slater determinants required to represent it, a quantity known as
the Slater rank. A pure fermionic state is considered entangled if
and only if its Slater rank is greater than one. This is the direct analogue
to the Schmidt rank criterion for distinguishable particles.

\paragraph{Example: Pauli Exclusion and Forced Entanglement.}
A canonical example is the case of two electrons in the same spatial orbital
within an atom. Electrons are spin-$1/2$ fermions, and their total wave
function $\ket{\Psi}_{\text{total}} = \ket{\psi}_{\text{spatial}} \otimes
\ket{\chi}_{\text{spin}}$ must be antisymmetric.
\begin{itemize}
	\item \textbf{Spatial State:} Since both electrons occupy the same
		orbital, their spatial wave function $\ket{\psi}_{\text{spatial}}$
		is symmetric under particle exchange.
	\item \textbf{Spin State:} To ensure the total wave function is
		antisymmetric, the spin part $\ket{\chi}_{\text{spin}}$ must be
		antisymmetric. For two spin-$1/2$ particles, the unique (up to a
		phase) antisymmetric spin state is the spin-singlet state:
		\begin{equation}
			\ket{\Psi^-} = \frac{1}{\sqrt{2}} (\ket{\uparrow}_1 \otimes
			\ket{\downarrow}_2 - \ket{\downarrow}_1 \otimes \ket{\uparrow}_2).
		\end{equation}
\end{itemize}
This singlet state is a maximally entangled Bell state. Therefore, the Pauli
exclusion principle \textit{forces} the two electrons into a maximally
entangled spin state. This is a profound departure from the case of
distinguishable particles, where entanglement is a possible configuration but
not a necessity.

In summary, for distinguishable systems, entanglement is a property that may
or may not be present in a state prepared within the tensor product space.
For indistinguishable fermions, the algebraic structure imposed by the
symmetrization postulate constrains the available states, and in cases like
two electrons in the same orbital, this constraint mandates the existence of
entanglement.

\subsection{Operators and the Density Matrix Formalism}
\label{sub:operators_density}

An operator $O_A$ acting only on subsystem $A$ is represented on the
composite space as $O_A \otimes I_B$, where $I_B$ is the
identity operator on $\mathcal{H}_B$. Such operators are called
local operators. A general operator on $\mathcal{H}_{AB}$ is a
linear combination of tensor products of local operators.

The inner product on $\mathcal{H}_{AB}$ is defined by extending the inner
products of the subsystems linearly:
\begin{equation}
	(\bra{\phi_A} \otimes \bra{\phi_B}) (\ket{\psi_A} \otimes \ket{\psi_B})
	= \braket{\phi_A | \psi_A}_{\mathcal{H}_A} \braket{\phi_B | \psi_B}_{\mathcal{H}_B}.
\end{equation}

While pure states $\ket{\Psi}$ describe systems with maximal information,
entanglement forces us to consider situations of incomplete information,
especially when looking at only one part of an entangled pair. The
density matrix (or density operator), $\rho$, provides the necessary
formalism.

For a pure state $\ket{\Psi}$, the density matrix is the projection operator:
\begin{equation}
	\rho_{\text{pure}} = \ket{\Psi}\bra{\Psi}.
\end{equation}
A key property of a pure state's density matrix is that it is a projector,
meaning $\rho^2 = \rho$. This leads to a simple test for purity:
\begin{equation}
	\mathrm{Tr}(\rho^2) = 1,
\end{equation}
where the value $\mathrm{Tr}(\rho^2)$ is known as the \textit{purity} of the
state.

If we have statistical knowledge of a system (an ensemble of pure states
$\{\ket{\psi_i}\}$ with probabilities $\{p_i\}$), the state is a
mixed state, described by the density matrix:
\begin{equation}
	\rho_{\text{mixed}} = \sum_i p_i \ket{\psi_i}\bra{\psi_i}.
\end{equation}
For any mixed state, the purity is less than one:
$\mathrm{Tr}(\rho_{\text{mixed}}^2) < 1$.

\subsection{Quantifying Uncertainty: Von Neumann Entropy}
\label{sub:von_neumann_entropy}

The density matrix allows us to assign a single number to quantify the
degree of uncertainty or ``mixedness'' of a quantum state. This measure is the
Von Neumann entropy, defined as:
\begin{equation}
	S(\rho) = - \mathrm{Tr}(\rho \ln \rho).
\end{equation}
This is the quantum mechanical analogue of the Gibbs entropy in statistical
mechanics. If the eigenvalues of $\rho$ are $\lambda_i$, the entropy can be
calculated as $S = - \sum_i \lambda_i \ln \lambda_i$.

The Von Neumann entropy has several key properties:
\begin{itemize}
	\item For any pure state, $\rho = \ket{\psi}\bra{\psi}$, the
		entropy is zero, $S(\rho) = 0$. This corresponds to a state of
		maximal knowledge or zero uncertainty.
	\item For a maximally mixed state in a $d$-dimensional Hilbert
		space, where $\rho = I/d$, the entropy is maximal,
		$S(\rho) = \ln d$. This corresponds to a state of minimal knowledge.
\end{itemize}

The true power of this measure becomes apparent when applied to entangled
systems. Consider a composite system $AB$ in a pure state
$\ket{\Psi}_{AB}$, for which $S(\rho_{AB})=0$. If we are interested only in
subsystem $A$, its state is described by the reduced density matrix,
$\rho_A$, obtained by tracing over subsystem $B$:
\begin{equation}
	\rho_A = \mathrm{Tr}_B(\rho_{AB}) = \mathrm{Tr}_B(\ket{\Psi}_{AB}\bra{\Psi}_{AB}).
\end{equation}
If the global state $\ket{\Psi}_{AB}$ is entangled, the local state $\rho_A$
will be a mixed state, and its entropy $S(\rho_A)$ will be greater than zero.
This non-zero entropy of a subsystem, despite the global system being in a
pure state, is a fundamental signature of entanglement. For any pure
bipartite state, $S(\rho_A) = S(\rho_B)$, and this value is used as the
entropy of entanglement.

\section{Quantifying and Detecting Entanglement}
\label{sec:quantifying_detecting}

With the mathematical framework established, we need tools to answer two
practical questions: How much entanglement does a state possess? And how can
we experimentally detect it, especially in mixed states?

\subsection{Pure States: Schmidt Decomposition and Entropy of Entanglement}
\label{sub:pure_state_measures}

For pure bipartite states, a powerful theorem known as the
Schmidt decomposition provides a canonical form. Any state
$\ket{\Psi}_{AB}$ can be written as a single sum:
\begin{equation}
	\ket{\Psi}_{AB} = \sum_{k=1}^{r_S} \sqrt{\lambda_k} \ket{k}_A \otimes \ket{k}_B,
\end{equation}
where $\{\ket{k}_A\}$ and $\{\ket{k}_B\}$ are orthonormal bases for their
respective subsystems, and $\lambda_k > 0$ with $\sum_k \lambda_k = 1$. The
number of terms, $r_S$, is the Schmidt rank. A state is separable if
and only if its Schmidt rank is 1.

The degree of entanglement is uniquely quantified by the
entropy of entanglement. This is simply the Von Neumann entropy of
the reduced density matrix of either subsystem:
\begin{equation}
	E(\ket{\Psi}) = S(\rho_A) = -\mathrm{Tr}(\rho_A \ln \rho_A) = -\sum_{k=1}^{r_S} \lambda_k \ln \lambda_k.
\end{equation}
The Schmidt coefficients $\sqrt{\lambda_k}$ are directly related to the
eigenvalues of the reduced density matrix, elegantly connecting the
state's structure to its entanglement content.

\subsection{Mixed States: A Zoo of Measures}
\label{sub:mixed_state_measures}
Quantifying entanglement in mixed states is far more complex, as they lack
a unique decomposition. Several measures exist, each with a different
operational or geometric motivation.

\paragraph{Entanglement of Formation (\texorpdfstring{$E_F$}{E\_F}).}
This measure asks: what is the minimum average entanglement (of pure states)
required to create a given mixed state $\rho$? It is defined via a
``convex roof'' construction:
\begin{equation}
	E_F(\rho) = \min_{\{p_i, \ket{\psi_i}\}} \sum_i p_i E(\ket{\psi_i}),
\end{equation}
where the minimization is over all possible pure-state ensembles
$\{p_i, \ket{\psi_i}\}$ that decompose $\rho$.

\paragraph{Concurrence (\texorpdfstring{$C$}{C}).}
For the specific case of two-qubit systems, the Concurrence is a widely
used measure computationally related to $E_F$. It has the great advantage
of having a closed-form analytical solution, making it highly practical.

\paragraph{Relative Entropy of Entanglement (\texorpdfstring{$E_R$}{E\_R}).}
This measure takes a geometric approach, defining the entanglement of a
state $\rho$ as its ``distance'' to the set of all separable states (SEP):
\begin{equation}
	E_R(\rho) = \min_{\sigma \in \text{SEP}} S(\rho || \sigma),
\end{equation}
where $S(\rho || \sigma) = \mathrm{Tr}(\rho \ln \rho - \rho \ln \sigma)$ is
the quantum relative entropy.

\subsection{Detecting Mixed-State Entanglement}
\label{sub:detecting_entanglement}
Distinguishing a separable mixed state from an entangled one is a major
challenge. Several criteria have been developed.

\paragraph{The PPT Criterion.}
A simple but powerful test is the Positive Partial Transpose (PPT)
criterion. If a state $\rho_{AB}$ is separable, then its partial transpose
with respect to one subsystem, $\rho^{T_B}$, must be a valid (positive
semidefinite) density matrix. While this is a necessary condition for
separability in all dimensions, it is only a sufficient one for
$2 \times 2$ and $2 \times 3$ systems. The degree to which a state fails
this test can be quantified by a measure called Negativity.

\paragraph{Worked Example: The Werner State.}
To make this concrete, let's analyse a two-qubit Werner state, a mixture
of a maximally entangled Bell state and a maximally mixed state. Using the
singlet state $\ket{\Psi^-} = \frac{1}{\sqrt{2}}(\ket{01}-\ket{10})$, the
Werner state is defined for $p \in [0, 1]$ as:
\begin{equation}
	\rho_W = p \ket{\Psi^-}\bra{\Psi^-} + \frac{1-p}{4} \mathbb{I}_4.
\end{equation}
In the computational basis $\{\ket{00}, \ket{01}, \ket{10}, \ket{11}\}$, this is:
\begin{equation}
	\rho_W = \frac{1}{4}
	\begin{pmatrix}
		1-p & 0 & 0 & 0 \\
		0 & 1+p & -2p & 0 \\
		0 & -2p & 1+p & 0 \\
		0 & 0 & 0 & 1-p
	\end{pmatrix}.
\end{equation}
We apply the partial transpose to the second qubit (subsystem B):
\begin{equation}
	\rho_W^{T_B} = \frac{1}{4}
	\begin{pmatrix}
		1-p & 0 & 0 & -2p \\
		0 & 1+p & 0 & 0 \\
		0 & 0 & 1+p & 0 \\
		-2p & 0 & 0 & 1-p
	\end{pmatrix}.
\end{equation}
The state is entangled if this matrix has a negative eigenvalue. The four
eigenvalues of $\rho_W^{T_B}$ are found to be:
\begin{equation}
	\lambda_{1,2,3} = \frac{1+p}{4}, \quad \lambda_4 = \frac{1-3p}{4}.
\end{equation}
The first three are always non-negative for $p \in [0,1]$. However,
$\lambda_4$ becomes negative if $1-3p < 0$, which means $p > 1/3$.
Therefore, the Werner state is entangled for $p > 1/3$. The Negativity for
this state is the absolute value of the single negative eigenvalue:
\begin{equation}
	\mathcal{N}(\rho_W) = \max\left(0, -\lambda_4\right) = \max\left(0, \frac{3p-1}{4}\right).
\end{equation}

\paragraph{Entanglement Witnesses.}
An entanglement witness is a Hermitian operator $W$ that acts as a
detector. It is constructed such that its expectation value is non-negative
for all separable states but can be negative for at least one entangled state:
\begin{itemize}
	\item $\mathrm{Tr}(W \rho_{\text{sep}}) \ge 0$ for all $\rho_{\text{sep}} \in \text{SEP}$.
	\item If one measures $\mathrm{Tr}(W \rho) < 0$, the state $\rho$ is certified as entangled.
\end{itemize}
Geometrically, $W$ defines a hyperplane that separates a specific entangled
state from the convex set of separable states, making it a targeted tool for
experimental detection.

\section{Dynamics of Entanglement}
\label{sec:entanglement_dynamics}

Beyond the static characterization of entangled states, a crucial area of
study concerns how entanglement evolves in time. The dynamics are typically
governed by the system's Hamiltonian, its interaction with an external
environment, or abrupt changes to its governing parameters. Understanding
these dynamics is paramount for quantum information processing, where entangled
states must be generated, manipulated, and protected from decoherence.

\subsection{Propagation in Closed Many-Body Systems}
\label{sub:propagation}

In a many-body system with local interactions, entanglement is not a static
property but can propagate through the system. If an entangled pair is
created locally in a quantum spin chain, the correlations will spread over
time. A fundamental result in this area is the existence of a finite maximum
speed for the propagation of information, rigorously established by the
Lieb-Robinson bounds, which imply a linear ``light cone'' for
correlations.
\begin{itemize}
	\item \textbf{Ballistic Spreading:} In clean, one-dimensional systems
		like the XX model, entanglement spreads ballistically. The time $t$
		it takes for entanglement to propagate over a distance of $d$ lattice
		sites is proportional to the inverse of the coupling strength $J$,
		i.e., $t \approx d/J$. This establishes a finite propagation velocity
		determined by the system's microscopic parameters.
	\item \textbf{Entanglement Waves:} In harmonic systems, which can be
		described by Gaussian states, the propagation is characterized by an
		``entanglement wave''. This wave arrives at a distant site after a
		finite time determined by the sound velocity of the underlying model,
		after which the entanglement exhibits damped oscillations.
	\item \textbf{Non-Ergodic Behaviour:} Unlike classical thermal systems,
		the entanglement dynamics in integrable quantum systems (like the XY
		chain) can be non-ergodic. The two-site entanglement does not
		necessarily relax to a steady equilibrium value, instead exhibiting
		persistent oscillations.
\end{itemize}

\subsection{Non-Equilibrium Dynamics: Quantum Quenches}
\label{sub:quenches}

A quantum quench---a sudden change in a system's Hamiltonian---drives
the system into a non-equilibrium state, revealing universal features of
entanglement dynamics. A common protocol involves preparing a system in the
ground state of a Hamiltonian $H_0$ and then evolving it with a different
Hamiltonian $H_1$.

\paragraph{Block Entropy Growth and Quasiparticle Picture.}
A key result, particularly for 1D systems analysed via conformal field theory
(CFT), describes the evolution of the von Neumann entropy $S_A(t)$ of a
subsystem (block) of length $\ell$. After a quench from a gapped to a
critical phase (e.g., in the quantum Ising model), the entropy exhibits a
characteristic pattern:
\begin{enumerate}
	\item An initial linear growth in time: $S_A(t) \propto t$.
	\item Saturation to a value proportional to the block size: $S_A(t) \to \text{const} \cdot \ell$.
\end{enumerate}
This behaviour is elegantly explained by a quasiparticle picture. The quench
is viewed as creating pairs of entangled quasiparticles at every point in
space, which then propagate freely in opposite directions. In spin systems, these
quasiparticles are magnons, and their propagation velocity $v$ corresponds to
the spin wave velocity. A block $A$ becomes entangled with the rest
of the system when one member of a pair enters $A$ while the other does not.
The linear growth persists until quasiparticles created at the centre of the
block have had time to exit it, leading to saturation at a time $t^* \approx \ell/(2v)$.

\paragraph{Dynamical Phase Transitions.}
Entanglement can also signal dynamical phase transitions, which are
non-analytic behaviours in observables as a function of time. For instance,
if the transverse magnetic field in a spin chain is suddenly quenched to zero,
the two-site entanglement can periodically vanish and reappear at specific
times, exhibiting critical behaviour as a function of the initial field
strength that is not apparent in local observables like magnetization.

\subsection{Generation, Control, and Chaos}
\label{sub:generation_control}

The natural time evolution of many-body systems can be harnessed as a
resource for creating and manipulating complex entangled states essential for
quantum computation.

\begin{itemize}
	\item \textbf{State Generation:} Specific Hamiltonians can generate
		highly entangled states from simple product state inputs. For example,
		the evolution of short XX spin chains can produce generalized
		W-states at discrete time intervals. Similarly, evolving a
		product state under an Ising-type interaction can generate
		cluster states, which are the universal resource for
		one-way quantum computing.
	\item \textbf{Entanglement Extraction:} Entanglement residing in the
		equilibrium state of a many-body system can be ``swapped'' to external
		probes. By locally interacting a pair of probe qubits with an
		entangled spin chain, the entanglement can be transferred from the
		chain to the probes, demonstrating a genuinely non-local transfer
		mechanism.
	\item \textbf{Chaos and Disorder:} The dynamics of entanglement are highly
		sensitive to the system's spectral properties. In systems that can be
		tuned towards quantum chaos, or in the presence of disorder, the
		evolution of entanglement serves as a powerful diagnostic tool to
		distinguish between different dynamical regimes, such as perturbative,
		chaotic, or many-body localized phases.
\end{itemize}

\subsection{Entanglement in Open Systems: Decay and Sudden Death}
\label{sub:decay}

When a quantum system interacts with an external environment, its quantum
coherence and entanglement degrade over time---a process known as
decoherence. Entanglement is typically far more fragile than the coherence of
individual subsystems.

A striking feature of this decay is the phenomenon of
Entanglement Sudden Death (ESD). Contrary to the typical asymptotic
decay of local properties (like the polarization of a single spin), the
entanglement between two subsystems can vanish completely at a finite time.
After this point, the system evolves in a purely separable state, even
though the individual subsystems have not yet fully decohered. This dramatic,
non-classical decay pattern highlights the unique fragility of quantum
correlations and has been confirmed in numerous experiments.

\section{Alternative Perspectives}
\label{sec:alternative_perspectives}

While the Hilbert space formalism is the standard, alternative mathematical
frameworks can offer different physical insights by reformulating the fundamental
nature of quantum phenomena. One such approach, rooted in Clifford (Geometric)
Algebra, re-contextualizes entanglement itself.

\subsection{A Geometric Interpretation: The Clifford Formalism}
\label{sub:clifford_formalism}

This perspective posits that entanglement is not an intrinsic property of a
composite quantum state, but rather a relationship between the reference
frames in which the constituent states are defined. The goal is to ground
the ``spooky'' non-locality of quantum mechanics in a more tangible geometric
foundation.

\paragraph{The Algebra of Physical Space (APS).}
The mathematical framework for this interpretation is Geometric Algebra,
specifically the Algebra of Physical Space (APS). APS is a powerful language
for expressing geometric relationships that may be obscured in the standard
matrix formulation. A key feature is its geometric representation of spin, where
spin states are treated as rotors (operators that perform rotations).
For example, the spinor for a spin-down state, $\psi_{\downarrow}$, can be
interpreted as a $\pi$-rotation about the $e_2$ axis acting on the spin-up
spinor, $\psi_{\uparrow}$. This approach prioritizes the study of
transformation operators over the resulting states.

\paragraph{Entanglement as a Transformation Operator (The Entanglor).}
The core thesis of this view is that entanglement is encoded within a
transformation operator, termed an entanglor, rather than the state
itself. An entangled state is seen as the result of an entanglor acting on
a separable state. These entanglors are derived by factoring the Bell states.
A Bell state can be expressed as a linear superposition of bipartite rotors
acting on a simple separable state, such as the bipartite spin-up state
$\Upsilon_{\uparrow\uparrow} \equiv \psi_{\uparrow} \otimes \psi_{\uparrow}$.

The entanglor operator contains all the geometric information describing the
superposition of relative rotations between the reference frames. The four
entanglors, denoted $R_{\Psi^{\pm}}$ and $R_{\Phi^{\pm}}$, are themselves
interrelated by simpler transformations, such as single-subspace rotors
($R_{\pi} \otimes 1$ or $1 \otimes R_{\pi}$), confirming their fundamental
connection as a family of geometric transformations.

\paragraph{Relativistic Context and Entangling Eigenspinors.}
A significant advantage of APS is its natural compatibility with special
relativity. This allows the concept of entanglement, now embodied by the
entanglor, to be subjected to Lorentz transformations (boosts). In this view,
a boost represents the information relating two reference frames' relative
velocities.

By combining a separable bipartite boost operator ($B$) with an entanglor
($R_{X^{\pm}}$), one can construct an entangling eigenspinor,
$\Lambda_{AB}$:
\begin{equation}
	\Lambda_{AB} \equiv B \cdot R_{X^{\pm}}.
\end{equation}
This composite operator, $\Lambda_{AB}$, contains the complete information
about both the relative velocities (the boost) and the entanglement
(the superposition of relative rotations) between the two subsystems'
reference frames. It has been proposed that these entangling eigenspinors are
solutions to a bipartite form of the classical Dirac equation.

\paragraph{Conceptual and Interpretive Implications.}
This geometric reinterpretation has profound conceptual implications. The
complete physical description of the system's correlations and relative
motion is contained within the entangling eigenspinor, $\Lambda_{AB}$.
Applying this transformation to a state and then using projectors to analyse
the resulting spin state provides an incomplete picture, as the underlying
geometric information is lost.

This underscores the necessity of treating entanglement as information encoded
in the transformation itself. The ``weirdness'' of quantum mechanics is shifted
from the state to the process: the wave function collapse is not a physical
change in a spooky object, but the revelation of which specific geometric
transformation (relative rotation) described the relationship between the
particles. In this view, quantum processes and transformations are more
fundamental than quantum states.
