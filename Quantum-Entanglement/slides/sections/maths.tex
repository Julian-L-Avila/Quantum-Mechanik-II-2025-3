%==============================================================================
\section{Mathematical Framework for Entanglement}
%==============================================================================

%------------------------------------------------------------------------------
\begin{frame}{Motivation: From Feature to Formalism}
  \begin{block}{The Origin of Entanglement}
    \begin{itemize}
      \item Emerged as a mathematical feature of early quantum mechanics.
      \pause
      \item Now rigorously described via experiment and theory.
      \pause
      \item Core Idea: How to represent states of \emph{composite quantum
        systems}.
    \end{itemize}
  \end{block}
  \pause

  \begin{alertblock}{A Note on Interpretation}
    While the math is well-established, its physical interpretation remains a
    subject of active debate---a common theme in quantum theory.
  \end{alertblock}
\end{frame}

%------------------------------------------------------------------------------
\begin{frame}{Departure from Classical State Spaces}
  \begin{block}{Classical Composite Systems}
    For a two-particle system, the total state space is the
    \alert{Cartesian product} of individual phase spaces:
    \[
      \Gamma_{AB} = \Gamma_A \times \Gamma_B
    \]
    The state $(x_A, x_B)$ of the parts completely defines the whole.
  \end{block}
  \pause
  \begin{alertblock}{A Naive Quantum Extrapolation}
    A simple guess might be $\mathcal{H}_{AB} = \mathcal{H}_A \times
    \mathcal{H}_B$. \pause \textbf{This is incorrect.}
    \begin{itemize}
      \item It fails to incorporate the \emph{superposition principle} for
        combined states.
      \item An ordered pair $(\ket{\psi_A}, \ket{\phi_B})$ cannot represent
        linear combinations.
    \end{itemize}
  \end{alertblock}
\end{frame}

%------------------------------------------------------------------------------
\begin{frame}{The Tensor Product Structure}
  \begin{block}{The Correct Postulate}
    The state space for a composite quantum system is the
    \emph{tensor product} of the individual Hilbert spaces:
    \begin{equation}
      \mathcal{H}_{AB} = \mathcal{H}_A \otimes \mathcal{H}_B
    \end{equation}
  \end{block}
  \pause
  \begin{block}{General Pure State}
    A general pure state $\ket{\Psi} \in \mathcal{H}_{AB}$ is a superposition:
    \begin{equation}
      \ket{\Psi} = \sum_{i,j} c_{ij} (\ket{a_i} \otimes \ket{b_j})
    \end{equation}
    where $\{\ket{a_i}\}$ and $\{\ket{b_j}\}$ are orthonormal bases for
    $\mathcal{H}_A$ and $\mathcal{H}_B$.
    \pause
    \begin{center}
      Dimension:
      \( \dim(\mathcal{H}_{AB}) = \dim(\mathcal{H}_A) \cdot \dim(\mathcal{H}_B) \)
    \end{center}
  \end{block}
\end{frame}

%------------------------------------------------------------------------------
\begin{frame}{Two Fundamental Classes of States}
  \begin{columns}[T]
    \begin{column}{0.5\textwidth}
      \begin{block}{Separable (Product) States}
        A state is \emph{separable} if it can be written as a single tensor product:
        \begin{equation}
          \ket{\Psi}_{\text{sep}} = \ket{\psi_A} \otimes \ket{\psi_B}
        \end{equation}
        Subsystems have definite, independent properties (classical analogue).
      \end{block}
    \end{column}
    \pause
    \begin{column}{0.5\textwidth}
      \begin{alertblock}{Entangled States}
        A state is \emph{entangled} if it \emph{cannot} be written in the
        separable form.
        \begin{itemize}
          \item Uniquely quantum-mechanical correlation.
          \item Measurement on A instantly affects B.
        \end{itemize}
      \end{alertblock}
    \end{column}
  \end{columns}
\end{frame}

%------------------------------------------------------------------------------
\begin{frame}{A Complication: Identical Particles}
  \begin{block}{The Symmetrization Postulate}
    The formalism $\mathcal{H}_{AB} = \mathcal{H}_A \otimes \mathcal{H}_B$
    implicitly assumes particles are distinguishable.
    \pause

    For identical particles, quantum mechanics imposes the
    \alert{symmetrization postulate}:
    \begin{itemize}[<+->]
      \item Total wave function must be \textbf{symmetric} under particle
        exchange for \textbf{bosons}.
      \item Total wave function must be \textbf{antisymmetric} under particle
        exchange for \textbf{fermions}.
    \end{itemize}
  \end{block}
\end{frame}

%------------------------------------------------------------------------------
\begin{frame}{Consequence for Fermions}
  \begin{alertblock}{A Restricted State Space}
    \begin{itemize}
      \item Physically allowed states live in the \emph{antisymmetric
        subspace} $\mathcal{H} \wedge \mathcal{H}$, not the full tensor
        product space $\mathcal{H} \otimes \mathcal{H}$.
      \pause
      \item The standard definition of separability, which relies on particle
        labels (`particle A' vs `particle B'), becomes inapplicable.
    \end{itemize}
  \end{alertblock}
\end{frame}

%------------------------------------------------------------------------------
\begin{frame}{Entanglement for Identical Particles}
  \begin{block}{A New Paradigm: Mode Entanglement}
    Entanglement is reformulated based on correlations between modes (e.g.,
    spatial vs. spin), not between labeled particles.
    \pause
    \begin{itemize}
      \item The role of a separable state is now played by a single
        \alert{Slater determinant}.
    \end{itemize}
    \pause
    For two fermions in single-particle states $\ket{\phi_1}, \ket{\phi_2}$:
    \[
      \ket{\Psi}_{\text{Slater}} = \frac{1}{\sqrt{2}}
      \left( \ket{\phi_1}_1 \otimes \ket{\phi_2}_2 -
      \ket{\phi_2}_1 \otimes \ket{\phi_1}_2 \right)
    \]
  \end{block}
\end{frame}

%------------------------------------------------------------------------------
\begin{frame}{Defining Fermionic Entanglement: Slater Rank}
  \begin{alertblock}{Condition for Entanglement}
    A pure fermionic state is considered \alert{entangled if and only if
    its Slater rank is greater than one}.
    \pause
    \begin{itemize}
      \item This means it requires a superposition of multiple Slater
        determinants to be described.
      \item Slater Rank 1 $\iff$ ``Separable'' (unentangled)
      \item Slater Rank $>$ 1 $\iff$ Entangled
    \end{itemize}
  \end{alertblock}
\end{frame}

%------------------------------------------------------------------------------
\begin{frame}{Example: Pauli Exclusion \emph{Forces} Entanglement}
  \begin{block}{Two Electrons in the Same Spatial Orbital}
    The total wave function $\ket{\Psi}_{\text{total}} =
    \ket{\psi}_{\text{spatial}} \otimes \ket{\chi}_{\text{spin}}$ must be
    antisymmetric.
    \begin{enumerate}[<+->]
      \item \textbf{Spatial State:} $\ket{\psi}_{\text{spatial}}$ is
        \emph{symmetric} (same orbital).
      \item \textbf{Spin State:} To ensure total antisymmetry,
        $\ket{\chi}_{\text{spin}}$ \emph{must be antisymmetric}.
      \item \textbf{Result:} The unique antisymmetric spin state for two
        spin-1/2 particles is the maximally entangled \alert{spin-singlet}:
        \[
          \ket{\Psi^-} = \frac{1}{\sqrt{2}} (\ket{\uparrow}_1 \otimes
          \ket{\downarrow}_2 - \ket{\downarrow}_1 \otimes \ket{\uparrow}_2)
        \]
    \end{enumerate}
  \end{block}
\end{frame}

%------------------------------------------------------------------------------
\begin{frame}{Example: Pauli Exclusion (Conclusion)}
  \begin{alertblock}{Conclusion}
    The Pauli exclusion principle \emph{forces} the two electrons into a
    maximally entangled Bell state.
    \pause
    \begin{center}
    For indistinguishable particles, entanglement is not just a possibility
    but can be a \textbf{necessity} imposed by fundamental symmetries.
    \end{center}
  \end{alertblock}
\end{frame}

%------------------------------------------------------------------------------
\begin{frame}{Formalism: Operators on Composite Systems}
    \begin{block}{Local Operators}
      An operator $O_A$ on subsystem $A$ is represented on
      $\mathcal{H}_{AB}$ as a \alert{local operator}:
      \[
        O_A \to O_A \otimes I_B
      \]
    \end{block}
    \pause
    \begin{block}{Inner Product}
      The inner product is defined by linear extension:
      \[
        (\bra{\phi_A} \otimes \bra{\phi_B}) (\ket{\psi_A} \otimes
        \ket{\psi_B}) = \braket{\phi_A}{\psi_A} \braket{\phi_B}{\psi_B}
      \]
    \end{block}
\end{frame}

%------------------------------------------------------------------------------
\begin{frame}{Why We Need the Density Matrix ($\rho$)}
  \begin{alertblock}{Handling Incomplete Information}
    \begin{itemize}
      \item When we look at only one part of an entangled pair, we have
        incomplete information about it.
      \pause
      \item The state of such a subsystem cannot be described by a state
        vector.
      \pause
      \item The density matrix formalism is essential for describing these
        subsystems.
    \end{itemize}
  \end{alertblock}
\end{frame}

%------------------------------------------------------------------------------
\begin{frame}{The Density Matrix: Pure vs. Mixed States}
  \begin{block}{Pure States}
    For a pure state $\ket{\Psi}$, the density matrix is a projection operator:
    \begin{equation}
      \rho_{\text{pure}} = \ket{\Psi}\bra{\Psi}
    \end{equation}
    It is a projector ($\rho^2 = \rho$), so purity can be tested:
    \[
      \alert{\Tr(\rho^2) = 1}
    \]
  \end{block}
  \pause
  \vspace{-1em}
  \begin{block}{Mixed States}
    For a statistical ensemble of pure states $\{p_i, \ket{\psi_i}\}$:
    \begin{equation}
      \rho_{\text{mixed}} = \sum_i p_i \ket{\psi_i}\bra{\psi_i}
    \end{equation}
    For any mixed state, the purity is less than one:
    $\alert{\Tr(\rho_{\text{mixed}}^2) < 1}$.
  \end{block}
\end{frame}

%------------------------------------------------------------------------------
\begin{frame}{Quantifying Uncertainty: Von Neumann Entropy}
  \begin{block}{Definition}
    The degree of uncertainty or ``mixedness'' is quantified by the Von
    Neumann entropy:
    \begin{equation}
      S(\rho) = - \Tr(\rho \ln \rho) = - \sum_i \lambda_i \ln \lambda_i
    \end{equation}
    where $\lambda_i$ are the eigenvalues of $\rho$.
    \begin{itemize}[<+->]
      \item Pure state ($\rho = \ket{\psi}\bra{\psi}$): $S(\rho) = 0$
        (maximal knowledge).
      \item Maximally mixed state ($\rho=I/d$): $S(\rho) = \ln d$
        (minimal knowledge).
    \end{itemize}
  \end{block}
\end{frame}

%------------------------------------------------------------------------------
\begin{frame}{Entropy as a Signature of Entanglement}
  \begin{alertblock}{The Entropy of Entanglement}
    Consider a composite system $AB$ in a pure state $\ket{\Psi}_{AB}$, so
    $S(\rho_{AB})=0$.
    \pause

    The state of subsystem A is found via the \emph{reduced density matrix}:
    \[ \rho_A = \Tr_B(\rho_{AB}) \]
    \pause
    If $\ket{\Psi}_{AB}$ is entangled, $\rho_A$ will be a mixed state, and its
    entropy \alert{$S(\rho_A) > 0$}.
    \pause

    This quantity, $S(\rho_A)$, is the \emph{entropy of entanglement}.
  \end{alertblock}
\end{frame}

%==============================================================================
\section{Quantifying and Detecting Entanglement}
%==============================================================================

%------------------------------------------------------------------------------
\begin{frame}{Pure States: Schmidt Decomposition}
  \begin{block}{The Schmidt Decomposition Theorem}
    Any pure bipartite state $\ket{\Psi}_{AB}$ can be written in a special
    orthonormal basis:
    \begin{equation}
      \ket{\Psi}_{AB} = \sum_{k=1}^{r_S} \sqrt{\lambda_k} \ket{k}_A \otimes \ket{k}_B
    \end{equation}
    \begin{itemize}
      \item $\{\ket{k}_A\}$ and $\{\ket{k}_B\}$ are orthonormal bases (the
        Schmidt bases).
      \item The number of non-zero terms, $r_S$, is the \emph{Schmidt rank}.
    \end{itemize}
  \end{block}
\end{frame}

%------------------------------------------------------------------------------
\begin{frame}{Schmidt Rank and Entanglement}
  \begin{alertblock}{A Simple Criterion}
    A state is \emph{separable if and only if its Schmidt rank is 1}.
    \pause

    For pure states, the degree of entanglement is uniquely quantified by the
    \emph{entropy of entanglement}:
    \begin{equation}
      E(\ket{\Psi}) = S(\rho_A) = -\sum_{k=1}^{r_S} \lambda_k \ln \lambda_k
    \end{equation}
    (where $\lambda_k$ are the squares of the Schmidt coefficients).
  \end{alertblock}
\end{frame}

%------------------------------------------------------------------------------
\begin{frame}{Mixed States: A Zoo of Measures}
  Quantifying mixed-state entanglement is complex; no single measure exists.
  \pause
  \begin{block}{Entanglement of Formation (\texorpdfstring{$E_F$}{E\_F})}
    \textit{Question:} What is the minimum average pure-state entanglement
    needed to create $\rho$?
    \[ E_F(\rho) = \min_{\{p_i, \ket{\psi_i}\}} \sum_i p_i E(\ket{\psi_i}) \]
  \end{block}
  \pause
  \begin{block}{Relative Entropy of Entanglement (\texorpdfstring{$E_R$}{E\_R})}
    \textit{Question:} How ``distant'' is $\rho$ from the set of separable states
    (SEP)?
    \[ E_R(\rho) = \min_{\sigma \in \text{SEP}} S(\rho || \sigma) = \min_{\sigma
    \in \text{SEP}} \Tr(\rho \ln \rho - \rho \ln \sigma) \]
  \end{block}
\end{frame}

%------------------------------------------------------------------------------
\begin{frame}{Detecting Mixed-State Entanglement: The PPT Criterion}
  \begin{block}{The Positive Partial Transpose (PPT) Test}
    A simple but powerful necessary condition for separability.
    \begin{enumerate}
      \item<1-> Start with a state $\rho_{AB}$.
      \pause
      \item<2-> Compute the \emph{partial transpose} on one subsystem, e.g., B:
        $\rho^{T_B}$.
      \pause
      \item<3-> Check if $\rho^{T_B}$ is positive semidefinite (all
        eigenvalues $\ge 0$).
    \end{enumerate}
  \end{block}
  \pause
  \begin{alertblock}{The Punchline}
    \begin{itemize}
        \item If $\rho_{AB}$ is separable $\implies \rho^{T_B}$ is positive.
        \pause
        \item Therefore, if $\rho^{T_B}$ has any \alert{negative eigenvalues},
        the state $\rho_{AB}$ is certified as \textbf{entangled}.
    \end{itemize}
    \footnotesize{(This condition is also sufficient only for $2\times2$ and
    $2\times3$ systems.)}
  \end{alertblock}
\end{frame}

%------------------------------------------------------------------------------
\begin{frame}{Worked Example: The Werner State}
  \begin{block}{Definition}
    A mixture of a Bell state and a maximally mixed state ($p \in [0, 1]$):
    \[ \rho_W = p \ket{\Psi^-}\bra{\Psi^-} + \frac{1-p}{4} \mathbb{I}_4 \]
    where $\ket{\Psi^-} = \frac{1}{\sqrt{2}}(\ket{01}-\ket{10})$.
    \pause

    In the computational basis $\{\ket{00}, \ket{01}, \ket{10}, \ket{11}\}$:
    \[
      \rho_W = \frac{1}{4}
      \begin{pmatrix}
        1-p & 0 & 0 & 0 \\
        0 & 1+p & -2p & 0 \\
        0 & -2p & 1+p & 0 \\
        0 & 0 & 0 & 1-p
      \end{pmatrix}
    \]
  \end{block}
\end{frame}

%------------------------------------------------------------------------------
\begin{frame}{Worked Example: Applying the PPT Criterion}
  \begin{block}{The Partial Transpose}
    We apply the transpose operation only to the second qubit's subspace.
    This swaps the $\ket{01}\bra{10}$ and $\ket{10}\bra{01}$ matrix elements.
    \[
      \rho_W^{T_B} = \frac{1}{4}
      \begin{pmatrix}
        1-p & 0 & 0 & \alert{-2p} \\
        0 & 1+p & 0 & 0 \\
        0 & 0 & 1+p & 0 \\
        \alert{-2p} & 0 & 0 & 1-p
      \end{pmatrix}
    \]
  \end{block}
\end{frame}

%------------------------------------------------------------------------------
\begin{frame}{Worked Example: Werner State (Conclusion)}
  \begin{block}{Eigenvalues of the Partial Transpose}
    The eigenvalues of $\rho_W^{T_B}$ are:
    \begin{equation}
      \lambda_{1,2,3} = \frac{1+p}{4} \quad \lambda_4 = \frac{1-3p}{4}
    \end{equation}
  \end{block}
  \vspace{-1em}
  \pause
  \begin{alertblock}{Result}
    The eigenvalue $\lambda_4$ becomes negative if $1-3p < 0$:
    \[ \alert{p > 1/3} \]
    Therefore, the Werner state is \textbf{entangled for $p > 1/3$}.
    \pause

    The magnitude of the negative eigenvalue is a measure of entanglement
    (a ``negativity''):
    \( \mathcal{N}(\rho_W) = \max\left(0, -\lambda_4\right) \)
  \end{alertblock}
\end{frame}

%------------------------------------------------------------------------------
\begin{frame}{Detecting Entanglement: Witnesses}
  \begin{block}{Entanglement Witness}
    An entanglement witness is a Hermitian operator $W$ designed such that:
    \begin{itemize}
      \item $\Tr(W \rho_{\text{sep}}) \ge 0$ for \emph{all} separable states.
      \pause
      \item There exists at least one entangled state $\rho_{\text{ent}}$ with
        $\Tr(W \rho_{\text{ent}}) < 0$.
    \end{itemize}
  \end{block}
  \pause
  \begin{alertblock}{Experimental Implication}
    If an experiment measures an expectation value $\braket{W} = \Tr(W \rho) <
    0$, the state $\rho$ is \textbf{certified as entangled}.
    \pause
    \begin{center}
    Geometrically, $W$ is a hyperplane separating $\rho_{\text{ent}}$ from
    the convex set of separable states.
    \end{center}
  \end{alertblock}
\end{frame}

%==============================================================================
\section{Dynamics of Entanglement}
%==============================================================================

%------------------------------------------------------------------------------
\begin{frame}{Entanglement in Motion}
  \begin{block}{Why Study Entanglement Dynamics?}
    We must understand how entanglement evolves under the influence of:
    \begin{itemize}
      \item The system's Hamiltonian (internal dynamics).
      \item Interaction with an environment (open system dynamics).
    \end{itemize}
    \pause
    This is crucial for quantum information, where we must \textbf{generate},
    \textbf{manipulate}, and \textbf{protect} entangled states.
  \end{block}
  \pause
  \begin{alertblock}{Propagation in Closed Systems}
    In many-body systems with local interactions, entanglement propagates.
    The \alert{Lieb-Robinson bounds} establish a finite maximum speed for
    information, creating a linear ``light cone'' for correlations.
  \end{alertblock}
\end{frame}

%------------------------------------------------------------------------------
\begin{frame}{Non-Equilibrium Dynamics: Quantum Quenches}
  \begin{block}{Quantum Quench}
    A sudden change in a system's Hamiltonian ($H_0 \to H_1$) drives the
    system out of equilibrium, revealing universal entanglement dynamics.
  \end{block}
  \pause
  \begin{alertblock}{The Quasiparticle Picture}
    For a subsystem of length $\ell$, the entropy $S_A(t)$ shows a universal
    pattern:
    \begin{enumerate}
      \item Initial \alert{linear growth}: $S_A(t) \propto t$.
      \item \alert{Saturation} to a volume-law: $S_A(t) \to \text{const} \cdot \ell$.
    \end{enumerate}
    \pause
    \textit{Explanation:} The quench creates entangled quasiparticle pairs.
    Entanglement grows as pairs are split by the subsystem boundary.
  \end{alertblock}
\end{frame}

%------------------------------------------------------------------------------
\begin{frame}{Entanglement in Open Systems: Fragility}
  \begin{block}{Decoherence}
    Interaction with an external environment degrades a system's coherence
    and entanglement over time.
    \pause

    Entanglement is typically far more fragile than the coherence of
    individual subsystems.
  \end{block}
  \pause
  \begin{alertblock}{Entanglement Sudden Death (ESD)}
    A striking feature of entanglement decay.
    \begin{itemize}
      \item Unlike local coherence (which decays asymptotically), entanglement
        can \textbf{vanish completely at a finite time}.
      \pause
      \item After this time, the global state is separable, even if
        subsystems remain coherent.
    \end{itemize}
  \end{alertblock}
\end{frame}

%==============================================================================
\section{Alternative Perspectives}
%==============================================================================

%------------------------------------------------------------------------------
\begin{frame}{Beyond Hilbert Space: Geometric Algebra}
  \begin{block}{An Alternative View}
    Alternative mathematical frameworks can offer different physical insights
    beyond the standard Hilbert space formalism.
  \end{block}
  \pause
  \begin{alertblock}{Entanglement in Geometric Algebra}
    \begin{itemize}
      \item Entanglement is not an intrinsic property of a state, but a
        relationship between reference frames.
      \pause
      \item It is described by a superposition of \emph{relative rotations}.
      \pause
      \item The focus shifts from abstract states to the transformation
        operators themselves.
    \end{itemize}
  \end{alertblock}
\end{frame}

%------------------------------------------------------------------------------
\begin{frame}{The Entanglor Operator}
  \begin{block}{The Algebra of Physical Space (APS)}
    \begin{itemize}
      \item The language is Geometric Algebra, a powerful tool for describing
        geometric relationships.
      \item In this framework, spin states are represented as \emph{rotors}
        (operators that perform rotations).
    \end{itemize}
  \end{block}
  \pause
  \begin{alertblock}{Entanglement as a Transformation}
    \textbf{Central Thesis:} Entanglement is encoded within a transformation
    operator, the \alert{entanglor}, not in the state itself.
    \begin{itemize}
      \item Entangled State = (Entanglor) acting on a (Separable State).
      \item The entanglor contains the geometric info of the superposition of
        relative rotations between particle reference frames.
    \end{itemize}
  \end{alertblock}
\end{frame}

%------------------------------------------------------------------------------
\begin{frame}{Relativistic Context and Conceptual Implications}
  \begin{block}{Entangling Eigenspinors}
    A key advantage of APS is its natural compatibility with special
    relativity.
    \pause
    A Lorentz boost (relative velocity) can be combined with an entanglor
    (relative rotation) into a single composite operator: the
    \alert{entangling eigenspinor} $\Lambda_{AB}$.
  \end{block}
  \pause
  \begin{alertblock}{Shift in Perspective}
    This reinterpretation has profound conceptual implications:
    \begin{itemize}
      \item The ``weirdness'' of QM is shifted from the state to the
        \emph{transformation process}.
      \pause
      \item Wave function collapse is not a physical change, but the
        \textbf{revelation of the specific geometric transformation} that
        was present all along.
    \end{itemize}
  \end{alertblock}
\end{frame}
