\section{Mathematical Framework for Entanglement}

%------------------------------------------------------------------------------

\begin{frame}{Motivation: From Feature to Formalism}
  \begin{block}{Recap of the Origin of Entanglement}
    Quantum entanglement first emerged as a mathematical feature of early
    quantum mechanics.
    \pause

    Subsequent experimental confirmations and theoretical advancements have
    enabled a more rigorous description.
    \pause

    The core of this description lies in how we represent the state of a
    \emph{composite quantum system}.
  \end{block}
  \pause

  \begin{alertblock}{A Note on Interpretation}
    While the mathematical framework is well-established, its physical
    interpretation remains a subject of active debate---a common theme in
    quantum theory.
  \end{alertblock}
\end{frame}

%------------------------------------------------------------------------------

\begin{frame}{Departure from Classical State Spaces}
  \begin{block}{Classical Composite Systems}
    The state of a composite system is built from its parts. For a two-particle
    system, the total state is a point $(x_A, x_B)$ in the \alert{Cartesian
    product} of individual phase spaces:
    \[
      \Gamma_{AB} = \Gamma_A \times \Gamma_B
    \]
    Specifying the state of each subsystem completely specifies the state of the
    whole.
  \end{block}
  \pause
  \begin{alertblock}{A Naive Quantum Extrapolation}
    One might guess the quantum state space is $\mathcal{H}_{AB} = \mathcal{H}_A
    \times \mathcal{H}_B$.
    \pause

    \textbf{This is incorrect.} It fails to incorporate the \emph{superposition
    principle} for combined states. An ordered pair $(\ket{\psi_A},
    \ket{\phi_B})$ doesn't accommodate linear combinations.
  \end{alertblock}
\end{frame}

%------------------------------------------------------------------------------

\begin{frame}{The Tensor Product Structure}
  \begin{block}{The Correct Postulate}
    The state space for a composite quantum system is constructed using the
    \emph{tensor product} of the individual Hilbert spaces:
    \begin{equation}
      \mathcal{H}_{AB} = \mathcal{H}_A \otimes \mathcal{H}_B
    \end{equation}
  \end{block}
  \pause
  \vspace{-1em}
  \begin{block}{General Pure State}
    A general pure state $\ket{\Psi}$ in this composite space is a superposition of basis states:
    \begin{equation}
      \ket{\Psi} = \sum_{i,j} c_{ij} (\ket{a_i} \otimes \ket{b_j})
    \end{equation}
    where $\{\ket{a_i}\}$ and $\{\ket{b_j}\}$ are orthonormal bases for
    $\mathcal{H}_A$ and $\mathcal{H}_B$.
    \pause

    The dimension of the new space is the product of the constituent dimensions:
    \(
      \dim(\mathcal{H}_{AB}) = \dim(\mathcal{H}_A) \cdot \dim(\mathcal{H}_B)
    \)
  \end{block}
\end{frame}

%------------------------------------------------------------------------------

\begin{frame}{Two Fundamental Classes of States}
  \begin{columns}[T]
    \begin{column}{0.5\textwidth}
      \begin{block}{Separable (Product) States}
        A state is \emph{separable} if it can be written as a single tensor product:
        \begin{equation}
          \ket{\Psi}_{\text{sep}} = \ket{\psi_A} \otimes \ket{\psi_B}
        \end{equation}
        Subsystems have definite, independent properties, analogous to the classical case.
      \end{block}
    \end{column}
    \pause
    \begin{column}{0.5\textwidth}
      \begin{alertblock}{Entangled States}
        A state $\ket{\Psi} \in \mathcal{H}_{AB}$ that \emph{cannot} be written
        in the separable form is called \emph{entangled}.

        These states represent a uniquely quantum-mechanical correlation.
        Measurement on subsystem A instantaneously influences subsystem B,
        regardless of distance.
      \end{alertblock}
    \end{column}
  \end{columns}
\end{frame}

%------------------------------------------------------------------------------

\begin{frame}{Formalism: Operators and the Density Matrix}
  \begin{block}{Operators and Inner Product}
    \begin{itemize}
      \item An operator $O_A$ on subsystem $A$ is represented on
        $\mathcal{H}_{AB}$ as a \alert{local operator}: $O_A \otimes I_B$.
        \pause
      \item The inner product is defined by linear extension:
        \[
          (\bra{\phi_A} \otimes \bra{\phi_B}) (\ket{\psi_A} \otimes
          \ket{\psi_B}) = \braket{\phi_A | \psi_A} \braket{\phi_B | \psi_B}
        \]
    \end{itemize}
  \end{block}
  \pause
  \begin{alertblock}{Why We Need the Density Matrix ($\rho$)}
    Entanglement forces us to consider situations of incomplete information,
    especially when looking at only one part of an entangled pair.

    The density matrix formalism is essential for this.
  \end{alertblock}
\end{frame}

%------------------------------------------------------------------------------

\begin{frame}{The Density Matrix: Pure vs. Mixed States}
  \begin{block}{Pure States}
    For a pure state $\ket{\Psi}$, the density matrix is a projection operator:
    \begin{equation}
      \rho_{\text{pure}} = \ket{\Psi}\bra{\Psi}
    \end{equation}
    It is a projector ($\rho^2 = \rho$), leading to a simple test for purity:
    \[
      \alert{\Tr(\rho^2) = 1}
    \]
  \end{block}
  \pause
  \vspace{-1em}
  \begin{block}{Mixed States}
    For a statistical ensemble of pure states $\{\ket{\psi_i}\}$ with
    probabilities $\{p_i\}$, the state is mixed:
    \begin{equation}
      \rho_{\text{mixed}} = \sum_i p_i \ket{\psi_i}\bra{\psi_i}
    \end{equation}
    For any mixed state, the purity is less than one:
    $\alert{\Tr(\rho_{\text{mixed}}^2) < 1}$.
  \end{block}
\end{frame}

%------------------------------------------------------------------------------

\begin{frame}{Quantifying Uncertainty: Von Neumann Entropy}
  \begin{block}{Definition}
    The degree of uncertainty or ``mixedness'' is quantified by the Von Neumann
    entropy:
    \begin{equation}
      S(\rho) = - \Tr(\rho \ln \rho) = - \sum_i \lambda_i \ln \lambda_i
    \end{equation}
    where $\lambda_i$ are the eigenvalues of $\rho$.
    \begin{itemize}[<+->]
      \item For a pure state, $S(\rho) = 0$ (maximal knowledge).
      \item For a maximally mixed state ($\rho=I/d$), $S(\rho) = \ln d$ (minimal knowledge).
    \end{itemize}
  \end{block}
\end{frame}

\begin{frame}
  \begin{alertblock}{Entropy as a Signature of Entanglement}
    Consider a composite system $AB$ in a pure state $\ket{\Psi}_{AB}$, so
    $S(\rho_{AB})=0$.

    \pause

    The state of subsystem A is the \emph{reduced density matrix}:
    \[ \rho_A = \Tr_B(\rho_{AB}) \]
    If $\ket{\Psi}_{AB}$ is entangled, $\rho_A$ will be a mixed state, and its
    entropy \alert{$S(\rho_A) > 0$}.

    \pause

    This is the \emph{entropy of entanglement}.
  \end{alertblock}
\end{frame}

%------------------------------------------------------------------------------

\section{Quantifying and Detecting Entanglement}

%------------------------------------------------------------------------------

\begin{frame}{Pure States: Schmidt Decomposition}
  \begin{block}{The Schmidt Decomposition Theorem}
    Any pure bipartite state $\ket{\Psi}_{AB}$ can be written in a canonical form:
    \begin{equation}
      \ket{\Psi}_{AB} = \sum_{k=1}^{r_S} \sqrt{\lambda_k} \ket{k}_A \otimes \ket{k}_B
    \end{equation}
    \begin{itemize}
      \item $\{\ket{k}_A\}$ and $\{\ket{k}_B\}$ are orthonormal bases.
      \item The number of terms, $r_S$, is the \emph{Schmidt rank}.
    \end{itemize}
  \end{block}
\end{frame}

\begin{frame}
  \begin{alertblock}{Schmidt Rank and Entanglement}
    A state is \emph{separable if and only if its Schmidt rank is 1}.

    The degree of entanglement is uniquely quantified by the \emph{entropy of entanglement}:
    \begin{equation}
      E(\ket{\Psi}) = S(\rho_A) = -\sum_{k=1}^{r_S} \lambda_k \ln \lambda_k
    \end{equation}
  \end{alertblock}
\end{frame}

%------------------------------------------------------------------------------

\begin{frame}{Mixed States: A Zoo of Measures}
  Quantifying entanglement in mixed states is far more complex as they lack a
  unique decomposition. Several measures exist.
  \pause
  \begin{block}{Entanglement of Formation (\texorpdfstring{$E_F$}{E\_F})}
    What is the minimum average pure-state entanglement required to create $\rho$?
    \[ E_F(\rho) = \min_{\{p_i, \ket{\psi_i}\}} \sum_i p_i E(\ket{\psi_i}) \]
  \end{block}
  \pause
  \begin{block}{Relative Entropy of Entanglement (\texorpdfstring{$E_R$}{E\_R})}
    A geometric approach: how "distant" is $\rho$ from the set of all separable
    states (SEP)?
    \[ E_R(\rho) = \min_{\sigma \in \text{SEP}} S(\rho || \sigma) = \min_{\sigma
    \in \text{SEP}} \Tr(\rho \ln \rho - \rho \ln \sigma) \]
  \end{block}
\end{frame}

%------------------------------------------------------------------------------

\begin{frame}{Detecting Mixed-State Entanglement: The PPT Criterion}
  \begin{block}{The Positive Partial Transpose (PPT) Criterion}
    A simple but powerful test based on a necessary condition for separability.
    \begin{enumerate}
      \item<1-> Start with a state $\rho_{AB}$.
        \pause
      \item<2-> Compute the \emph{partial transpose} with respect to one
        subsystem, e.g., B: $\rho^{T_B}$.
        \pause
      \item<3-> Check if $\rho^{T_B}$ is a valid density matrix (i.e., it is
        positive semidefinite, meaning all its eigenvalues are non-negative).
    \end{enumerate}
  \end{block}
  \pause
  \begin{alertblock}{Conclusion}
    If $\rho_{AB}$ is separable, then $\rho^{T_B}$ \emph{must be} positive semidefinite.

    Therefore, if $\rho^{T_B}$ has any \emph{negative eigenvalues}, the state
    $\rho_{AB}$ is certified as \emph{entangled}.

    \footnotesize{(This condition is necessary and sufficient only for
    $2\times2$ and $2\times3$ systems.)}
  \end{alertblock}
\end{frame}

%------------------------------------------------------------------------------

\begin{frame}{Worked Example: The Werner State}
  \begin{block}{Definition}
    A mixture of a maximally entangled Bell state and a maximally mixed state ($p \in [0, 1]$):
    \[ \rho_W = p \ket{\Psi^-}\bra{\Psi^-} + \frac{1-p}{4} \mathbb{I}_4 \]
    where $\ket{\Psi^-} = \frac{1}{\sqrt{2}}(\ket{01}-\ket{10})$. In the computational basis:
    \[
      \rho_W = \frac{1}{4}
      \begin{pmatrix}
        1-p & 0 & 0 & 0 \\
        0 & 1+p & -2p & 0 \\
        0 & -2p & 1+p & 0 \\
        0 & 0 & 0 & 1-p
      \end{pmatrix}
    \]
  \end{block}
\end{frame}

\begin{frame}
  \begin{block}{Applying the PPT Criterion}
    The partial transpose with respect to the second qubit is:
    \[
      \rho_W^{T_B} = \frac{1}{4}
      \begin{pmatrix}
        1-p & 0 & 0 & \alert{-2p} \\
        0 & 1+p & 0 & 0 \\
        0 & 0 & 1+p & 0 \\
        \alert{-2p} & 0 & 0 & 1-p
      \end{pmatrix}
    \]
  \end{block}
\end{frame}

%------------------------------------------------------------------------------

\begin{frame}{Worked Example: The Werner State (Conclusion)}
  \begin{block}{Eigenvalues of the Partial Transpose}
    The eigenvalues of $\rho_W^{T_B}$ are found to be:
    \begin{align*}
      \lambda_{1,2,3} &= \frac{1+p}{4} \quad (\text{always } \ge 0) \\
      \lambda_4 &= \frac{1-3p}{4}
    \end{align*}
  \end{block}
  \vspace{-2em}
  \pause
  \begin{alertblock}{Result}
    The fourth eigenvalue $\lambda_4$ becomes negative if $1-3p < 0$, which means:
    \( p > 1/3 \)
    Therefore, the Werner state is \emph{entangled for $p > 1/3$}.
  \pause

    A measure of entanglement can be defined from the negative eigenvalue:
    \(
      \mathcal{N}(\rho_W) = \max\left(0, -\lambda_4\right) = \max\left(0,
      \frac{3p-1}{4}\right)
    \)
  \end{alertblock}
\end{frame}

%------------------------------------------------------------------------------

\begin{frame}{Detecting Entanglement: Witnesses}
  \begin{block}{Entanglement Witness}
    An \emph{entanglement witness} is a Hermitian operator $W$ that acts as a
    detector. It is constructed to satisfy:
    \begin{itemize}
      \item $\Tr(W \rho_{\text{sep}}) \ge 0$ for \emph{all} separable states
        $\rho_{\text{sep}}$.
        \pause
      \item There exists at least one entangled state $\rho_{\text{ent}}$ such
        that $\Tr(W \rho_{\text{ent}}) < 0$.
    \end{itemize}
  \end{block}
  \pause
  \begin{alertblock}{Experimental Implication}
    If an experiment measures an expectation value $\braket{W} = \Tr(W \rho) <
    0$, the state $\rho$ is \emph{certified as entangled}.
    \vspace{1em}
    Geometrically, $W$ defines a hyperplane that separates a specific entangled
    state from the convex set of separable states.
  \end{alertblock}
\end{frame}

%------------------------------------------------------------------------------

\section{Alternative Perspectives}

%------------------------------------------------------------------------------

\begin{frame}{Beyond Hilbert Space: Geometric Algebra}
  \begin{block}{An Alternative View}
    While the Hilbert space formalism is the standard, alternative mathematical
    frameworks can offer different physical insights.
  \end{block}
  \pause
  \begin{alertblock}{Entanglement in Geometric Algebra}
    \begin{itemize}
      \item This approach interprets entanglement not as an intrinsic property
        of a state, but as a relationship between reference frames.
        \pause
      \item It is described by a superposition of \emph{relative rotations}.
        \pause
      \item This shifts the focus from analysing abstract states to analysing
        the fundamental transformation operators themselves.
    \end{itemize}
  \end{alertblock}
\end{frame}

%------------------------------------------------------------------------------
