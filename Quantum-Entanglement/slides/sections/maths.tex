\section{Mathematical Framework for Entanglement}

%------------------------------------------------------------------------------

\begin{frame}{Motivation: From Feature to Formalism}
  \begin{block}{Recap of the Origin of Entanglement}
    Quantum entanglement first emerged as a mathematical feature of early
    quantum mechanics.
    \pause

    Subsequent experimental confirmations and theoretical advancements have
    enabled a more rigorous description.
    \pause

    The core of this description lies in how we represent the state of a
    \emph{composite quantum system}.
  \end{block}
  \pause

  \begin{alertblock}{A Note on Interpretation}
    While the mathematical framework is well-established, its physical
    interpretation remains a subject of active debate---a common theme in
    quantum theory.
  \end{alertblock}
\end{frame}

%------------------------------------------------------------------------------

\begin{frame}{Departure from Classical State Spaces}
  \begin{block}{Classical Composite Systems}
    The state of a composite system is built from its parts. For a two-particle
    system, the total state is a point $(x_A, x_B)$ in the \alert{Cartesian
    product} of individual phase spaces:
    \[
      \Gamma_{AB} = \Gamma_A \times \Gamma_B
    \]
    Specifying the state of each subsystem completely specifies the state of the
    whole.
  \end{block}
  \pause
  \begin{alertblock}{A Naive Quantum Extrapolation}
    One might guess the quantum state space is $\mathcal{H}_{AB} = \mathcal{H}_A
    \times \mathcal{H}_B$.
    \pause

    \textbf{This is incorrect.} It fails to incorporate the \emph{superposition
    principle} for combined states. An ordered pair $(\ket{\psi_A},
    \ket{\phi_B})$ doesn't accommodate linear combinations.
  \end{alertblock}
\end{frame}

%------------------------------------------------------------------------------

\begin{frame}{The Tensor Product Structure}
  \begin{block}{The Correct Postulate}
    The state space for a composite quantum system is constructed using the
    \emph{tensor product} of the individual Hilbert spaces:
    \begin{equation}
      \mathcal{H}_{AB} = \mathcal{H}_A \otimes \mathcal{H}_B
    \end{equation}
  \end{block}
  \pause
  \vspace{-1em}
  \begin{block}{General Pure State}
    A general pure state $\ket{\Psi}$ in this composite space is a superposition of basis states:
    \begin{equation}
      \ket{\Psi} = \sum_{i,j} c_{ij} (\ket{a_i} \otimes \ket{b_j})
    \end{equation}
    where $\{\ket{a_i}\}$ and $\{\ket{b_j}\}$ are orthonormal bases for
    $\mathcal{H}_A$ and $\mathcal{H}_B$.
    \pause

    The dimension of the new space is the product of the constituent dimensions:
    \(
      \dim(\mathcal{H}_{AB}) = \dim(\mathcal{H}_A) \cdot \dim(\mathcal{H}_B)
    \)
  \end{block}
\end{frame}

%------------------------------------------------------------------------------

\begin{frame}{Two Fundamental Classes of States}
  \begin{columns}[T]
    \begin{column}{0.5\textwidth}
      \begin{block}{Separable (Product) States}
        A state is \emph{separable} if it can be written as a single tensor product:
        \begin{equation}
          \ket{\Psi}_{\text{sep}} = \ket{\psi_A} \otimes \ket{\psi_B}
        \end{equation}
        Subsystems have definite, independent properties, analogous to the classical case.
      \end{block}
    \end{column}
    \pause
    \begin{column}{0.5\textwidth}
      \begin{alertblock}{Entangled States}
        A state $\ket{\Psi} \in \mathcal{H}_{AB}$ that \emph{cannot} be written
        in the separable form is called \emph{entangled}.

        These states represent a uniquely quantum-mechanical correlation.
        Measurement on subsystem A instantaneously influences subsystem B,
        regardless of distance.
      \end{alertblock}
    \end{column}
  \end{columns}
\end{frame}

%------------------------------------------------------------------------------

\begin{frame}{A Fundamental Complication: Identical Particles}
  \begin{block}{The Symmetrization Postulate}
    The tensor product formalism $\mathcal{H}_{AB} = \mathcal{H}_A \otimes
    \mathcal{H}_B$ implicitly assumes particles are distinguishable.
    \pause

    However, for a system of identical particles, quantum mechanics imposes the
    \alert{symmetrization postulate}:
    \begin{itemize}[<+->]
      \item The total wave function must be \textbf{symmetric} under particle exchange for \textbf{bosons}.
      \item The total wave function must be \textbf{antisymmetric} under particle exchange for \textbf{fermions}.
    \end{itemize}
  \end{block}
\end{frame}

\begin{frame}
  \begin{alertblock}{Consequence for Fermions}
    The physically allowed states do not live in the full tensor product space
    $\mathcal{H} \otimes \mathcal{H}$, but in its \emph{antisymmetric subspace}
    $\mathcal{H} \wedge \mathcal{H}$.

    The standard definition of separability, which relies on particle labels,
    becomes inapplicable.
  \end{alertblock}
\end{frame}

%------------------------------------------------------------------------------

\begin{frame}{Redefining Entanglement for Fermions}
  \begin{block}{Mode Entanglement and Slater Rank}
    For identical particles, entanglement is reformulated based on correlations
    between different degrees of freedom (e.g., spatial and spin) or between
    modes.
    \pause
    \begin{itemize}
      \item The role of a separable state is now played by a single
        \alert{Slater determinant}. For two fermions in single-particle states
        $\ket{\phi_1}$ and $\ket{\phi_2}$:
        \[
          \ket{\Psi}_{\text{Slater}} = \frac{1}{\sqrt{2}}
          \left( \ket{\phi_1}_1 \otimes \ket{\phi_2}_2 -
          \ket{\phi_2}_1 \otimes \ket{\phi_1}_2 \right)
        \]
        \pause
      \item A pure fermionic state is considered \alert{entangled if and only if
        its Slater rank is greater than one} (i.e., it requires a superposition
        of multiple Slater determinants to be described).
    \end{itemize}
  \end{block}
\end{frame}

%------------------------------------------------------------------------------

\begin{frame}{Example: Pauli Exclusion \emph{Forces} Entanglement}
  \begin{block}{Two Electrons in the Same Spatial Orbital}
    Consider two electrons (spin-1/2 fermions) occupying the same spatial
    orbital. The total wave function $\ket{\Psi}_{\text{total}} =
    \ket{\psi}_{\text{spatial}} \otimes \ket{\chi}_{\text{spin}}$ must be
    antisymmetric.
    \begin{enumerate}[<+->]
      \item \textbf{Spatial State:} Since both electrons are in the same
        orbital, their spatial wave function $\ket{\psi}_{\text{spatial}}$ is
        necessarily \emph{symmetric} under particle exchange.
      \item \textbf{Spin State:} To ensure the total wave function is
        antisymmetric, the spin part $\ket{\chi}_{\text{spin}}$ \emph{must be
        antisymmetric}.
      \item \textbf{Result:} The unique antisymmetric spin state for two
        spin-1/2 particles is the maximally entangled \alert{spin-singlet
        state}:
        \[
          \ket{\Psi^-} = \frac{1}{\sqrt{2}} (\ket{\uparrow}_1 \otimes
          \ket{\downarrow}_2 - \ket{\downarrow}_1 \otimes \ket{\uparrow}_2)
        \]
    \end{enumerate}
  \end{block}
\end{frame}

\begin{frame}
  \begin{alertblock}{Conclusion}
    The Pauli exclusion principle \emph{forces} the two electrons into a
    maximally entangled Bell state. For indistinguishable particles,
    entanglement is not just a possibility but can be a necessity imposed by
    fundamental symmetries.
  \end{alertblock}
\end{frame}

%------------------------------------------------------------------------------

\begin{frame}{Formalism: Operators and the Density Matrix}
  \begin{block}{Operators and Inner Product}
    \begin{itemize}
      \item An operator $O_A$ on subsystem $A$ is represented on
        $\mathcal{H}_{AB}$ as a \alert{local operator}: $O_A \otimes I_B$.
        \pause
      \item The inner product is defined by linear extension:
        \[
          (\bra{\phi_A} \otimes \bra{\phi_B}) (\ket{\psi_A} \otimes
          \ket{\psi_B}) = \braket{\phi_A | \psi_A} \braket{\phi_B | \psi_B}
        \]
    \end{itemize}
  \end{block}
  \pause
  \begin{alertblock}{Why We Need the Density Matrix ($\rho$)}
    Entanglement forces us to consider situations of incomplete information,
    especially when looking at only one part of an entangled pair.

    The density matrix formalism is essential for this.
  \end{alertblock}
\end{frame}

%------------------------------------------------------------------------------

\begin{frame}{The Density Matrix: Pure vs. Mixed States}
  \begin{block}{Pure States}
    For a pure state $\ket{\Psi}$, the density matrix is a projection operator:
    \begin{equation}
      \rho_{\text{pure}} = \ket{\Psi}\bra{\Psi}
    \end{equation}
    It is a projector ($\rho^2 = \rho$), leading to a simple test for purity:
    \[
      \alert{\Tr(\rho^2) = 1}
    \]
  \end{block}
  \pause
  \vspace{-1em}
  \begin{block}{Mixed States}
    For a statistical ensemble of pure states $\{\ket{\psi_i}\}$ with
    probabilities $\{p_i\}$, the state is mixed:
    \begin{equation}
      \rho_{\text{mixed}} = \sum_i p_i \ket{\psi_i}\bra{\psi_i}
    \end{equation}
    For any mixed state, the purity is less than one:
    $\alert{\Tr(\rho_{\text{mixed}}^2) < 1}$.
  \end{block}
\end{frame}

%------------------------------------------------------------------------------

\begin{frame}{Quantifying Uncertainty: Von Neumann Entropy}
  \begin{block}{Definition}
    The degree of uncertainty or ``mixedness'' is quantified by the Von Neumann
    entropy:
    \begin{equation}
      S(\rho) = - \Tr(\rho \ln \rho) = - \sum_i \lambda_i \ln \lambda_i
    \end{equation}
    where $\lambda_i$ are the eigenvalues of $\rho$.
    \begin{itemize}[<+->]
      \item For a pure state, $S(\rho) = 0$ (maximal knowledge).
      \item For a maximally mixed state ($\rho=I/d$), $S(\rho) = \ln d$ (minimal knowledge).
    \end{itemize}
  \end{block}
\end{frame}

\begin{frame}
  \begin{alertblock}{Entropy as a Signature of Entanglement}
    Consider a composite system $AB$ in a pure state $\ket{\Psi}_{AB}$, so
    $S(\rho_{AB})=0$.

    \pause

    The state of subsystem A is the \emph{reduced density matrix}:
    \[ \rho_A = \Tr_B(\rho_{AB}) \]
    If $\ket{\Psi}_{AB}$ is entangled, $\rho_A$ will be a mixed state, and its
    entropy \alert{$S(\rho_A) > 0$}.

    \pause

    This is the \emph{entropy of entanglement}.
  \end{alertblock}
\end{frame}

%------------------------------------------------------------------------------

\section{Quantifying and Detecting Entanglement}

%------------------------------------------------------------------------------

\begin{frame}{Pure States: Schmidt Decomposition}
  \begin{block}{The Schmidt Decomposition Theorem}
    Any pure bipartite state $\ket{\Psi}_{AB}$ can be written in a canonical form:
    \begin{equation}
      \ket{\Psi}_{AB} = \sum_{k=1}^{r_S} \sqrt{\lambda_k} \ket{k}_A \otimes \ket{k}_B
    \end{equation}
    \begin{itemize}
      \item $\{\ket{k}_A\}$ and $\{\ket{k}_B\}$ are orthonormal bases.
      \item The number of terms, $r_S$, is the \emph{Schmidt rank}.
    \end{itemize}
  \end{block}
\end{frame}

\begin{frame}
  \begin{alertblock}{Schmidt Rank and Entanglement}
    A state is \emph{separable if and only if its Schmidt rank is 1}.

    The degree of entanglement is uniquely quantified by the \emph{entropy of entanglement}:
    \begin{equation}
      E(\ket{\Psi}) = S(\rho_A) = -\sum_{k=1}^{r_S} \lambda_k \ln \lambda_k
    \end{equation}
  \end{alertblock}
\end{frame}

%------------------------------------------------------------------------------

\begin{frame}{Mixed States: A Zoo of Measures}
  Quantifying entanglement in mixed states is far more complex as they lack a
  unique decomposition. Several measures exist.
  \pause
  \begin{block}{Entanglement of Formation (\texorpdfstring{$E_F$}{E\_F})}
    What is the minimum average pure-state entanglement required to create $\rho$?
    \[ E_F(\rho) = \min_{\{p_i, \ket{\psi_i}\}} \sum_i p_i E(\ket{\psi_i}) \]
  \end{block}
  \pause
  \begin{block}{Relative Entropy of Entanglement (\texorpdfstring{$E_R$}{E\_R})}
    A geometric approach: how ``distant'' is $\rho$ from the set of all separable
    states (SEP)?
    \[ E_R(\rho) = \min_{\sigma \in \text{SEP}} S(\rho || \sigma) = \min_{\sigma
    \in \text{SEP}} \Tr(\rho \ln \rho - \rho \ln \sigma) \]
  \end{block}
\end{frame}

%------------------------------------------------------------------------------

\begin{frame}{Detecting Mixed-State Entanglement: The PPT Criterion}
  \begin{block}{The Positive Partial Transpose (PPT) Criterion}
    A simple but powerful test based on a necessary condition for separability.
    \begin{enumerate}
      \item<1-> Start with a state $\rho_{AB}$.
        \pause
      \item<2-> Compute the \emph{partial transpose} with respect to one
        subsystem, e.g., B: $\rho^{T_B}$.
        \pause
      \item<3-> Check if $\rho^{T_B}$ is a valid density matrix (i.e., it is
        positive semidefinite, meaning all its eigenvalues are non-negative).
    \end{enumerate}
  \end{block}
  \pause
  \begin{alertblock}{Conclusion}
    If $\rho_{AB}$ is separable, then $\rho^{T_B}$ \emph{must be} positive semidefinite.

    Therefore, if $\rho^{T_B}$ has any \emph{negative eigenvalues}, the state
    $\rho_{AB}$ is certified as \emph{entangled}.

    \footnotesize{(This condition is necessary and sufficient only for
    $2\times2$ and $2\times3$ systems.)}
  \end{alertblock}
\end{frame}

%------------------------------------------------------------------------------

\begin{frame}{Worked Example: The Werner State}
  \begin{block}{Definition}
    A mixture of a maximally entangled Bell state and a maximally mixed state ($p \in [0, 1]$):
    \[ \rho_W = p \ket{\Psi^-}\bra{\Psi^-} + \frac{1-p}{4} \mathbb{I}_4 \]
    where $\ket{\Psi^-} = \frac{1}{\sqrt{2}}(\ket{01}-\ket{10})$. In the computational basis:
    \[
      \rho_W = \frac{1}{4}
      \begin{pmatrix}
        1-p & 0 & 0 & 0 \\
        0 & 1+p & -2p & 0 \\
        0 & -2p & 1+p & 0 \\
        0 & 0 & 0 & 1-p
      \end{pmatrix}
    \]
  \end{block}
\end{frame}

\begin{frame}
  \begin{block}{Applying the PPT Criterion}
    The partial transpose with respect to the second qubit is:
    \[
      \rho_W^{T_B} = \frac{1}{4}
      \begin{pmatrix}
        1-p & 0 & 0 & \alert{-2p} \\
        0 & 1+p & 0 & 0 \\
        0 & 0 & 1+p & 0 \\
        \alert{-2p} & 0 & 0 & 1-p
      \end{pmatrix}
    \]
  \end{block}
\end{frame}

%------------------------------------------------------------------------------

\begin{frame}{Worked Example: The Werner State (Conclusion)}
  \begin{block}{Eigenvalues of the Partial Transpose}
    The eigenvalues of $\rho_W^{T_B}$ are found to be:
    \begin{align*}
      \lambda_{1,2,3} &= \frac{1+p}{4} \quad (\text{always } \ge 0) \\
      \lambda_4 &= \frac{1-3p}{4}
    \end{align*}
  \end{block}
  \vspace{-2em}
  \pause
  \begin{alertblock}{Result}
    The fourth eigenvalue $\lambda_4$ becomes negative if $1-3p < 0$, which means:
    \( p > 1/3 \)
    Therefore, the Werner state is \emph{entangled for $p > 1/3$}.
    \pause

    A measure of entanglement can be defined from the negative eigenvalue:
    \(
    \mathcal{N}(\rho_W) = \max\left(0, -\lambda_4\right) = \max\left(0,
    \frac{3p-1}{4}\right)
    \)
  \end{alertblock}
\end{frame}

%------------------------------------------------------------------------------

\begin{frame}{Detecting Entanglement: Witnesses}
  \begin{block}{Entanglement Witness}
    An \emph{entanglement witness} is a Hermitian operator $W$ that acts as a
    detector. It is constructed to satisfy:
    \begin{itemize}
      \item $\Tr(W \rho_{\text{sep}}) \ge 0$ for \emph{all} separable states
        $\rho_{\text{sep}}$.
        \pause
      \item There exists at least one entangled state $\rho_{\text{ent}}$ such
        that $\Tr(W \rho_{\text{ent}}) < 0$.
    \end{itemize}
  \end{block}
  \pause
  \begin{alertblock}{Experimental Implication}
    If an experiment measures an expectation value $\braket{W} = \Tr(W \rho) <
    0$, the state $\rho$ is \emph{certified as entangled}.
    \vspace{1em}
    Geometrically, $W$ defines a hyperplane that separates a specific entangled
    state from the convex set of separable states.
  \end{alertblock}
\end{frame}

%------------------------------------------------------------------------------

\section{Dynamics of Entanglement}

%------------------------------------------------------------------------------

\begin{frame}{Entanglement in Motion}
  \begin{block}{Why Study Entanglement Dynamics?}
    Beyond static characterization, we must understand how entanglement evolves
    in time. This is governed by the system's Hamiltonian and its interaction
    with the environment.
    \pause

    Understanding these dynamics is paramount for quantum information
    processing, where entangled states must be:
    \begin{itemize}
      \item Generated
      \item Manipulated
      \item Protected from decoherence
    \end{itemize}
  \end{block}
  \pause
  \begin{alertblock}{Propagation in Closed Systems}
    In many-body systems with local interactions, entanglement is not static; it
    propagates. The \alert{Lieb-Robinson bounds} establish a finite maximum
    speed for information propagation, implying a linear ``light cone'' for
    correlations.
  \end{alertblock}
\end{frame}

\begin{frame}{Non-Equilibrium Dynamics: Quantum Quenches}
  \begin{block}{Quantum Quench}
    A \emph{quantum quench}—a sudden change in a system's Hamiltonian ($H_0 \to
    H_1$)—drives the system far from equilibrium, revealing universal
    entanglement dynamics.
  \end{block}
  \pause
  \begin{alertblock}{The Quasiparticle Picture}
    For a subsystem of length $\ell$, the von Neumann entropy $S_A(t)$ shows a
    characteristic pattern:
    \begin{enumerate}
      \item An initial \alert{linear growth} in time: $S_A(t) \propto t$.
      \item \alert{Saturation} to a value proportional to the block size:
        $S_A(t) \to \text{const} \cdot \ell$.
    \end{enumerate}
    \pause
    This is explained by a quasiparticle picture: the quench creates entangled
    pairs of quasiparticles that propagate freely.
    A block becomes entangled as one particle enters while its partner does not.
    Saturation occurs when quasiparticles from the block's center can escape, at
    $t^* \approx \ell/(2v)$.
  \end{alertblock}
\end{frame}

\begin{frame}{Entanglement in Open Systems: Fragility}
  \begin{block}{Decoherence}
    When a quantum system interacts with an external environment, its coherence
    and entanglement degrade over time—a process called decoherence.
    \pause

    Entanglement is typically far more fragile than the coherence of individual
    subsystems.
  \end{block}
  \pause
  \begin{alertblock}{Entanglement Sudden Death (ESD)}
    A striking feature of this decay is \alert{Entanglement Sudden Death}.
    \begin{itemize}
      \item Unlike the asymptotic decay of local properties, the entanglement
        between subsystems can \textbf{vanish completely at a finite time}.
      \item After this point, the system evolves in a purely separable state,
        even though individual subsystems may still possess quantum coherence.
    \end{itemize}
    This non-classical decay highlights the unique fragility of quantum
    correlations.
  \end{alertblock}
\end{frame}

%------------------------------------------------------------------------------

\section{Alternative Perspectives}

%------------------------------------------------------------------------------

\begin{frame}{Beyond Hilbert Space: Geometric Algebra}
  \begin{block}{An Alternative View}
    While the Hilbert space formalism is the standard, alternative mathematical
    frameworks can offer different physical insights.
  \end{block}
  \pause
  \begin{alertblock}{Entanglement in Geometric Algebra}
    \begin{itemize}
      \item This approach interprets entanglement not as an intrinsic property
        of a state, but as a relationship between reference frames.
        \pause
      \item It is described by a superposition of \emph{relative rotations}.
        \pause
      \item This shifts the focus from analysing abstract states to analysing
        the fundamental transformation operators themselves.
    \end{itemize}
  \end{alertblock}
\end{frame}

\begin{frame}{The Entanglor Operator}
  \begin{block}{The Algebra of Physical Space (APS)}
    The mathematical language is Geometric Algebra, a powerful tool for
    expressing geometric relationships. In this framework, spin states are
    represented as \emph{rotors} (operators that perform rotations).
  \end{block}
  \pause
  \begin{alertblock}{Entanglement as a Transformation}
    The central thesis: entanglement is encoded within a transformation
    operator, the \alert{entanglor}, not in the state itself.
    \begin{itemize}
      \item An entangled state is the result of an entanglor acting on a
        separable state.
      \item The entanglor contains all the geometric information describing the
        superposition of relative rotations between the reference frames of the
        constituent particles.
    \end{itemize}
  \end{alertblock}
\end{frame}

\begin{frame}{Relativistic Context and Conceptual Implications}
  \begin{block}{Entangling Eigenspinors}
    A key advantage of APS is its natural compatibility with special
    relativity.
    This allows a Lorentz boost (describing relative velocity) to be combined
    with an entanglor (describing relative rotation).
    \pause
    The result is a composite operator, the \alert{entangling eigenspinor}
    $\Lambda_{AB}$, which contains the complete information about both relative
    motion and entanglement between the subsystems' reference frames.
  \end{block}
  \pause
  \begin{alertblock}{Shift in Perspective}
    This reinterpretation has profound conceptual implications:
    \begin{itemize}
      \item The ``weirdness'' of quantum mechanics is shifted from the state to
        the \emph{transformation process}.
      \item Wave function collapse is not a physical change to a spooky object,
        but the \textbf{revelation of which specific geometric transformation}
        described the relationship between the particles all along.
    \end{itemize}
  \end{alertblock}
\end{frame}
