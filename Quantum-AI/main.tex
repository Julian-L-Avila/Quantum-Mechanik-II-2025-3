\documentclass[11pt,a4paper]{article}

% --- PACKAGES ---
\usepackage[utf8]{inputenc}
\usepackage{amsmath, amssymb, amsfonts, amsthm, mathtools} % Consolidated math packages
\usepackage{graphicx}
\usepackage{physics} % Provides \bra, \ket, \dyad, \expval, \qty, etc. (braket package is redundant)
\usepackage[a4paper, margin=1in]{geometry}
\usepackage{hyperref} % For clickable links and better navigation
\hypersetup{colorlinks=true, linkcolor=blue, citecolor=blue, urlcolor=blue}

% --- TITLE AND AUTHOR ---
\title{Construction of a Complete Set of Commuting Observables for a 4D Hilbert Space with Degeneracies}
\author{Student Name}
\date{\today}

\begin{document}

\maketitle

\begin{abstract}
This document details the systematic construction of a Complete Set of Commuting Observables (CSCO) for a quantum system in a four-dimensional Hilbert space, $\mathcal{H}$. Starting from a deliberately constructed degenerate observable $A$, we introduce subsequent commuting observables $B$ and $C$ to sequentially lift the degeneracies. We analyze the process of measurement, including the calculation of probabilities, state collapse, and the algebraic construction of projection operators using both spectral decomposition and the minimal polynomial method.
\end{abstract}

\section{Problem Statement and First Observable \texorpdfstring{$A$}{A}}

We consider a quantum system described by a state vector in a Hilbert space $\mathcal{H}$. The initial information about the system is provided by an observable $A$ which possesses a degenerate spectrum. Our goal is to find a set of additional observables that commute with $A$ and each other, such that their collective set of eigenvalues uniquely specifies any basis vector of $\mathcal{H}$.

\subsection{Constraints and Construction of Operator \texorpdfstring{$A$}{A}}
The primary observable $A$ is a self-adjoint operator, $A: \mathcal{H} \to \mathcal{H}$, subject to the following constraints:
\begin{itemize}
    \item $A$ is diagonal in the computational basis $\{\ket{1}, \ket{2}, \ket{3}, \dots\}$.
    \item The spectrum of $A$, $\sigma(A)$, contains exactly two distinct real eigenvalues, $\alpha \neq \beta$.
    \item One eigenspace, $V_\alpha$, corresponding to eigenvalue $\alpha$, is three-dimensional: $\dim(V_\alpha) = 3$.
    \item The total dimension of the Hilbert space is minimal, but no greater than 4.
\end{itemize}
The spectral theorem dictates that $\mathcal{H}$ is the orthogonal direct sum of the eigenspaces of $A$, so $\dim(\mathcal{H}) = \dim(V_\alpha) + \dim(V_\beta)$. Given $\dim(V_\alpha) = 3$ and the requirement that $\dim(V_\beta) \ge 1$, the constraint $\dim(\mathcal{H}) \le 4$ uniquely determines that $\dim(V_\beta) = 1$ and $\dim(\mathcal{H}) = 4$. Our Hilbert space is thus isomorphic to $\mathbb{C}^4$.

We adopt the standard orthonormal computational basis $\{\ket{1}, \ket{2}, \ket{3}, \ket{4}\}$. Since $A$ must be diagonal in this basis, these basis vectors are its eigenvectors. To satisfy the dimensionality constraints, we assign three of them to the eigenvalue $\alpha$ and one to $\beta$. A canonical matrix representation for $A$ is:
$$
A \mapsto \mathbf{A} =
\begin{pmatrix}
\alpha & 0 & 0 & 0 \\
0 & \alpha & 0 & 0 \\
0 & 0 & \alpha & 0 \\
0 & 0 & 0 & \beta
\end{pmatrix}
$$
The eigenspaces are therefore $V_\alpha = \text{span}\{\ket{1}, \ket{2}, \ket{3}\}$ and $V_\beta = \text{span}\{\ket{4}\}$.

\section{Measurement Formalism for Operator \texorpdfstring{$A$}{A}}

To analyze a measurement, we consider a general state $\ket{\psi} \in \mathcal{H}$ prepared in a basis that is incompatible with the eigenbasis of $A$.

\subsection{A General State in a Superposition Basis}

Let us define a new orthonormal basis, the $V$-basis, denoted by $\{\ket{v_i}\}_{i=1}^4$:
\begin{align*}
    \ket{v_1} &= \frac{1}{\sqrt{2}}(\ket{1} + \ket{2}) & \ket{v_3} &= \frac{1}{\sqrt{2}}(\ket{3} + \ket{4}) \\
    \ket{v_2} &= \frac{1}{\sqrt{2}}(\ket{1} - \ket{2}) & \ket{v_4} &= \frac{1}{\sqrt{2}}(\ket{3} - \ket{4})
\end{align*}
An arbitrary normalized state $\ket{\psi}$ can be expressed as a superposition in this basis:
$$
\ket{\psi} = \sum_{i=1}^{4} c_i \ket{v_i}, \quad \text{with} \quad \sum_{i=1}^{4} |c_i|^2 = 1
$$
To analyze the measurement of $A$, we must express $\ket{\psi}$ in the computational basis (the eigenbasis of $A$). This is a change of basis:
\begin{align*}
\ket{\psi} &= c_1 \qty( \frac{\ket{1} + \ket{2}}{\sqrt{2}} ) + c_2 \qty( \frac{\ket{1} - \ket{2}}{\sqrt{2}} ) + c_3 \qty( \frac{\ket{3} + \ket{4}}{\sqrt{2}} ) + c_4 \qty( \frac{\ket{3} - \ket{4}}{\sqrt{2}} ) \\
&= \frac{1}{\sqrt{2}} \qty[ (c_1+c_2)\ket{1} + (c_1-c_2)\ket{2} + (c_3+c_4)\ket{3} + (c_3-c_4)\ket{4} ]
\end{align*}
The column vector representation of $\ket{\psi}$ in the computational basis is:
$$
\ket{\psi} \mapsto \boldsymbol{\psi}_{A} = \frac{1}{\sqrt{2}}
\begin{pmatrix}
c_1 + c_2 \\
c_1 - c_2 \\
c_3 + c_4 \\
c_3 - c_4
\end{pmatrix}
$$
This transformation is mediated by a unitary matrix $U$ whose columns are the basis vectors $\ket{v_i}$ expressed in the computational basis, such that $\boldsymbol{\psi}_A = U \boldsymbol{\psi}_V$.

\subsection{Projection Operators and Probabilities}

The measurement outcomes are the eigenvalues of $A$. The probability of measuring a particular eigenvalue is found using projection operators. The projectors onto the eigenspaces $V_\alpha$ and $V_\beta$ are:
$$
P_\alpha = \dyad{1}{1} + \dyad{2}{2} + \dyad{3}{3} \quad \text{and} \quad P_\beta = \dyad{4}{4}
$$
The probability $P(\lambda)$ of measuring eigenvalue $\lambda$ for a system in state $\ket{\psi}$ is $P(\lambda) = \norm{P_\lambda \ket{\psi}}^2 = \bra{\psi} P_\lambda \ket{\psi}$.

\paragraph{Probability of measuring $\alpha$:}
The projection of $\ket{\psi}$ onto $V_\alpha$ is $P_\alpha \ket{\psi} = \frac{1}{\sqrt{2}} [ (c_1+c_2)\ket{1} + (c_1-c_2)\ket{2} + (c_3+c_4)\ket{3} ]$.
\begin{align*}
P(\alpha) &= \norm{P_\alpha \ket{\psi}}^2 = \frac{1}{2} \qty( |c_1+c_2|^2 + |c_1-c_2|^2 + |c_3+c_4|^2 ) \\
&= \frac{1}{2} \qty[ 2(|c_1|^2 + |c_2|^2) + |c_3+c_4|^2 ] = |c_1|^2 + |c_2|^2 + \frac{1}{2}|c_3+c_4|^2
\end{align*}

\paragraph{Probability of measuring $\beta$:}
The projection of $\ket{\psi}$ onto $V_\beta$ is $P_\beta \ket{\psi} = \frac{1}{\sqrt{2}} (c_3-c_4)\ket{4}$.
$$
P(\beta) = \norm{P_\beta \ket{\psi}}^2 = \frac{1}{2} |c_3-c_4|^2
$$
As required, $P(\alpha) + P(\beta) = \sum |c_i|^2 = 1$.

\subsection{Post-Measurement States and Expectation Value}
Upon measurement, the state collapses to the normalized projection onto the corresponding eigenspace.
\begin{itemize}
    \item If the outcome is $\alpha$, the state becomes $\ket{\psi'}_\alpha = \frac{P_\alpha \ket{\psi}}{\norm{P_\alpha \ket{\psi}}}$. The state is now confined to the 3D subspace $V_\alpha$, but is not fully determined.
    \item If the outcome is $\beta$, the state becomes $\ket{\psi'}_\beta = \frac{P_\beta \ket{\psi}}{\norm{P_\beta \ket{\psi}}} = e^{i\phi}\ket{4}$, which is a uniquely determined state (up to a global phase).
\end{itemize}
The expectation value of $A$ is given by $\expval{A}{\psi} = \alpha P(\alpha) + \beta P(\beta)$.

\subsection{Algebraic Construction of Projectors}
An elegant and powerful method to construct projectors utilizes the functional calculus of operators. Since $A$ is diagonalizable, its minimal polynomial is $m(\lambda) = (\lambda-\alpha)(\lambda-\beta)$. The projector $P_\alpha$ can be expressed as a polynomial in $A$, namely the Lagrange interpolating polynomial $q_\alpha(\lambda)$ that satisfies $q_\alpha(\alpha)=1$ and $q_\alpha(\beta)=0$.
$$
q_\alpha(\lambda) = \frac{\lambda - \beta}{\alpha - \beta} \implies P_\alpha = \frac{A - \beta I}{\alpha - \beta}
$$
This algebraic formula yields the exact same matrix operator as the geometric construction via sum of dyads, providing a crucial consistency check.

\section{Lifting the Degeneracy: Operator \texorpdfstring{$B$}{B}}

The 3-fold degeneracy of eigenvalue $\alpha$ implies $A$ is not a complete observable. We introduce a second observable $B$ to partially lift this degeneracy. For $A$ and $B$ to be simultaneously measurable, they must commute: $[A,B]=0$. This condition implies that $B$ must be block-diagonal with respect to the eigenspace decomposition of $A$.

\subsection{Construction of Operator \texorpdfstring{$B$}{B}}
We design $B$ to act non-trivially on $V_\alpha$ and to possess its own degenerate spectrum. Within $V_\alpha$, let $B$ have two eigenvalues: $\gamma$ (2-fold degenerate) and $\delta$ (non-degenerate), with $\gamma \neq \delta$.
The eigenspaces of $B$ within $V_\alpha$ are:
\begin{itemize}
    \item $W_\gamma = \text{span}\{\ket{b_1}, \ket{b_2}\}$ for eigenvalue $\gamma$.
    \item $W_\delta = \text{span}\{\ket{b_3}\}$ for eigenvalue $\delta$.
\end{itemize}
To make $B$ non-diagonal in the computational basis, we choose an orthonormal basis for these eigenspaces that mixes $\{\ket{1}, \ket{2}, \ket{3}\}$:
$$
\ket{b_1} = \ket{3}, \quad \ket{b_2} = \frac{1}{\sqrt{2}}(\ket{1}+\ket{2}), \quad \ket{b_3} = \frac{1}{\sqrt{2}}(\ket{1}-\ket{2})
$$
The action of $B$ restricted to $V_\alpha$ is $B_\alpha = \gamma(\dyad{b_1}{b_1} + \dyad{b_2}{b_2}) + \delta\dyad{b_3}{b_3}$. To complete the definition of $B$ on $\mathcal{H}$, we define its action on $V_\beta=\text{span}\{\ket{4}\}$, for instance by assigning it the eigenvalue $\delta$. The full matrix for $B$ in the computational basis is:
$$
\mathbf{B} = \begin{pmatrix}
\frac{\gamma+\delta}{2} & \frac{\gamma-\delta}{2} & 0 & 0 \\
\frac{\gamma-\delta}{2} & \frac{\gamma+\delta}{2} & 0 & 0 \\
0 & 0 & \gamma & 0 \\
0 & 0 & 0 & \delta
\end{pmatrix}
$$
By construction, $[A,B]=0$.

\subsection{Sequential Measurement of \texorpdfstring{$B$}{B}}
Suppose a measurement of $A$ yields the outcome $\alpha$. The system is now in the state $\ket{\psi'}_\alpha$. We now measure $B$. The possible outcomes are $\gamma$ and $\delta$. The conditional probability $P(\text{outcome of } B | \text{outcome of } A)$ is calculated as:
$$
P(\gamma|\alpha) = \norm{P_\gamma \ket{\psi'}_\alpha}^2 = \bra{\psi'}_\alpha P_\gamma \ket{\psi'}_\alpha
$$
where $P_\gamma$ is the projector onto the eigenspace $W_\gamma$. From its basis vectors, $P_\gamma = \dyad{b_1}{b_1} + \dyad{b_2}{b_2}$.
It is often simpler to work with the unnormalized post-A-measurement state, $\ket{\phi_A} = P_\alpha\ket{\psi}$. The conditional probability is then:
$$
P(\gamma|\alpha) = \frac{\norm{P_\gamma \ket{\phi_A}}^2}{\norm{\ket{\phi_A}}^2} = \frac{|c_1|^2 + \frac{1}{2}|c_3+c_4|^2}{|c_1|^2 + |c_2|^2 + \frac{1}{2}|c_3+c_4|^2}
$$
And similarly for the outcome $\delta$, with projector $P_\delta = \dyad{b_3}{b_3}$:
$$
P(\delta|\alpha) = \frac{\norm{P_\delta \ket{\phi_A}}^2}{\norm{\ket{\phi_A}}^2} = \frac{|c_2|^2}{|c_1|^2 + |c_2|^2 + \frac{1}{2}|c_3+c_4|^2}
$$
If the outcome is $(\alpha, \delta)$, the state collapses to $\ket{b_3}$, which is uniquely specified. However, if the outcome is $(\alpha, \gamma)$, the state collapses into the 2D subspace $W_\gamma$, indicating a remaining degeneracy.

\section{Completing the Set: Operator \texorpdfstring{$C$}{C}}
To lift the final degeneracy, we introduce a third operator $C$ that commutes with both $A$ and $B$. This requires that $C$ respects the simultaneous eigenspace decomposition of $\{A,B\}$. The only remaining degenerate eigenspace is $W_{\alpha,\gamma} \equiv W_\gamma$.

\subsection{Construction of Operator \texorpdfstring{$C$}{C}}
We define $C$ to act non-trivially only within $W_\gamma$. Let its eigenvalues be $\lambda_1 \neq \lambda_2$. Its eigenvectors, $\{\ket{c_1}, \ket{c_2}\}$, must form an orthonormal basis for $W_\gamma$. We choose:
\begin{align*}
    \ket{c_1} &= \frac{1}{\sqrt{2}}(\ket{b_1} + \ket{b_2}) = \frac{1}{2}(\ket{1}+\ket{2}) + \frac{1}{\sqrt{2}}\ket{3} \\
    \ket{c_2} &= \frac{1}{\sqrt{2}}(\ket{b_1} - \ket{b_2}) = -\frac{1}{2}(\ket{1}+\ket{2}) + \frac{1}{\sqrt{2}}\ket{3}
\end{align*}
We define the action of $C$ to be zero on all other simultaneous eigenspaces. The spectral decomposition is $C = \lambda_1 \dyad{c_1}{c_1} + \lambda_2 \dyad{c_2}{c_2}$. This guarantees $[C,A]=[C,B]=0$.

\subsection{Matrix Representation of \texorpdfstring{$C$}{C}}
To find the matrix $\mathbf{C}$ in the computational basis, we compute the dyads:
\begin{align*}
\dyad{c_1}{c_1} &= \frac{1}{4}\qty(\dyad{1}{1}+\dyad{1}{2}+\dyad{2}{1}+\dyad{2}{2}) + \frac{1}{2}\dyad{3}{3} + \frac{1}{2\sqrt{2}}\qty(\dyad{1}{3}+\dyad{3}{1}+\dyad{2}{3}+\dyad{3}{2}) \\
\dyad{c_2}{c_2} &= \frac{1}{4}\qty(\dyad{1}{1}+\dyad{1}{2}+\dyad{2}{1}+\dyad{2}{2}) + \frac{1}{2}\dyad{3}{3} - \frac{1}{2\sqrt{2}}\qty(\dyad{1}{3}+\dyad{3}{1}+\dyad{2}{3}+\dyad{3}{2})
\end{align*}
Combining these gives the matrix for $C$:
$$
\mathbf{C} = \frac{1}{4}
\begin{pmatrix}
\lambda_1+\lambda_2 & \lambda_1+\lambda_2 & \sqrt{2}(\lambda_1-\lambda_2) & 0 \\
\lambda_1+\lambda_2 & \lambda_1+\lambda_2 & \sqrt{2}(\lambda_1-\lambda_2) & 0 \\
\sqrt{2}(\lambda_1-\lambda_2) & \sqrt{2}(\lambda_1-\lambda_2) & 2(\lambda_1+\lambda_2) & 0 \\
0 & 0 & 0 & 0
\end{pmatrix}
$$

\subsection{Final Measurement Probabilities}
Suppose the measurement sequence has yielded outcomes $(\alpha, \gamma)$. The unnormalized state is $\ket{\phi_{AB}} = P_\gamma P_\alpha \ket{\psi} = c_1\ket{b_2} + \frac{c_3+c_4}{\sqrt{2}}\ket{b_1}$.
We now measure $C$. The probability of obtaining $\lambda_1$ is $P(\lambda_1|\alpha, \gamma) = \frac{|\braket{c_1}{\phi_{AB}}|^2}{\norm{\ket{\phi_{AB}}}^2}$.
The required inner product is $\braket{c_1}{\phi_{AB}} = \frac{1}{\sqrt{2}}\qty(c_1 + \frac{c_3+c_4}{\sqrt{2}})$.
The probability is:
$$
P(\lambda_1|\alpha, \gamma) = \frac{\frac{1}{2}\abs{c_1 + \frac{c_3+c_4}{\sqrt{2}}}^2}{|c_1|^2 + \frac{1}{2}|c_3+c_4|^2}
$$
And for $\lambda_2$:
$$
P(\lambda_2|\alpha, \gamma) = \frac{\frac{1}{2}\abs{-c_1 + \frac{c_3+c_4}{\sqrt{2}}}^2}{|c_1|^2 + \frac{1}{2}|c_3+c_4|^2}
$$
These probabilities sum to 1. After this final measurement, the state collapses to either $\ket{c_1}$ or $\ket{c_2}$, and the state of the system is now uniquely determined.

\section{Conclusion: The Complete Set of Commuting Observables}
The set of operators $\{A, B, C\}$ forms a CSCO. Their simultaneous eigenvectors form a complete orthonormal basis for the Hilbert space $\mathcal{H}$. Each basis vector is uniquely specified by a triplet of eigenvalues $(a,b,c)$. The complete basis and corresponding eigenvalues are summarized below:

\begin{table}[h!]
\centering
\begin{tabular}{c|c|ccc}
\hline\hline
\textbf{Basis Vector} & \textbf{Definition in Computational Basis} & \multicolumn{3}{c}{\textbf{Eigenvalues}} \\
 & & $A$ & $B$ & $C$ \\
\hline
$\ket{c_1}$ & $\frac{1}{2}(\ket{1}+\ket{2}) + \frac{1}{\sqrt{2}}\ket{3}$ & $\alpha$ & $\gamma$ & $\lambda_1$ \\
$\ket{c_2}$ & $-\frac{1}{2}(\ket{1}+\ket{2}) + \frac{1}{\sqrt{2}}\ket{3}$ & $\alpha$ & $\gamma$ & $\lambda_2$ \\
$\ket{b_3}$ & $\frac{1}{\sqrt{2}}(\ket{1}-\ket{2})$ & $\alpha$ & $\delta$ & $0$ \\
$\ket{4}$ & $\ket{4}$ & $\beta$ & $\delta$ & $0$ \\
\hline\hline
\end{tabular}
\caption{The CSCO basis and their corresponding eigenvalues.}
\label{tab:csco_basis}
\end{table}

A sequential measurement of $A$, then $B$, then $C$ will project an arbitrary initial state $\ket{\psi}$ onto one of these four basis vectors, with probabilities calculable at each stage. The process demonstrates how introducing compatible observables resolves spectral degeneracies and allows for the complete determination of a quantum state.

%%%%%%%%%%%%%%%%%%%%%%%%%%%%%%%%%%%%%%%%%%%%%%%%%%%%%%%%%%%%%%%%%%%%%%%%%%%%%%%
%                      SECTION 6: TIME EVOLUTION
%%%%%%%%%%%%%%%%%%%%%%%%%%%%%%%%%%%%%%%%%%%%%%%%%%%%%%%%%%%%%%%%%%%%%%%%%%%%%%%

\section{Time Evolution of the System}

Having fully specified the state of our system through a sequence of measurements, we now consider its dynamics, governed by the Time-Dependent Schrödinger Equation (TDSE). For a discrete spectrum and a time-independent Hamiltonian $H$, the equation and its formal solution are:
$$
i\hbar \frac{d}{dt}\ket{\psi(t)} = H \ket{\psi(t)} \quad \implies \quad \ket{\psi(t)} = \sum_n e^{-iE_n t/\hbar} \ket{E_n}\braket{E_n}{\psi(0)}
$$
where $\{\ket{E_n}\}$ is the orthonormal basis of energy eigenvectors (autokets) with corresponding energy eigenvalues $\{E_n\}$.

\subsection{Defining a Hamiltonian from the CSCO}

To proceed, we must define a Hamiltonian for the system. A physically and algebraically motivated choice is to construct the Hamiltonian from our set of commuting observables. Since $A$, $B$, and $C$ all commute with each other, any function of them also commutes. Let us define $H$ as a linear combination of our CSCO:
$$
H = k_A A + k_B B + k_C C
$$
where $k_A, k_B, k_C \in \mathbb{R}$ are constants that define the energy scale associated with each observable. This construction guarantees that $H$ is compatible with $A$, $B$, and $C$, meaning they can all be measured simultaneously without uncertainty.

\subsection{Energy Spectrum and Eigenstates}

A direct consequence of our construction is that the simultaneous eigenbasis of the CSCO, which we have painstakingly constructed, is also the eigenbasis of our Hamiltonian $H$. Let's rename our basis vectors to reflect that they are energy eigenstates:
\begin{align*}
    \ket{E_1} &:= \ket{c_1} \\
    \ket{E_2} &:= \ket{c_2} \\
    \ket{E_3} &:= \ket{b_3} \\
    \ket{E_4} &:= \ket{4}
\end{align*}
The energy eigenvalue for each state is found by applying $H$ and using the known eigenvalues of $A$, $B$, and $C$:
\begin{align*}
    H\ket{E_1} &= (k_A\alpha + k_B\gamma + k_C\lambda_1)\ket{E_1} \implies E_1 = k_A\alpha + k_B\gamma + k_C\lambda_1 \\
    H\ket{E_2} &= (k_A\alpha + k_B\gamma + k_C\lambda_2)\ket{E_2} \implies E_2 = k_A\alpha + k_B\gamma + k_C\lambda_2 \\
    H\ket{E_3} &= (k_A\alpha + k_B\delta + k_C \cdot 0)\ket{E_3} \implies E_3 = k_A\alpha + k_B\delta \\
    H\ket{E_4} &= (k_A\beta + k_B\delta + k_C \cdot 0)\ket{E_4} \implies E_4 = k_A\beta + k_B\delta
\end{align*}
By choosing the constants $k_A, k_B, k_C$ appropriately, one can ensure that the energy spectrum is non-degenerate, with each energy level corresponding to a unique state vector.

\subsection{General Solution for \texorpdfstring{$\ket{\psi(t)}$}{psi(t)}}

The time evolution of our arbitrary initial state, $\ket{\psi(0)}$, is now determined. First, we project the initial state onto the energy eigenbasis to find the expansion coefficients $d_n = \braket{E_n}{\psi(0)}$.
$$
\ket{\psi(0)} = d_1\ket{E_1} + d_2\ket{E_2} + d_3\ket{E_3} + d_4\ket{E_4}
$$
The state at any subsequent time $t$ is then given by evolving each component with its characteristic complex phase:
$$
\ket{\psi(t)} = d_1 e^{-iE_1 t/\hbar}\ket{E_1} + d_2 e^{-iE_2 t/\hbar}\ket{E_2} + d_3 e^{-iE_3 t/\hbar}\ket{E_3} + d_4 e^{-iE_4 t/\hbar}\ket{E_4}
$$
This expression is the complete solution to the dynamics of the system. It demonstrates how an initial superposition state evolves as a coherent "rotation" in Hilbert space, with each energy eigenstate component acquiring phase at a rate determined by its energy. This concludes our construction and analysis of a complete quantum mechanical problem.

\end{document}

\end{document}
