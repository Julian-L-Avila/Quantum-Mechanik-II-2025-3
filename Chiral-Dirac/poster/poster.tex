% https://github.com/andiac/gemini-cam
% a fork of https://github.com/anishathalye/gemini
% also refer to https://github.com/k4rtik/uchicago-poster

\documentclass[final]{beamer}

% ====================
% Packages
% ====================

\usepackage[T1]{fontenc}
\usepackage{lmodern}
\usepackage[orientation=portrait,size=a0,scale=1.15]{beamerposter}
\usetheme{gemini}
\usecolortheme{nott}
\usepackage{graphicx}
\usepackage{booktabs}
\usepackage{tikz}
\usepackage{pgfplots}
\usepackage{svg}
\usepackage{cleveref}
\pgfplotsset{compat=1.14}
\usepackage{anyfontsize}
\usepackage{biblatex}

\addbibresource{../chiral-references.bib}

% ====================
% Lengths
% ====================

% If you have N columns, choose \sepwidth and \colwidth such that
% (N+1)*\sepwidth + N*\colwidth = \paperwidth
\newlength{\sepwidth}
\newlength{\colwidth}
\setlength{\sepwidth}{0.025\paperwidth}
\setlength{\colwidth}{0.45\paperwidth}

\newcommand{\separatorcolumn}{\begin{column}{\sepwidth}\end{column}}

% ====================
% Title
% ====================

\title{A Chiral Symmetric Dirac Equation \\
Watson \& Musielak (Int. J. Mod. Phys. A, 2020)}
\author{Julian L. Avila-Martinez \and Laura Y. Herrera-Martinez \and Sebastian
Rodriguez-Garcia}

\institute[shortinst]{F\'isica, Universidad Distrital Francisco Jos\'e de Caldas}

\footercontent{
  \href{https://github.com/Julian-L-Avila/Quantum-Mechanik-II-2025-3}{github.com/Julian-L-Avila/Quantum-Mechanik-II-2025-3}
}

% ====================
% Logo (optional)
% ====================

% use this to include logos on the left and/or right side of the header:
\logoleft{\hspace{8ex}
  \includesvg[height=8.5cm]{./figures/Logo_Escudo_invertido.svg}
}

% ====================
% Body
% ====================

\begin{document}

\begin{frame}[t]
\begin{columns}[t]
\separatorcolumn

\begin{column}{\colwidth}

  \begin{block}{What is Chirality?}

  \end{block}

  \begin{alertblock}{Chiral Symmetric Dirac Equation}

    \begin{equation}\label{eq:decs}
      \left( i \gamma^\mu \partial_\mu - m e^{- i 2 \alpha \gamma^5} \right)
      \psi = 0
    \end{equation}

  \end{alertblock}

  \begin{block}{Chiral Angle and Mass}

    \heading{What is the Chiral Angle?}
    The DECS can be re-written (\cref{eq:decs}):

    \begin{equation}
      \left( i \gamma^\mu \partial_\mu - M - \tilde{M} \gamma^5 \right) \psi = 0
    \end{equation}

    Where:
    \begin{itemize}
      \item \( M = m \cos{2\alpha}\) is the standard \textbf{scalar mass}.
      \item \( \tilde{M} = - i m \sin{2\alpha} \) is a \textbf{pseudoscalar
        mass}. The chiral angle α mixes the scalar and pseudoscalar mass terms.
    \end{itemize}

    \heading{Generating Mass: A Two-Field Higgs Model}
    These \( M \) and \( \tilde{M} \) terms can arise from Yukawa couplings to
    two Higgs-like fields:
    a scalar \( \phi_{1} \) and a pseudoscalar \( \phi_{2} \).

    \begin{equation}\label{eq:yukawa-lagrangian}
      \mathcal{L}_{Y} \approx  - \frac{\lambda_{1} v_{1}}{\sqrt{2}} \bar{\psi}
      \psi - \frac{\lambda_{2} v_{2}}{\sqrt{2}} \bar{\psi} \gamma^5 \psi
    \end{equation}

  This \emph{fixes} the parameters \( m \) and \( \alpha \):
  \begin{gather}
    m = \sqrt{\frac{\lambda_{1}^2 v_{1}^2 - \lambda_{2}^2 v_{2}^2}{2} } \\
    \alpha = \frac{i}{4} \ln{\left( \frac{\lambda_{1} v_{1} + \lambda_{2}
    v_{2}}{\lambda_{1} v_{1} - \lambda_{2} v_{2}} \right)}
  \end{gather}

  \end{block}

\end{column}

\separatorcolumn

\begin{column}{\colwidth}

  \begin{block}{Neutrinos and Dark Matter}
  
    \heading{Neutrino Failure}
    The paper proposes its main application in explaining ``anomalously small''
    neutrino masses. It suggests the scalar (\( M \) and the pseudoscalar \(
    \tilde{M} \) mass contributions, which arise form \(\phi_{1} \) and \(
    \phi_{2} \) respectively, nearly cancel each other out.

    \begin{equation}
      m_{\nu} = \frac{1}{2 \sqrt{2}} \left( \lambda_{1} v_1 - \lambda_{2} v_2 \right)
    \end{equation}

    However, the paper's derivation contains a fundamental contradiction. It
    states that the neutrino field has no right-chiral component (Eq. 26: \(
    \frac{1}{2}( 1 + \gamma^5) \nu = 0 \)). As is well-known, a Dirac mass term
    (both scalar and pseudoscalar) requires both chiral components to be
    non-zero.

    \heading{Dark Matter Candidate}
    \begin{itemize}
      \item The pseudoscalar Higgs field, \( \phi_2 \), is proposed as a Dark
        Matter candidate.
      \item It couples to Standard Model particles only via this pseudoscalar
        Yukawa coupling, primarily interacting with neutrinos.
      \item This makes it massive, long-lived, and ``dark,'' satisfying the
        requirements for a WIMP-like particle.
    \end{itemize}
    
  \end{block}

  \begin{block}{References}

    \nocite{*}
    \footnotesize{\printbibliography}

  \end{block}

\end{column}
\separatorcolumn

\end{columns}
\end{frame}

\end{document}
