% https://github.com/andiac/gemini-cam
% a fork of https://github.com/anishathalye/gemini
% also refer to https://github.com/k4rtik/uchicago-poster

\documentclass[final]{beamer}

% ====================
% Packages
% ====================

\usepackage[T1]{fontenc}
\usepackage{lmodern}
\usepackage[orientation=portrait,size=a0,scale=1.15]{beamerposter}
\usetheme{gemini}
\usecolortheme{nott}
\usepackage{graphicx}
\usepackage{booktabs}
\usepackage{tikz}
\usepackage{pgfplots}
\usepackage{svg}
\usepackage{cleveref}
\pgfplotsset{compat=1.14}
\usepackage{anyfontsize}
\usepackage{biblatex}

\addbibresource{../chiral-references.bib}

% ====================
% Lengths
% ====================

% If you have N columns, choose \sepwidth and \colwidth such that
% (N+1)*\sepwidth + N*\colwidth = \paperwidth
\newlength{\sepwidth}
\newlength{\colwidth}
\setlength{\sepwidth}{0.025\paperwidth}
\setlength{\colwidth}{0.45\paperwidth}

\newcommand{\separatorcolumn}{\begin{column}{\sepwidth}\end{column}}

% ====================
% Title
% ====================

\title{A Chiral Symmetric Dirac Equation \\
Watson \& Musielak (Int. J. Mod. Phys. A, 2020)}
\author{Julian L. Avila-Martinez \and Laura Y. Herrera-Martinez \and Sebastian
Rodriguez-Garcia}

\institute[shortinst]{F\'isica, Universidad Distrital Francisco Jos\'e de Caldas}

\footercontent{
  \href{https://github.com/Julian-L-Avila/Quantum-Mechanik-II-2025-3}{github.com/Julian-L-Avila/Quantum-Mechanik-II-2025-3}
}

% ====================
% Logo (optional)
% ====================

% use this to include logos on the left and/or right side of the header:
\logoleft{\hspace{8ex}
  \includesvg[height=8.5cm]{./figures/Logo_Escudo_invertido.svg}
}

% ====================
% Body
% ====================

\begin{document}

\begin{frame}[t]
\begin{columns}[t]
\separatorcolumn

\begin{column}{\colwidth}

  \begin{block}{What is Chirality?}
    A Lorentz-invariant property distinguishing how spinor components transform
    under boosts $\Lambda$:

    \begin{gather}
      \psi = \psi_L + \psi_R \\
      \psi_L \to \Lambda \psi_L \quad , \quad \psi_R \to \Lambda^{-1} \psi_R
    \end{gather}

    The components transform under inequivalent Lorentz group representations:
    \begin{itemize}
      \item \textbf{Left-handed ($\psi_L$):} $(\frac{1}{2}, 0)$
      \item \textbf{Right-handed ($\psi_R$):} $(0, \frac{1}{2})$
    \end{itemize}
  \end{block}


  \begin{alertblock}{Chiral Symmetric Dirac Equation}

    \begin{equation}\label{eq:decs}
      \left( i \gamma^\mu \partial_\mu - m e^{- i 2 \alpha \gamma^5} \right)
      \psi = 0
    \end{equation}

    \heading{The properties that the DECS must satisfy}

    In the theory of irreducible representations of the Poincaré group,
    each irreducible representation is defined by how the objects it describes
    transform under $\mathcal{P}$.

    \begin{itemize}

      \item \textbf{Poincaré Principle:}
        Physics is unchanged under Lorentz transformations and translations.

      \item \textbf{Gauge Invariance:}
        Interactions emerge from requiring local symmetry of the Lagrangian.
      \item \textbf{Locality:}  Locality: Fields interact only at the same space–time point.
    \end{itemize}
  \end{alertblock}

  \begin{block}{DECS derivation}

    \heading{1. Start with the translation operator eigenvalue equation:}
    Given that $ P_\mu = i \partial_\mu $:

    \begin{equation}
      i\,\partial_{\mu} \phi = k _{\mu} \phi \quad \longmapsto \quad X^\mu \partial_\mu
      \mathbf{\psi} = - Y \mathbf{\psi}
    \end{equation}

    \heading{2. Impose Lorentz invariance:}

    This yields the most general first-order equation with two degrees of
    freedom ($y_L/x_R$, $y_R/x_L$):

    \begin{equation}\label{eq:pro}
      \left[
        i\gamma^{\mu}\partial_{\mu}
        + i\!\left(
          \dfrac{y_L}{x_R}P_L
          + \dfrac{y_R}{x_L}P_R
        \right)
      \right]\psi = 0
    \end{equation}

    \heading{3. Squaring the Hamiltonian reveals the mass term:}

    The squared Hamiltonian derived from \cref{eq:pro} reads:

    \begin{equation}
      H^2\psi =
      \left(
        \partial_k\partial^k -
        \dfrac{y_L y_R}{x_L x_R}
      \right)\psi
    \end{equation}

    \heading{4. Identify the mass $m$ and chiral angle $\alpha$:}
    The remaining degrees of freedom are identified as:

    \begin{equation}
      m \equiv \pm i
      \sqrt{\dfrac{y_L y_R}{x_L x_R}}
      \quad
      \alpha \equiv -\dfrac{i}{2}
      \ln\!\left(
        \mp\sqrt{\dfrac{x_L y_L}{x_R y_R}}
      \right)
    \end{equation}

    Substituting these into \cref{eq:pro} gives the DECS \cref{eq:decs}.
  \end{block}

  \begin{block}{Chiral Angle and Mass}

    \heading{What is the Chiral Angle?}
    Rewriting the DECS \cref{eq:decs} reveals scalar and pseudoscalar mass terms:

    \begin{equation}
      \left( i \gamma^\mu \partial_\mu - M - \tilde{M} \gamma^5 \right) \psi = 0
    \end{equation}

    Where $\alpha$ mixes the \textbf{scalar mass $M$} and \textbf{pseudoscalar
    mass $\tilde{M}$}:
    \begin{itemize}
      \item \( M = m \cos{2\alpha}\) (Scalar)
      \item \( \tilde{M} = - i m \sin{2\alpha} \) (Pseudoscalar)
    \end{itemize}
  \end{block}
\end{column}

\separatorcolumn

\begin{column}{\colwidth}

  \begin{block}{Chiral Angle and Mass}

    \heading{Generating Mass: A Two-Field Higgs Model}
    These \( M \) and \( \tilde{M} \) terms can arise from Yukawa couplings to
    two Higgs-like fields:
    a scalar \( \phi_{1} \) and a pseudoscalar \( \phi_{2} \).

    \begin{equation}\label{eq:yukawa-lagrangian}
      \mathcal{L}_{Y} \approx  - \frac{\lambda_{1} v_{1}}{\sqrt{2}} \bar{\psi}
      \psi - \frac{\lambda_{2} v_{2}}{\sqrt{2}} \bar{\psi} \gamma^5 \psi
    \end{equation}

    This \emph{fixes} the parameters \( m \) and \( \alpha \):
    \begin{gather}
      m = \sqrt{\frac{\lambda_{1}^2 v_{1}^2 - \lambda_{2}^2 v_{2}^2}{2} } \\
      \alpha = \frac{i}{4} \ln{\left( \frac{\lambda_{1} v_{1} + \lambda_{2}
      v_{2}}{\lambda_{1} v_{1} - \lambda_{2} v_{2}} \right)}
    \end{gather}

  \end{block}

  \begin{block}{Neutrinos and Dark Matter}
    \heading{Neutrino Mass Proposal (and its flaw)}
    \begin{itemize}
      \item \textbf{Paper's Goal:} Explain small $\nu$ mass by cancellation between $M$ and $\tilde{M}$:
        \begin{equation}
          m_{\nu} = \frac{1}{2 \sqrt{2}} \left( \lambda_{1} v_1 - \lambda_{2} v_2 \right)
        \end{equation}
      \item \textbf{Fundamental Contradiction:}
      \item The paper \textbf{states} $\nu$ has no right-chiral component
        ($\frac{1}{2}( 1 + \gamma^5) \nu = 0$).
      \item \emph{But...} Dirac mass terms (both $M$ and $\tilde{M}$) \textbf{require}
        both left and right chiral components to be non-zero.
    \end{itemize}

    \heading{Dark Matter Candidate}
    Despite the neutrino flaw, the model offers a DM candidate:
    \begin{itemize}
      \item The \textbf{pseudoscalar Higgs $\phi_2$} is proposed as Dark Matter.
      \item It couples to the SM only via the pseudoscalar Yukawa (primarily to neutrinos).
      \item This makes it massive, long-lived, and ``dark''—a viable WIMP-like particle.
    \end{itemize}
  \end{block}

  \begin{alertblock}{Knowledge Gap}
    \begin{itemize}
      \item Hermiticity of the Chiral Symmetric Hamiltonian
      \item A Chiral Equation for higher spins (Proca, Rarita-Schwinger-like equations)
      \item Detection and measurement of the free parameters that specify the model
      \item Validity of the proposed neutrino mass as a Dirac mass
    \end{itemize}
  \end{alertblock}

  \begin{block}{Conclusions}
    \begin{itemize}
      \item The DECS is a valid generalization of the Dirac equation, formally introducing scalar ($M$) and pseudoscalar ($\tilde{M}$) mass terms mixed by a chiral angle $\alpha$.
      \item A two-field Higgs model ($\phi_1, \phi_2$) can generate these mass terms.
      \item \textbf{Key Flaw:} The paper's application of this model to neutrinos is contradictory, as Dirac mass terms cannot apply to a purely left-chiral field.
      \item The pseudoscalar Higgs ($\phi_2$) remains a viable WIMP-like Dark Matter candidate.
    \end{itemize}
  \end{block}

  \begin{block}{References}

    \nocite{*}
    \footnotesize{\printbibliography}

  \end{block}









\end{column}
\separatorcolumn

\end{columns}
\end{frame}

\end{document}
