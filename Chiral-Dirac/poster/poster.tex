% https://github.com/andiac/gemini-cam
% a fork of https://github.com/anishathalye/gemini
% also refer to https://github.com/k4rtik/uchicago-poster

\documentclass[final]{beamer}

% ====================
% Packages
% ====================

\usepackage[T1]{fontenc}
\usepackage{lmodern}
\usepackage[orientation=portrait,size=a0,scale=1.15]{beamerposter}
\usetheme{gemini}
\usecolortheme{nott}
\usepackage{graphicx}
\usepackage{booktabs}
\usepackage{tikz}
\usepackage{pgfplots}
\usepackage{svg}
\usepackage{cleveref}
\pgfplotsset{compat=1.14}
\usepackage{anyfontsize}
\usepackage{biblatex}

\addbibresource{../chiral-references.bib}

% ====================
% Lengths
% ====================

% If you have N columns, choose \sepwidth and \colwidth such that
% (N+1)*\sepwidth + N*\colwidth = \paperwidth
\newlength{\sepwidth}
\newlength{\colwidth}
\setlength{\sepwidth}{0.025\paperwidth}
\setlength{\colwidth}{0.45\paperwidth}

\newcommand{\separatorcolumn}{\begin{column}{\sepwidth}\end{column}}

% ====================
% Title
% ====================

\title{A Chiral Symmetric Dirac Equation \\
Watson \& Musielak (Int. J. Mod. Phys. A, 2020)}
\author{Julian L. Avila-Martinez \and Laura Y. Herrera-Martinez \and Sebastian
Rodriguez-Garcia}

\institute[shortinst]{F\'isica, Universidad Distrital Francisco Jos\'e de Caldas}

\footercontent{
  \href{https://github.com/Julian-L-Avila/Quantum-Mechanik-II-2025-3}{github.com/Julian-L-Avila/Quantum-Mechanik-II-2025-3}
}

% ====================
% Logo (optional)
% ====================

% use this to include logos on the left and/or right side of the header:
\logoleft{\hspace{8ex}
  \includesvg[height=8.5cm]{./figures/Logo_Escudo_invertido.svg}
}

% ====================
% Body
% ====================

\begin{document}

\begin{frame}[t]
\begin{columns}[t]
\separatorcolumn

\begin{column}{\colwidth}

  \begin{block}{What is Chirality?}
  Chirality is a Lorentz-invariant property of spinor fields that distinguishes how their left- and right-handed components transform under Lorentz transformations.


  
  
  %It is defined as the eigenvalue of the
  %pseudoscalar (volume) element of the underlying Clifford algebra,
  %which in 3+1D spacetime is the \gamma^5 operator.
\begin{equation}
\psi =  
\begin{pmatrix}  
\chi_L \  
\chi_R  
\end{pmatrix},  
\qquad  
\chi_L \to \Lambda_L \chi_L, \quad  
\chi_R \to \Lambda_R \chi_R  
\end{equation}


The two chiral components transform with inequivalent representations of the Lorentz group:  
left-handed spinors with $((\tfrac{1}{2}, 0)$), and right-handed spinors with $((0, \tfrac{1}{2}))$.


\begin{alertblock}{Chiral Symmetric Dirac Equation}

\begin{equation}\label{eq:decs}
      \left( i \gamma^\mu \partial_\mu - m e^{- i 2 \alpha \gamma^5} \right)
      \psi = 0
    \end{equation}

  \heading{The properties that the DECS must satisfy}

      In the theory of irreducible representations of the Poincaré group \cite{wigner-1939}, 
each irreducible representation is defined by how the objects it describes 
transform under $\mathcal{P}$.

\begin{itemize}

  \item \textbf{Poincaré Principle:}  
  Physics is unchanged under Lorentz transformations and translations.

  \item \textbf{Gauge Invariance:}  
 Interactions emerge from requiring local symmetry of the Lagrangian.
  \item \textbf{Locality:}  Locality: Fields interact only at the same space–time point.
  


\end{itemize}



  \end{alertblock}

  \heading{The DECS drivation:}

When this operator acts on an individual state, defined as $ P_{\mu} \equiv i\,\partial_{\mu}
$ and the following eigenvalue equation is obtained:


\begin{equation}
 i\,\partial_{\mu} \phi = k _{\mu} \phi.
\end{equation}

Starting from the translational eigenvalue relation, Lorentz invariance is imposed to determine the operator structure of the relativistic wave equation.
This requirement constrains the possible tensorial forms of the matrices involved, leading to the most general Lorentz-invariant first-order equation for a bispinor field.


\begin{equation}\label{eq:pro}
\Big[
i\gamma^{\mu}\partial_{\mu}
+ i\!\left(
\dfrac{y_L}{x_R}P_L
+ \dfrac{y_R}{x_L}P_R
\right)
\Big]\psi = 0.
\end{equation}


which represents the most general Lorentz-invariant first-order equation 
for a bispinor field and shows that there are only two independent degrees of freedom 
in the derived equation.  
The squared Hamiltonian derived from (\cref{eq:pro}) reads:

\begin{equation}
H^2\psi =
\left(
\partial_k\partial^k -
\dfrac{y_L y_R}{x_L x_R}
\right)\psi
\end{equation}

so that the propagation mass term naturally emerges and the remaining degree of freedom is identified with the choice 
of chiral basis, introducing the chiral angle $\alpha$ defined by

\begin{equation}
m \equiv \pm i
\sqrt{\dfrac{y_L y_R}{x_L x_R}}
\quad
\alpha \equiv -\dfrac{i}{2}
\ln\!\left(
\mp\sqrt{\dfrac{x_L y_L}{x_R y_R}}
\right)
\end{equation}

The final compact form of our Dirac equation with chiral symmetry is (\cref{eq:decs}), 
\end{block}



  

  \begin{block}{Chiral Angle and Mass}

    \heading{What is the Chiral Angle?}
    The DECS can be re-written (\cref{eq:decs}):

    \begin{equation}
      \left( i \gamma^\mu \partial_\mu - M - \tilde{M} \gamma^5 \right) \psi = 0
    \end{equation}

    Where:
    \begin{itemize}
      \item \( M = m \cos{2\alpha}\) is the standard \textbf{scalar mass}.
      \item \( \tilde{M} = - i m \sin{2\alpha} \) is a \textbf{pseudoscalar
        mass}. The chiral angle α mixes the scalar and pseudoscalar mass terms.
    \end{itemize}
  \end{block}

  

\end{column}

\separatorcolumn

\begin{column}{\colwidth}

  \begin{block}{Chiral Angle and Mass}

    

    \heading{Generating Mass: A Two-Field Higgs Model}
    These \( M \) and \( \tilde{M} \) terms can arise from Yukawa couplings to
    two Higgs-like fields:
    a scalar \( \phi_{1} \) and a pseudoscalar \( \phi_{2} \).

    \begin{equation}\label{eq:yukawa-lagrangian}
      \mathcal{L}_{Y} \approx  - \frac{\lambda_{1} v_{1}}{\sqrt{2}} \bar{\psi}
      \psi - \frac{\lambda_{2} v_{2}}{\sqrt{2}} \bar{\psi} \gamma^5 \psi
    \end{equation}

  This \emph{fixes} the parameters \( m \) and \( \alpha \):
  \begin{gather}
    m = \sqrt{\frac{\lambda_{1}^2 v_{1}^2 - \lambda_{2}^2 v_{2}^2}{2} } \\
    \alpha = \frac{i}{4} \ln{\left( \frac{\lambda_{1} v_{1} + \lambda_{2}
    v_{2}}{\lambda_{1} v_{1} - \lambda_{2} v_{2}} \right)}
  \end{gather}

  \end{block}

  \begin{block}{Neutrinos and Dark Matter}
  
    \heading{Neutrino Failure}
    The paper proposes its main application in explaining ``anomalously small''
    neutrino masses. It suggests the scalar (\( M \) and the pseudoscalar \(
    \tilde{M} \) mass contributions, which arise form \(\phi_{1} \) and \(
    \phi_{2} \) respectively, nearly cancel each other out.

    \begin{equation}
      m_{\nu} = \frac{1}{2 \sqrt{2}} \left( \lambda_{1} v_1 - \lambda_{2} v_2 \right)
    \end{equation}

    However, the paper's derivation contains a \textbf{fundamental
    contradiction}. It
    states that the neutrino field has no right-chiral component (Eq. 26: \(
    \frac{1}{2}( 1 + \gamma^5) \nu = 0 \)). As is well-known, a Dirac mass term
    (both scalar and pseudoscalar) requires both chiral components to be
    non-zero.

    \heading{Dark Matter Candidate}
    \begin{itemize}
      \item The pseudoscalar Higgs field, \( \phi_2 \), is proposed as a Dark
        Matter candidate.
      \item It couples to Standard Model particles only via this pseudoscalar
        Yukawa coupling, primarily interacting with neutrinos.
      \item This makes it massive, long-lived, and ``dark,'' satisfying the
        requirements for a WIMP-like particle.
    \end{itemize}
    
  \end{block}

  \begin{alertblock}{Knowledge Gap}
\begin{itemize}
    \item Hermiticity of the Chiral Symmetric Hamiltonian
    \item A Chiral Equation for higher spins (Proca, Rarita-Schwinger-like equations)
    \item Detection and measurement of the free parameters that specify the model
    \item Validity of the proposed neutrino mass as a Dirac mass
\end{itemize}
\end{alertblock}

\begin{block}{Conclusions}
\begin{itemize}
    \item The Chiral Symmetric Dirac Equation as a valid physical description of reality.
    \item The existence of both scalar and pseudoscalar mass.
    \item A new proposal for the origin of a pseudoscalar Higgs as dark matter.
\end{itemize}
\end{block}





  \begin{block}{References}

    \nocite{*}
    \footnotesize{\printbibliography}

  \end{block}







  

\end{column}
\separatorcolumn

\end{columns}
\end{frame}

\end{document}
