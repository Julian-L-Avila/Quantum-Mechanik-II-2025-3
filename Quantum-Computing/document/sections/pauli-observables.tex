\section{Observables de Pauli en el subespacio lógico}
\label{sec:observables_pauli}

En esta sección, se definen los operadores análogos a las matrices de Pauli,
pero que actúan exclusivamente dentro del subespacio lógico de 2D,
$\mathcal{H}_L$.

\subsection{Definición y Representación Matricial}

Los operadores de Pauli en la base lógica $\{|0_L\rangle, |1_L\rangle\}$ se
definen como:
\begin{align*}
  X_L &= |0_L\rangle\langle 1_L| + |1_L\rangle\langle 0_L| \\
  Y_L &= -i|0_L\rangle\langle 1_L| + i|1_L\rangle\langle 0_L| \\
  Z_L &= |0_L\rangle\langle 0_L| - |1_L\rangle\langle 1_L|
\end{align*}

Para encontrar sus representaciones matriciales, usamos la correspondencia de
los vectores base:
\[
  |0_L\rangle \to \begin{pmatrix} 1 \\ 0 \end{pmatrix} \quad \text{y} \quad |1_L\rangle \to \begin{pmatrix} 0 \\ 1 \end{pmatrix}
\]
Y los operadores (proyectores externos) correspondientes:
\begin{align*}
  |0_L\rangle\langle 0_L| &\to \begin{pmatrix} 1 \\ 0 \end{pmatrix} \begin{pmatrix} 1 & 0 \end{pmatrix} = \begin{pmatrix} 1 & 0 \\ 0 & 0 \end{pmatrix} \\
  |1_L\rangle\langle 1_L| &\to \begin{pmatrix} 0 \\ 1 \end{pmatrix} \begin{pmatrix} 0 & 1 \end{pmatrix} = \begin{pmatrix} 0 & 0 \\ 0 & 1 \end{pmatrix} \\
  |0_L\rangle\langle 1_L| &\to \begin{pmatrix} 1 \\ 0 \end{pmatrix} \begin{pmatrix} 0 & 1 \end{pmatrix} = \begin{pmatrix} 0 & 1 \\ 0 & 0 \end{pmatrix} \\
  |1_L\rangle\langle 0_L| &\to \begin{pmatrix} 0 \\ 1 \end{pmatrix} \begin{pmatrix} 1 & 0 \end{pmatrix} = \begin{pmatrix} 0 & 0 \\ 1 & 0 \end{pmatrix}
\end{align*}

Sustituyendo estos en las definiciones, obtenemos las representaciones
matriciales (que coinciden con las matrices de Pauli estándar, $\sigma_x,
\sigma_y, \sigma_z$):

\[
  X_L \to \begin{pmatrix} 0 & 1 \\ 0 & 0 \end{pmatrix} + \begin{pmatrix} 0 & 0 \\ 1 & 0 \end{pmatrix} = \begin{pmatrix} 0 & 1 \\ 1 & 0 \end{pmatrix}
\]
\[
  Y_L \to -i\begin{pmatrix} 0 & 1 \\ 0 & 0 \end{pmatrix} + i\begin{pmatrix} 0 & 0 \\ 1 & 0 \end{pmatrix} = \begin{pmatrix} 0 & -i \\ i & 0 \end{pmatrix}
\]
\[
  Z_L \to \begin{pmatrix} 1 & 0 \\ 0 & 0 \end{pmatrix} - \begin{pmatrix} 0 & 0 \\ 0 & 1 \end{pmatrix} = \begin{pmatrix} 1 & 0 \\ 0 & -1 \end{pmatrix}
\]

\subsection{Verificación de la Relación de Conmutación}

Verificamos la relación $[X_L, Y_L] = 2iZ_L$ usando las matrices obtenidas. El
conmutador se define como $[X_L, Y_L] = X_L Y_L - Y_L X_L$.

Primero, calculamos los productos:
\begin{align*}
  X_L Y_L &= \begin{pmatrix} 0 & 1 \\ 1 & 0 \end{pmatrix} \begin{pmatrix} 0 & -i \\ i & 0 \end{pmatrix}
  = \begin{pmatrix} (0)(0) + (1)(i) & (0)(-i) + (1)(0) \\ (1)(0) + (0)(i) & (1)(-i) + (0)(0) \end{pmatrix}
  = \begin{pmatrix} i & 0 \\ 0 & -i \end{pmatrix} \\
  Y_L X_L &= \begin{pmatrix} 0 & -i \\ i & 0 \end{pmatrix} \begin{pmatrix} 0 & 1 \\ 1 & 0 \end{pmatrix}
  = \begin{pmatrix} (0)(0) + (-i)(1) & (0)(1) + (-i)(0) \\ (i)(0) + (0)(1) & (i)(1) + (0)(0) \end{pmatrix}
  = \begin{pmatrix} -i & 0 \\ 0 & i \end{pmatrix}
\end{align*}

Ahora, calculamos el conmutador:
\[
  [X_L, Y_L] = X_L Y_L - Y_L X_L = \begin{pmatrix} i & 0 \\ 0 & -i \end{pmatrix} - \begin{pmatrix} -i & 0 \\ 0 & i \end{pmatrix}
  = \begin{pmatrix} i - (-i) & 0 \\ 0 & -i - i \end{pmatrix}
  = \begin{pmatrix} 2i & 0 \\ 0 & -2i \end{pmatrix}
\]

Finalmente, comparamos con el lado derecho de la ecuación, $2iZ_L$:
\[
  2iZ_L = 2i \begin{pmatrix} 1 & 0 \\ 0 & -1 \end{pmatrix} = \begin{pmatrix} 2i & 0 \\ 0 & -2i \end{pmatrix}
\]
Ambos resultados son idénticos, con lo que se verifica la relación de
conmutación.

\subsection{Significado como Observables Físicos}

Estos operadores son hermíticos, lo que significa que corresponden a observables
físicos medibles.

\begin{itemize}
  \item \textbf{Observable $Z_L$:} El operador $Z_L$ es diagonal en la base
    lógica. Sus autoestados son $|0_L\rangle$ (con autovalor $+1$) y
    $|1_L\rangle$ (con autovalor $-1$). Como vimos en la Sección 3, medir en
    esta base $\{|0_L\rangle, |1_L\rangle\}$ equivale a medir la energía del
    sistema (distinguiendo entre $E_1$ y $E_2$).
  \item \textbf{Observable $X_L$:} El operador $X_L$ no es diagonal.
    Físicamente, representa transiciones (o ``bit-flips'') entre los estados
    base ($X_L|0_L\rangle = |1_L\rangle$). Como observable, sus autoestados son
    los estados de superposición $|+_L\rangle = \frac{1}{\sqrt{2}}(|0_L\rangle +
    |1_L\rangle)$ y $|-_L\rangle = \frac{1}{\sqrt{2}}(|0_L\rangle -
    |1_L\rangle)$. Medir $X_L$ significa medir en esta base de superposición, lo
    cual es físicamente distinto a medir la energía.
  \item \textbf{Observable $Y_L$:} Similarmente, $Y_L$ representa transiciones
    con un desfase. Como observable, sus autoestados son
    $\frac{1}{\sqrt{2}}(|0_L\rangle + i|1_L\rangle)$ y
    $\frac{1}{\sqrt{2}}(|0_L\rangle - i|1_L\rangle)$. Medir $Y_L$ es medir en
    una base diferente a $Z_L$ y $X_L$.
\end{itemize}

El hecho de que no conmuten (ej. $[X_L, Y_L] \neq 0$) es la manifestación
matemática del Principio de Incertidumbre de Heisenberg: no es posible conocer
simultáneamente con precisión el valor de $X_L$ y $Y_L$. Si se mide $Z_L$
(energía) con certeza, el sistema está en $|0_L\rangle$ o $|1_L\rangle$, estados
en los que $X_L$ y $Y_L$ están en completa incertidumbre (tienen 50\% de
probabilidad para cada resultado).
