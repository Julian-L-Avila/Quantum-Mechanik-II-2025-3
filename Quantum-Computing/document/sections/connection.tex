\section{Conexión con la Ecuación de Schrödinger y Dinámica de Puertas}

\subsection{Ecuación de Schrödinger en el Subespacio Lógico}

La dinámica de cualquier estado cuántico $\ket{\psi(t)}$ está gobernada por
la ecuación de Schrödinger dependiente del tiempo:
%
$$
i\hbar \pdv_{t}\ket{\psi(t)} = \hat{H} \ket{\psi(t)}
$$
%
donde $\hat{H}$ es el Hamiltoniano total del sistema.

Nos interesa la dinámica \emph{restringida} al subespacio lógico
$\mathcal{H}_L = \text{span}\{\ket{0_L}, \ket{1_L}\}$.
Suponiendo que el sistema permanece confinado a este subespacio (es
decir, ignorando la fuga (leakage) a estados $n \ge 3$ ),
podemos proyectar el Hamiltoniano físico sobre esta base.

El Hamiltoniano de la partícula libre (sin perturbaciones externas) en el
pozo es $\hat{H}_0 = -\frac{\hbar^2}{2m}D_x^2 + V(x)$.
Los estados $\ket{0_L}$ y $\ket{1_L}$ son, por definición, autoestados
de $\hat{H}_0$ con autovalores $E_1$ y $E_2$ respectivamente
.

El Hamiltoniano efectivo para el sistema libre, $\hat{H}_{\text{free}}$,
actuando solo dentro de $\mathcal{H}_L$, se obtiene proyectando $\hat{H}_0$:
%
$$
\hat{H}_{\text{free}} = \hat{P}_L \hat{H}_0 \hat{P}_L
$$
%
donde $\hat{P}_L = \dyad{0_L} + \dyad{1_L}$ es el proyector sobre
$\mathcal{H}_L$. Los elementos de matriz de este Hamiltoniano son:
%
$$
\mel{j_L}{\hat{H}_0}{k_L} = E_k \braket{j_L}{k_L} = E_k \delta_{jk}
\quad \text{para } j,k \in \{0, 1\}
$$
%
Esto se debe a la ortonormalidad de la base. Por lo
tanto, el Hamiltoniano efectivo es diagonal en la base lógica:
%
$$
\hat{H}_{\text{free}} = E_1 \dyad{0_L} + E_2 \dyad{1_L}
$$
%
Y la ecuación de Schrödinger para un estado $\ket{\psi(t)} \in \mathcal{H}_L$
se convierte en:
%
$$
i\hbar \pdv{t}\ket{\psi(t)} = (E_1 \dyad{0_L} + E_2 \dyad{1_L}) \ket{\psi(t)}
$$

\subsection{Hamiltoniano Efectivo y Rotación en la Esfera de Bloch}

Podemos reescribir $\hat{H}_{\text{free}}$ en términos de los operadores
de Pauli $Z_L$ e $I_L$, usando las definiciones $Z_L = \dyad{0_L} - \dyad{1_L}$
 y $I_L = \dyad{0_L} + \dyad{1_L}$.
%
$$
\dyad{0_L} = \frac{1}{2}(I_L + Z_L) \quad \text{y} \quad
\dyad{1_L} = \frac{1}{2}(I_L - Z_L)
$$
%
Sustituyendo en $\hat{H}_{\text{free}}$:
%
$$
\hat{H}_{\text{free}} = E_1 \frac{1}{2}(I_L + Z_L) + E_2 \frac{1}{2}(I_L - Z_L)
$$
$$
\hat{H}_{\text{free}} = \frac{E_1 + E_2}{2} I_L + \frac{E_1 - E_2}{2} Z_L
$$
%
El operador de evolución temporal $U(t) = \exp(-i \hat{H}_{\text{free}} t / \hbar)$
es:
%
$$
U(t) = \exp\left( \frac{-i (E_1 + E_2) t}{2\hbar} \right) I_L 
       \exp\left( \frac{-i (E_1 - E_2) t}{2\hbar} Z_L \right)
$$
%
El primer término, $\exp(-i (E_1 + E_2) t / 2\hbar) I_L$, es una
\textbf{fase global}. No tiene consecuencias físicas medibles, ya que
el vector de estado en la esfera de Bloch es invariante bajo fases
globales.

El término físicamente relevante es el segundo. Definiendo la frecuencia
de transición (o frecuencia de Larmor del qubit) como
$\omega_{12} = (E_2 - E_1) / \hbar$, podemos reescribir el Hamiltoniano
físicamente relevante (ignorando la fase global) como:
%
$$
\hat{H}_{\text{eff}} = \frac{E_1 - E_2}{2} Z_L = -\frac{\hbar \omega_{12}}{2} Z_L
$$
%
El operador de evolución temporal (sin fase global) es:
%
$$
U_{\text{eff}}(t) = \exp\left( -i \hat{H}_{\text{eff}} t / \hbar \right) =
\exp\left( i \frac{\omega_{12} t}{2} Z_L \right)
$$
%
Este operador es, por definición, el operador de rotación alrededor
del eje $z$ de la esfera de Bloch por un ángulo $\theta_Z(t) = -\omega_{12} t$.
%
$$
U_{\text{eff}}(t) = R_Z(-\omega_{12} t)
$$
%
Esto demuestra que la \textbf{evolución libre} del qubit (cuando se deja
sin perturbar) es simplemente una rotación continua alrededor del eje $z$
a una velocidad constante $\omega_{12}$. Las poblaciones $|\alpha|^2$ y
$|\beta|^2$ (probabilidades de medir $E_1$ o $E_2$) permanecen constantes,
mientras que la fase relativa entre los dos estados evoluciona en el
tiempo. Esto es coherente con el hecho de que $\ket{0_L}$ y $\ket{1_L}$
son estados estacionarios (autoestados de energía).

\subsection{Interpretación como Puerta Cuántica}

La dinámica descrita por $U_{\text{eff}}(t)$ es en sí misma una
\textbf{puerta cuántica}. Si dejamos el sistema evolucionar libremente
durante un tiempo $t = \phi / (-\omega_{12})$, implementamos la
puerta $R_Z(\phi)$. Esta es una puerta de fase.

Sin embargo, para la computación cuántica universal, se requiere la
capacidad de implementar \emph{rotaciones arbitrarias} en la esfera de
Bloch, no solo rotaciones $R_Z$. Específicamente, se necesita control
sobre al menos un eje de rotación adicional (por ejemplo, $X_L$).

Esto se logra aplicando un Hamiltoniano de interacción externo
dependiente del tiempo, $\hat{H}_{\text{int}}(t)$. Como se discutió
previamente, la carga $q$ y el espín $s$ de la partícula proporcionan
los ``manejadores'' para este control.

Por ejemplo, la aplicación de un campo eléctrico externo $\vec{E}(t)$
 puede modular el potencial $V(x)$. Si este campo oscila a una
frecuencia $\omega$ cercana a la frecuencia de resonancia $\omega_{12}$,
inducirá transiciones entre $\ket{0_L}$ y $\ket{1_L}$. En el marco de la
aproximación de onda rotante (RWA), este Hamiltoniano de interacción,
visto en el marco giratorio, puede tomar la forma de:
%
$$
\hat{H}_{\text{RWA}} \propto \Omega X_L
$$
%
donde $\Omega$ es la frecuencia de Rabi, que es proporcional a la
amplitud del campo $\vec{E}$ aplicado.

La evolución bajo este Hamiltoniano, $U_X(t) = \exp(-i (\Omega' t) X_L)$,
corresponde a una \textbf{rotación alrededor del eje $x$}
(una puerta $R_X$).

Al tener la capacidad de generar rotaciones $R_Z$ (mediante la evolución
libre controlada por tiempos de espera) y rotaciones $R_X$ (mediante la
aplicación de campos resonantes), se puede construir cualquier
operación unitaria $U \in SU(2)$ (cualquier puerta de un solo qubit)
mediante la composición de estas rotaciones (por ejemplo, $U = R_Z(\phi_1)
R_X(\theta) R_Z(\phi_2)$).

Por lo tanto, la ``dinámica de puertas'' es la implementación de
operaciones $SU(2)$ deseadas mediante la conmutación controlada y
temporizada de Hamiltonianos de evolución libre ($\propto Z_L$) e
interacción ($\propto X_L$ o $Y_L$).
