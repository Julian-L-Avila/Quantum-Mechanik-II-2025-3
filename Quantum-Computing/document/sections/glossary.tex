\section{Glosario}

A continuación, se definen los conceptos clave introducidos en el desarrollo del
trabajo.

\begin{description}
  \item[\textbf{Espacio de Hilbert ($\mathcal{H}$):}]
    Es un espacio vectorial complejo (donde viven los kets, $|\psi\rangle$)
    dotado de un producto interno, que además es completo. En mecánica cuántica,
    el conjunto de todos los posibles estados de un sistema físico forma un
    espacio de Hilbert.

  \item[\textbf{Producto Interno:}]
    Es una función (denotada $\langle \cdot, \cdot \rangle$) que toma dos
    vectores de un espacio vectorial $V$ y devuelve un escalar en el campo $F$
    sobre el cual $V$ está definido (usualmente $\mathbb{R}$ o $\mathbb{C}$).
    Esta función debe satisfacer propiedades específicas: ser positiva-definida,
    tener simetría conjugada ($\langle v, w \rangle = \langle w, v \rangle^*$) y
    ser lineal en uno de sus argumentos (en mecánica cuántica, por convención,
    es lineal en el segundo argumento, el ket). En el contexto de un espacio de
    Hilbert complejo ($\mathbb{C}$), el producto interno
    $\langle\phi|\psi\rangle$ devuelve un número complejo. Su propósito es
    definir la geometría del espacio, permitiendo calcular normas ($\|\psi\|^2 =
    \langle\psi|\psi\rangle$), probabilidades y verificar la ortogonalidad
    ($\langle\phi|\psi\rangle = 0$).

  \item[\textbf{Estado Propio (Autoestado):}]
    Un vector $|\psi\rangle$ (distinto de cero) que, al ser actuado por un
    operador $\hat{A}$, no cambia su ``dirección'' en el espacio de Hilbert,
    sino que solo es multiplicado por un escalar $a$, llamado autovalor.
    Satisface la ecuación de autovalores $\hat{A}|\psi\rangle = a|\psi\rangle$.
    Los estados estacionarios $\psi_n$ son autoestados del Hamiltoniano
    $\hat{H}$.

  \item[\textbf{Superposición Lineal:}]
    Matemáticamente, es un sinónimo de combinación lineal. Es una expresión
    formada por la suma de vectores de un espacio vectorial (como los kets
    $|\psi_1\rangle, |\psi_2\rangle$), donde cada vector es multiplicado por un
    coeficiente escalar (como $\alpha, \beta \in \mathbb{C}$). La forma general
    es $|\Psi\rangle = \alpha|\psi_1\rangle + \beta|\psi_2\rangle$. El Principio
    de Superposición es el postulado fundamental de la mecánica cuántica que
    establece que si $|\psi_1\rangle$ y $|\psi_2\rangle$ son estados válidos de
    un sistema, cualquier superposición lineal de ellos (con $|\alpha|^2 +
    |\beta|^2 = 1$) también es un estado físico válido, que existe como una
    combinación de ambos hasta que se realiza una medición.

  \item[\textbf{Base Lógica:}]
    Un conjunto de estados ortonormales ($\{|0_L\rangle, |1_L\rangle\}$)
    elegidos de un espacio de Hilbert para representar bits de información. Se
    les llama ``lógicos'' o ``computacionales'' porque abstraen la física
    subyacente (en este caso, los niveles de energía) a los conceptos de 0 y 1.

  \item[\textbf{Qubit (Qubit Lógico):}]
    La unidad básica de información cuántica. Es un sistema cuántico de dos
    niveles (como los dos estados de energía $\psi_1, \psi_2$) cuya base lógica
    es $|0_L\rangle$ y $|1_L\rangle$. Su estado general es una superposición
    lineal $|\psi\rangle = \alpha|0_L\rangle + \beta|1_L\rangle$.

  \item[\textbf{Observable:}]
    Cualquier propiedad física de un sistema que, en principio, puede ser medida
    (ej. energía, posición, momento, espín). En el formalismo cuántico, cada
    observable está representado por un operador hermítico. Los resultados de
    una medición solo pueden ser los autovalores de dicho operador.

  \item[\textbf{Operador Hermítico:}]
    Un operador $\hat{A}$ que es igual a su propio adjunto (conjugado
    transpuesto), es decir, $\hat{A} = \hat{A}^\dagger$. Son fundamentales
    porque garantizan dos cosas: 1) Sus autovalores son siempre números reales
    (lo cual es necesario, ya que las mediciones dan resultados reales). 2) Sus
    autoestados forman una base ortonormal para el espacio de Hilbert.

  \item[\textbf{Producto Tensorial ($\otimes$):}]
    Es una operación algebraica que construye un nuevo espacio vectorial
    $\mathcal{H}_{AB} = \mathcal{H}_A \otimes \mathcal{H}_B$ a partir de dos
    espacios $\mathcal{H}_A$ y $\mathcal{H}_B$. Si $\dim(\mathcal{H}_A) = n$ y
    $\dim(\mathcal{H}_B) = m$, entonces $\dim(\mathcal{H}_{AB}) = n \times m$.
    Físicamente, es la operación matemática que permite construir el espacio de
    Hilbert de un sistema compuesto a partir de los espacios de sus subsistemas
    (ej. $\mathcal{H}_{total} = \mathcal{H}_L \otimes \mathcal{H}_{spin}$). Un
    estado en este espacio combinado especifica el estado de todas sus partes
    simultáneamente y permite la existencia del entrelazamiento.

  \item[\textbf{Espín:}]
    Una forma intrínseca de momento angular que posee una partícula (como el
    electrón). Es una propiedad puramente cuántica, sin un análogo clásico
    directo, que se describe como un grado de libertad interno de la partícula
    (ej. ``arriba'' $|\uparrow\rangle$ y ``abajo'' $|\downarrow\rangle$ para una
    partícula de espín $s=1/2$).

  \item[\textbf{Operadores de Pauli:}]
    Un conjunto de tres operadores hermíticos ($X_L, Y_L, Z_L$) que son
    fundamentales para describir la dinámica y los observables de un sistema de
    dos niveles (qubit). Físicamente, están relacionados con las mediciones del
    espín en las tres direcciones espaciales (x, y, z) y matemáticamente forman
    una base para los operadores del qubit.

  \item[\textbf{Entrelazamiento:}]
    Un fenómeno puramente cuántico que describe una correlación entre dos o más
    sistemas (como qubits) que no puede existir en la física clásica. Un sistema
    compuesto se dice entrelazado si su estado cuántico total no puede ser
    factorizado o descrito como una simple combinación de los estados
    individuales de sus subsistemas (es decir, no es un estado separable). Como
    resultado, una medición realizada sobre uno de los sistemas parece influir
    instantáneamente en el resultado de la medición sobre el otro, sin importar
    la distancia que los separe.

  \item[\textbf{Regla de Born:}] Postulado fundamental que conecta el formalismo
    matemático del vector de estado $\ket{\psi}$ con los resultados de una
    medición. Si un sistema se mide en una base ortonormal $\{\ket{\phi_i}\}$,
    la probabilidad de obtener el resultado asociado al autoestado $\ket{\phi_i}$
    es $P(i) = |\braket{\phi_i}{\psi}|^2$.

  \item[\textbf{Esfera de Bloch:}] Representación geométrica del espacio de estados
    puros de un qubit (un sistema de dos niveles) $\mathcal{H}_L$. Cada
    estado $\ket{\psi} = \cos(\theta/2)\ket{0_L} + e^{i\phi}\sin(\theta/2)\ket{1_L}$
    se mapea a un punto único en la superficie de una esfera de radio unidad
    mediante los ángulos $(\theta, \phi)$. Los estados ortogonales ocupan puntos
    antipodales.

  \item[\textbf{Puerta cuántica:}] Operación fundamental que transforma el estado de
    un qubit. Matemáticamente, se describe como una \emph{rotación unitaria} $U$
    (un operador en $SU(2)$) que actúa sobre el vector de estado en la esfera
    de Bloch. Físicamente, se implementa mediante la evolución temporal
    $U(t) = \exp(-i\hat{H}_{\text{eff}}t/\hbar)$ bajo un Hamiltoniano efectivo
    $\hat{H}_{\text{eff}}$ aplicado de forma controlada.

  \item[\textbf{Medida computacional:}] Proceso de medición de un qubit en la base
    lógica (o computacional) $\{\ket{0_L}, \ket{1_L}\}$. Corresponde a medir el
    observable $Z_L = \dyad{0_L} - \dyad{1_L}$. En el sistema físico del pozo
    de potencial, esta medición es físicamente equivalente a medir la energía
    (distinguiendo entre $E_1$ y $E_2$).

  \item[\textbf{Hamiltoniano efectivo:}] Operador que describe la dinámica de un
    sistema restringido a un subespacio de interés (como el subespacio lógico
    $\mathcal{H}_L$). Se obtiene proyectando el Hamiltoniano físico completo
    $\hat{H}$ sobre dicho subespacio ($\hat{H}_{\text{eff}} =
    \hat{P}_L \hat{H} \hat{P}_L$) y, comúnmente, omitiendo términos que solo
    contribuyen a una fase global, como los proporcionales al operador
    identidad $I_L$.

  \item[\textbf{Rotación unitaria:}] Transformación lineal $U$ en un espacio de Hilbert
    que preserva el producto interno (y, por tanto, la norma de los vectores y
    la probabilidad total). Satisface la condición $U^\dagger U = I$. Según la
    ecuación de Schrödinger, la evolución temporal de cualquier sistema cuántico
    cerrado es una rotación unitaria. Las puertas cuánticas son
    implementaciones físicas de rotaciones unitarias específicas.

  \item[\textbf{Subespacio invariante:}] Subespacio vectorial
    $\mathcal{S} \subseteq \mathcal{H}$ tal que, al aplicar un operador $\hat{A}$,
    todos los vectores de $\mathcal{S}$ se transforman en vectores que también
    pertenecen a $\mathcal{S}$ (es decir, $\hat{A}\ket{\psi} \in \mathcal{S}$
    para todo $\ket{\psi} \in \mathcal{S}$). En el diseño de un qubit lógico,
    el objetivo es que el subespacio $\mathcal{H}_L$ sea invariante bajo la
    evolución temporal $U(t)$ de las puertas, evitando la fuga (leakage) a
    estados fuera del subespacio computacional.

  \item[\textbf{Coherencia cuántica:}] Propiedad de un sistema cuántico para mantener
    una superposición de estados con una relación de fase relativa definida,
    como en $\ket{\psi} = \alpha\ket{0_L} + \beta\ket{1_L}$.
    La coherencia es la responsable de los términos de interferencia.
    Matemáticamente, la coherencia se cuantifica por los elementos fuera de la
    diagonal de la matriz de densidad $\rho$. La pérdida de esta relación de fase
    (decoherencia) destruye la superposición y la interferencia.
\end{description}
