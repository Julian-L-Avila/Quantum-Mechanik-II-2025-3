\section{Inclusión del espín y carga}

\subsection{Espacio de estados total (Producto Tensorial)}

El estado de la partícula no solo se describe por su posición (o nivel de
energía), sino también por su espín. Para describir el estado completo, debemos
combinar el espacio de estados de la parte espacial (nuestro qubit lógico) con
el espacio de estados del espín.

El espacio de estados total $\mathcal{H}_{total}$ es el producto tensorial del
subespacio lógico $\mathcal{H}_L$ (al que nos hemos restringido) y el espacio de
espín $\mathcal{H}_{spin}$:
\[
  \mathcal{H}_{total} = \mathcal{H}_{L} \otimes \mathcal{H}_{spin}
\]
Donde:
\begin{itemize}
  \item $\mathcal{H}_L = \text{span}\{|0_L\rangle, |1_L\rangle\}$ es el espacio
    de 2 dimensiones del qubit lógico (energía).
  \item $\mathcal{H}_{spin} = \text{span}\{|\uparrow\rangle,
    |\downarrow\rangle\}$ es el espacio de 2 dimensiones del espín $s=1/2$.
\end{itemize}
El espacio total resultante $\mathcal{H}_{total}$ es, por lo tanto, un espacio
de $2 \times 2 = 4$ dimensiones. La base de este espacio está formada por todas
las combinaciones posibles:
\[
  \{ |0_L\rangle \otimes |\uparrow\rangle, \quad |0_L\rangle \otimes
    |\downarrow\rangle, \quad |1_L\rangle \otimes |\uparrow\rangle, \quad
  |1_L\rangle \otimes |\downarrow\rangle \}
\]

A menudo, esto se abrevia como $\{ |0_L \uparrow\rangle, |0_L \downarrow\rangle,
|1_L \uparrow\rangle, |1_L \downarrow\rangle \}$

\subsection{Significado físico del Producto Tensorial}

El producto tensorial $\otimes$ es la herramienta matemática que nos permite
describir estados de sistemas compuestos o de un solo sistema con múltiples
grados de libertad (como en este caso, posición y espín).

\textbf{Significado Físico:} Un estado en $\mathcal{H}_{total}$ debe especificar
\textit{simultáneamente} la información de ambas partes. El producto tensorial
nos dice que el estado de la partícula es una combinación (ya sea separable o
entrelazada) de un estado en $\mathcal{H}_L$ y un estado en
$\mathcal{H}_{spin}$.

\textbf{Ejemplo Explícito:}
El estado que se propone es:
\[
  |\Psi\rangle = |0_L\rangle \otimes | \uparrow\rangle
\]
Este es un \textit{estado separable}. Su interpretación física es clara:
\begin{itemize}
  \item La partícula se encuentra en el estado de energía $E_1$ (correspondiente
    al estado lógico $|0_L\rangle$).
  \item Y, al mismo tiempo, su espín está en el estado ``arriba''
    ($|\uparrow\rangle$).
\end{itemize}
Un estado general en este espacio será una superposición de los 4 estados base,
lo que permite también la existencia de \textit{estados entrelazados} (no
separables) entre el espacio y el espín, como por ejemplo
$\frac{1}{\sqrt{2}}(|0_L \uparrow\rangle + |1_L \downarrow\rangle)$.

\subsection{Nuevos grados de libertad y efectos de la carga}

\subsubsection*{Nuevos Grados de Libertad (Espín)}
El espín añade un nuevo grado de libertad binario. Antes, nuestro sistema (el
qubit lógico $\mathcal{H}_L$) era un sistema de 2 niveles. Al incluir el espín,
el sistema completo $\mathcal{H}_{total}$ se convierte en un sistema de 4
niveles.

Esta inclusión tiene implicaciones físicas significativas:
\begin{enumerate}
  \item \textbf{Más información:} El sistema ahora puede almacenar dos qubits de
    información (uno en la energía/posición y otro en el espín).
  \item \textbf{Nuevas interacciones:} Se pueden diseñar interacciones que
    actúen selectivamente sobre la parte espacial o sobre la parte de espín.
  \item \textbf{Entrelazamiento:} Como se mencionó, permite la creación de
    entrelazamiento entre las propiedades espaciales y de espín de la
    \textit{misma} partícula.
\end{enumerate}

\subsubsection*{Efectos de la Carga Eléctrica ($q$)}
La carga y el espín hacen que la partícula sea sensible a los campos
electromagnéticos, lo cual es fundamental para su \textit{control}:
\begin{itemize}
  \item \textbf{Efecto de un Campo Eléctrico ($\vec{E}$):} La carga $q$
    interactuará con un campo eléctrico (p.ej., $\hat{V}_{E} =
    -q\vec{E}\cdot\vec{x}$). Un campo eléctrico puede usarse para modular el
    potencial del pozo $V(x)$, por ejemplo, ``inclinándolo''. Esto modificaría
    las funciones de onda $\psi_1, \psi_2$ y sus energías $E_1, E_2$. Un campo
    eléctrico permitiría, por tanto, controlar el qubit lógico espacial (p.ej.,
    inducir transiciones entre $|0_L\rangle$ y $|1_L\rangle$).

  \item \textbf{Efecto de un Campo Magnético ($\vec{B}$):} La partícula, al
    tener espín $s=1/2$, posee un momento dipolar magnético $\vec{\mu}$. Este
    momento magnético interactuará con un campo magnético externo $\vec{B}$ a
    través del término de Zeeman ($\hat{H}_{Z} \approx -\vec{\mu} \cdot
    \vec{B}$). Esta interacción afecta principalmente a los estados de espín
    ($|\uparrow\rangle$ y $|\downarrow\rangle$), rompiendo su degeneración en
    energía. Por lo tanto, un campo magnético (especialmente uno oscilante)
    permitiría controlar el qubit de espín.
\end{itemize}
